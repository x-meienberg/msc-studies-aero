\documentclass[8pt, landscape, fleqn]{scrartcl}
\setlength{\parindent}{0pt}
\usepackage[ngerman]{babel}
\usepackage[applemac]{inputenc}
\usepackage[dvips]{geometry}
\usepackage{latexsym}
\usepackage{multicol}
\usepackage{amsmath}
\usepackage{graphicx}
\usepackage{array}
\usepackage{booktabs}
\usepackage{amsmath}
\usepackage{mathtools}
\usepackage{ulem}
\usepackage{amsfonts}
\usepackage{dsfont}
\usepackage{charter} %%% Schreibart
%\renewcommand{\familydefault}{\sfdefault}



%%%%%%%%%%Paket für Chemische Formeln
\usepackage{chemformula} 
\usepackage[version=3]{mhchem}
%%%%%%%%%%%%%%%%% Farbe
\usepackage{color}

\pagestyle{plain}
\typearea{20}
\columnsep 30pt
\columnseprule .4pt
\setlength{\extrarowheight}{0.9em}

\renewcommand{\arraystretch}{0.8}

\makeatletter
\renewcommand{\section}{\@startsection{section}{1}{0mm}%
{-2\baselineskip}{0.8\baselineskip}%
{\hrule depth 0.2pt width\columnwidth\hrule depth1.5pt
width0.25\columnwidth\vspace*{1.2em}\Large\bfseries\rmfamily}}
\makeatother


\makeatletter
\renewcommand{\subsection}{\@startsection{subsection}{1}{0mm}%
{-2\baselineskip}{0.8\baselineskip}%
{\hrule depth 0.2pt width\columnwidth\hrule depth0.75pt
width0.25\columnwidth\vspace*{1.2em}\large\bfseries\rmfamily}}
\makeatother

\makeatletter
\renewcommand{\subsubsection}{\@startsection{subsubsection}{1}{0mm}%
{-2\baselineskip}{0.8\baselineskip}%
{\hrule depth 0.2pt width\columnwidth\vspace*{1.2em}\normalsize\bfseries\rmfamily}}
\makeatother

\newcommand{\Mx}[1]{\begin{bmatrix}#1\end{bmatrix}}
\begin{document}
\part*{\LARGE\textrm{Aeroelasticity $\hfill$ Xeno Meienberg}}
\begin{multicols*}{3}

\section*{Constants}
\begin{itemize}
\item Normdruck: $p_{ref} = 1 atm = 1.01325 \: bar$
\item Normtemperatur: $T_{ref} = 298 \: K \approx 25^{\circ} \: C$
\item Pferdest"arke:  $ 1 \: hp = 1 \: PS = 0.735 \: kW$
\item Elementarladung:  $e=1.60219 \cdot 10^{-19} \:C$
\item Faraday-Konstante: $F= N_A \cdot e = 96485.3 \frac{C}{mol} = \frac{A\cdot s}{mol}$
\item ppm = parts per million: $1 \: ppm = 10^{-6}$
\item Gaskonstante: $ \overline{R} = 8.314 \frac{J}{mol K}$, spez. -- $R = \frac{\overline{R}}{M} [\frac{J}{kg K}]$
\end{itemize}


\section*{Parameters}
\begin{itemize}
\item Aerodynamic Force $F_A$
\item Aerodynamic Moment $M_A$
\item Lift Coefficient $C_{l}=L/(1/2\rho V^2c) $
\item Drag Coefficient $C_d = D/(1/2\rho V^2c)$
\item Moment Coefficient $C_m = M_A/(1/2\rho V^2 c^2)$
\item Angle of Attack $\alpha$ angle between connection leading and the trailing edge and reference line
\item Lift curve slope $a = C_l / \alpha $
\end{itemize}

\section*{Steady Aerofoil and Wing Section Aerodynamics}

\begin{itemize}
    \item Aerofoil = 2-D wing section with goal to generate lift force perpendicular to the relative airspeed
    \item Convention: Lift is up, Drag is in direction of windspeed and Aerodynamic moment in clockwise direction acting on the aerodynamic center. Aerodynamic center is normally at the quarter chord position $c_{m,c/4}$ for syymetric airfoils. $x_{ac} = -m_0/2\pi +0.25$ with $m_0$ as a shape constant 
    \item Further assumptions: No viscosity, incompressible fluid, $Ma < 0.2,0.3$, no vortices, potential flow (Navier-Stokes)
    \item Another centre is the shear center (elastic axis) from mechanics
    \item $L = 1/2\rho V^2 c a \alpha$, with $a$ from tables (CFD and Wind Tunnel)
    \item $M_a = 1/2\rho V^2c^2 c_{m \phi}$ with $c_{m \phi}$ also from tables
\end{itemize}

\subsubsection*{Lift curve $C_l(\alpha)$ and drag curve $C_d(\alpha)$}

\begin{itemize}
    \item At small ranges of $\alpha$, both lift and drag increase with: $C_l \propto \alpha$ and $C_d \propto \alpha^2$
    \item In aeroelasticity and this course, $\alpha$ will be very small, hence drag will be negligble small
\end{itemize}

\subsubsection*{The aerodynamic moment $M_A$}

\begin{itemize}
    \item The aerodynamic moment is much more important than drag $C_d$
    \item $M_A$ varies with $\alpha$ in the small ranges of the angle of attack 
    \item {\bf Important to note:} There exist a point at which the aerodynamic moment does not depend on $\alpha$. This is the the aerodynamic centre
    \item The aerodynamic centre is not the same as the centre of pressure, which is defined as the point where the aerodynamic moment is zero given a certain angle of attack $\alpha$
    \item Symmetric airfoils at $\alpha=0$ have no aerodynamic moment at all times ($M_A = 0 = const$). At the aerodynamic centre for symmetric foils results into no moment
    \item Asymmetric airfoils at $\alpha=0$ have a non-zero aerodynamic moment at all times (all angles $\alpha$)
\end{itemize}


\subsubsection*{Assessment of $C_l/\alpha$}

\begin{itemize}
    \item The linear part of the lift curve is characterised by the slope $a = C_l / \alpha(M) = \frac{C_l/\alpha_{M=0}}{\sqrt{1-M^2}}$
    \item The Prandtl-Glauert factor is $1/\sqrt{1-M^2}$
    \item The factor is depending on the Mach number. The slop increases with increasing $M$ (between $0$ and $1$)
    \item The dependence on $Re$ is more subtle (p. 8)
\end{itemize}    

\subsection*{Extension to wing aerodynamics (p. 8)}

Aerofoil dynamics (2D) refer to the previous topics, however the 3-D case can be also modeled by through a couple examples. A finite wing is less stable and efficient than the airfoil since the tips have vortices
on at the wing tips. These ``induce'' a velocity, which locally reduces the angle of attack. An important parameter is the so called {\bf{Aspect Ratio} $AR = b^2/S$}. 
If the wing is assumed to be of surface $S = b\cdot c$, it follows $AR = b/c$.

\begin{itemize}
    \item The lift curve can become a function of $AR$ if due to the different tips. Approximately, the lift slope $a_0$ is adjusted via 
    following formula: 
    \item $a = a_0 \frac{AR}{AR+4}$
\end{itemize}

\subsubsection*{Strip Theory (p.9)}

\begin{itemize}
    \item If $AS$ is very small (delta wings), the integral of multiple airfoils
    \item Example, the wing is an elliptical $f(y) = \sqrt{1-(\frac{y}{b/2})^2} \cdot \overline{f}_{\phi}$
    \item $f(y) = a \alpha c = C_l c$ with $c = $ chord length. 
\end{itemize}

\section*{Steady-state (static) Aeroelasticity}
\subsection*{Typical Section}
2-D problem with a rigid wing and 2 degrees of freedom (free rotation / pitch and plunge). We can have multiple typical sections. The pitch is modeled via a torsional string and the plunge via a longitudinal one.
The idea is later on to model the torsional spring to be a torsional stiffness of a beam (since a real wing is actually a beam with a certain stiffness).

\begin{itemize}
    \item The torsion acts in a beam section on the shear centre, however in aeroelasticity on the elastic axis
    \item The goal of engineering is alwayt to move the shear center to the front (comes with risk to thin out
    the rear longeron and thicken the front longeron)
    \item In equilibrium, we know that the aerodynamic forces are equal to the spring forces
    \item $M_t + L \cdot e = \theta k_\theta = (q c^2 c_{m0}+ qca\theta e)\cdot b$
    \item 
\end{itemize}


\end{multicols*}
\end{document}

