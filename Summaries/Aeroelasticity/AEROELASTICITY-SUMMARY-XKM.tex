\documentclass[8pt, landscape, fleqn]{scrartcl}
\setlength{\parindent}{0pt}
\usepackage[ngerman]{babel}
\usepackage[applemac]{inputenc}
\usepackage[dvips]{geometry}
\usepackage{latexsym}
\usepackage{multicol}
\usepackage{amsmath}
\usepackage{graphicx}
\usepackage{array}
\usepackage{booktabs}
\usepackage{amsmath}
\usepackage{mathtools}
\usepackage{ulem}
\usepackage{amsfonts}
\usepackage{dsfont}
\usepackage{charter} %%% Schreibart
%\renewcommand{\familydefault}{\sfdefault}



%%%%%%%%%%Paket für Chemische Formeln
\usepackage{chemformula} 
\usepackage[version=3]{mhchem}
%%%%%%%%%%%%%%%%% Farbe
\usepackage{color}

\pagestyle{plain}
\typearea{20}
\columnsep 30pt
\columnseprule .4pt
\setlength{\extrarowheight}{0.9em}

\renewcommand{\arraystretch}{0.8}

\makeatletter
\renewcommand{\section}{\@startsection{section}{1}{0mm}%
{-2\baselineskip}{0.8\baselineskip}%
{\hrule depth 0.2pt width\columnwidth\hrule depth1.5pt
width0.25\columnwidth\vspace*{1.2em}\Large\bfseries\rmfamily}}
\makeatother


\makeatletter
\renewcommand{\subsection}{\@startsection{subsection}{1}{0mm}%
{-2\baselineskip}{0.8\baselineskip}%
{\hrule depth 0.2pt width\columnwidth\hrule depth0.75pt
width0.25\columnwidth\vspace*{1.2em}\large\bfseries\rmfamily}}
\makeatother

\makeatletter
\renewcommand{\subsubsection}{\@startsection{subsubsection}{1}{0mm}%
{-2\baselineskip}{0.8\baselineskip}%
{\hrule depth 0.2pt width\columnwidth\vspace*{1.2em}\normalsize\bfseries\rmfamily}}
\makeatother

\newcommand{\Mx}[1]{\begin{bmatrix}#1\end{bmatrix}}
\begin{document}
\part*{\LARGE\textrm{Aeroelasticity $\hfill$ Xeno Meienberg}}
\begin{multicols*}{3}

\subsection*{Parameters}
\begin{itemize}
\item Aerodynamic Force $\overline{F_A} = \overline{L}+\overline{D}$
\item Lift Force $L = \frac{1}{2}\rho V^2 c a \alpha = qca\alpha$ $[N/m]$
\item Drag Force $D$ $[N]$
\item Aerodynamic Moment $M_A = \frac{1}{2}\rho V^2 c^2 c_{m0} = qc^2c_{m0}$ $[N]$
\item Dynamic Pressure $q = 1/2 \rho V^2$ (Bernoulli) $[Pa]$
\item Chord Length $c$ $[m]$
\item Surface Area $S = b\cdot c$ $[m^2]$ (Rectangular)
\item Wing Span $b$ $[m]$ 
\item Lift Coefficient $C_{l}=L/(1/2\rho V^2\cdot S) $
\item Drag Coefficient $C_d = D/(1/2\rho V^2 \cdot S)$
\item Moment Coefficient $C_m = M_A/(1/2\rho V^2 \cdot c \cdot S)$
\item Angle of Attack $\alpha$ $[rad]$ (positive in clockwise direction)
\item Lift curve slope $a = C_{l/\alpha} = C_{l} / \alpha \approx tan(angle\,x-axis\,to\,curve)$
\item Pitch angle $\theta$ (Rotation w.r.t elastic axis)
\item Lunge $h$ (Deflection of elastic axis parallel to lift)
\item Initial (from geometry given) $\alpha_0$ and $c_{m0}$
\end{itemize}

\subsection*{Conventions throughout Course}

\begin{itemize}
    \item If L and D absolute $\rightarrow$ use calculations above
    \item If L and D per span unit $\rightarrow$ correct via dividing by $b$
    \item Sign conventions: Lift positive, Drag positive in x and y direction
    \item Moments and angles positive in clockwise direction
    \item Our system coordinate system is defined by the wing. The angle of attack is defined relative to it
    \item The variables which describe the airfoil motion are the pitch $\theta$ and the plunge $h$ which act at the shear centre of the wing
    \item bending = flapping, in-plane bending = lead-lagging (helicopter vocabulary)
\end{itemize}

\subsection*{Mathematical Basics}

\begin{itemize}
    \item A = $\begin{pmatrix} a & b \\ c & d
    \end{pmatrix}$
    \item Inverse of Matrix (2D): $A^{-1} = \frac{1}{ad-bc} \begin{pmatrix} d & -b \\ -c & a
    \end{pmatrix}$
    \item Inverse of Matrix (3D): 
        \subitem 1. det (A), then transpose A
        \subitem 2. Find the ajunct matrix (minors) of
        \subitem (cover row and column of element) of $A^T$ 
        \subitem and multiply with +- matrix
        \subitem $A^{-1} = 1/det(A)\cdot Adj(A^T)$
    \item The solutions of $A x = 0$ for a matrix $A$, $x$ cannot be just the trivial solution if A is not invertible
    \item $[rad] = \frac{\pi}{180} [deg]$
\end{itemize}



\section*{Steady Aerofoil and Wing Section Aerodynamics}

\begin{itemize}
    \item Aerofoil = 2-D wing section with goal to generate lift force perpendicular to the relative airspeed
    \item Convention: Lift is up, Drag is in direction of windspeed and Aerodynamic moment in clockwise direction acting on the aerodynamic center. Aerodynamic center is normally at the quarter chord position $c_{m,c/4}$ for syymetric airfoils. $x_{ac} = -m_0/2\pi +0.25$ with $m_0$ as a shape constant 
    \item Further assumptions: No viscosity, incompressible fluid, $Ma < 0.2,0.3$, no vortices, potential flow (Navier-Stokes)
    \item Another centre is the shear center (elastic axis) from mechanics
    \item $L = 1/2\rho V^2 c a \alpha$, with $a$ from tables (CFD and Wind Tunnel) $[N/m]$
    \item $M_A = 1/2\rho V^2c^2 c_{m0}$ with $c_{m0}$ also from tables $[N]$
\end{itemize}

\subsubsection*{Lift curve $C_l(\alpha)$ and drag curve $C_d(\alpha)$}

\begin{itemize}
    \item At small ranges of $\alpha$, both lift and drag increase with: $C_l \propto \alpha$ and $C_d \propto \alpha^2$
    \item In aeroelasticity and this course, $\alpha$ will be very small, hence drag will be negligble small
\end{itemize}

\subsubsection*{The aerodynamic moment $M_A$}

\begin{itemize}
    \item The aerodynamic moment is much more important than drag $C_d$
    \item $M_A$ varies with $\alpha$ in the small ranges of the angle of attack (very small, p. 7)
    \item {\bf Important to note:} There exist a point at which the aerodynamic moment does not depend on $\alpha$. This is the the aerodynamic centre
    \item The aerodynamic centre is not the same as the centre of pressure, which is defined as the point where the aerodynamic moment is zero given a certain angle of attack $\alpha$
    \item Symmetric airfoils at $\alpha=0$ have no aerodynamic moment at all times ($M_A = 0 = const$). At the aerodynamic centre for symmetric foils results into no moment
    \item Asymmetric airfoils at $\alpha=0$ have a non-zero aerodynamic moment at all times (all angles $\alpha$)
\end{itemize}


\subsubsection*{Assessment of $C_l/\alpha$ (Correction of value through Mach Number)}

\begin{itemize}
    \item The linear part of the lift curve is characterised by the slope $a = C_l / \alpha(M) = \frac{C_l/\alpha_{M=0}}{\sqrt{1-M^2}}$
    \item The Prandtl-Glauert factor is $1/\sqrt{1-M^2}$
    \item The factor is depending on the Mach number. The slope increases with increasing $M$ (between $0$ and $1$)
    \item The dependence on $Re$ is more subtle (p. 8)
    \item In supersonic regimes, the aerodynamic centre is shifted towards the back (more stability) however $C_{l/\alpha}$ decreases
\end{itemize}    

\subsection*{Extension to wing aerodynamics (p. 8)}

Aerofoil dynamics (2D) refer to the previous topics, however the 3-D case can be also modeled by through a couple examples. A finite wing is less stable and efficient than the airfoil since the tips have vortices
on at the wing tips. These ``induce'' a velocity, which locally reduces the angle of attack. An important parameter is the so called {\bf{Aspect Ratio} $AR = b^2/S$}. 
If the wing is assumed to be of surface $S = b\cdot c$, it follows $AR = b/c$.

\begin{itemize}
    \item The lift curve can become a function of $AR$ if due to the different tips. Approximately, the lift slope $a_0$ is adjusted via 
    following formula: 
    \item $a = a_0 \frac{AR}{AR+4}$
    \item The values $a$ and $c_{m0}$ will hence be corrected with a $a^*$ and $c_{m0}^*$
\end{itemize}

\subsubsection*{Strip Theory (p.9)}

\begin{itemize}
    \item If $AS$ is very small (delta wings), the integral of multiple airfoils
    \item Define multiple airfoils stacked next to each other along the span $b$
    \item Example, the wing is an elliptical $f(y) = \sqrt{1-(\frac{y}{b/2})^2} \cdot \overline{f}_{\phi}$
    \item $f(y) = a \alpha c = C_l c$ with $c = $ chord length. 
\end{itemize}

\section*{Steady-state (static) Aeroelasticity}
\subsection*{Typical Section = 1DOF model}
2-D problem with a rigid wing. We can have multiple typical sections stacked onto each other, which would be later adding dimensionality to the variable $\theta$.
The idea is later on to model the torsional spring to be a torsional stiffness of a beam (since a real wing is actually a beam with a certain stiffness).

\begin{itemize}
    \item The torsion acts in a beam section on the shear centre, however in aeroelasticity on the elastic axis
    \item The goal of engineering is always to move the shear center to the front (comes with risk to thin out
    the rear longeron and thicken the front longeron)
    \item In equilibrium, we know that the aerodynamic forces are equal to the spring forces
    \item $M_t + L \cdot e = \theta k_\theta = (q c^2 c_{m0}+ qca\theta e)\cdot b$ (moment equations)
    \item Pitching moment $M_t$ acting on section with $k_\theta$ stiffness
    \item $k_\theta \theta = qc (C_{l/\alpha} e (\theta + \alpha_0)+c C_{m0})$ (Momentum Equation)
    \item $k_h h = L = qcC_{l/\alpha } (\theta + \alpha_0)$ (Lift Equation)
\end{itemize}

\subsection*{Static Instability or Divergence}

\begin{itemize}
    \item If the elastic twist $\theta$ would become infinity for a given stiffness if the denominator of equation
    \item $\theta = qc \frac{C_{l,a} e \alpha_0 + c C_{m,0}}{k_\theta - cqC_{l,a}e}$, $\theta = \infty \Leftrightarrow$ denominator = 0
    \item If the dynamic pressure $q = \frac{k_\theta}{c C_{l,a}e} = q_{div} \rightarrow$ instable (divergence)
    \item Divergence = Static Instability
    \item $M_{tot} = (k_{\theta} -q Sae) \theta - qSa (e\alpha_0 + C_mc)$ (In equilibrium $M_{tot} = 0$) ($M_{tot} > 0$ if in anti-clockwise direction)
    \item Different interpretation: $\frac{\partial M_{tot}}{\partial \theta} \geq 0 \Leftrightarrow $ increasing total moment in section for increasing $\theta$ $ \Leftrightarrow \Delta \theta > 0$
    \item $\frac{\partial M_{tot}}{\partial \theta} \geq 0 \Leftrightarrow$ overall moment brings blade section back to original position
    \item The divergent dynamic pressure can be found by differentiating w.r.t. $\theta$ 
\end{itemize}

\subsubsection*{Lagrange Equation (Energy interpretation)}

\begin{itemize}
    \item $L = T-U$ (Kinetic and potential energy)
    \item $\frac{d}{dt}\frac{\partial L }{\partial \dot{x}_i} - \frac{\partial L}{\partial x_i} = 0$
    \item In statics: $\frac{\partial U}{\partial x_i} = 0$ (Potential energy conservative)
    \item Here: $\frac{\partial U}{\partial x_i} = \frac{\partial}{\partial x}\frac{\delta W}{\delta x}$ (Virtual work), hence for $\theta$
    \item $\frac{\partial U}{\partial \theta}= \frac{\partial}{\partial \theta}\frac{\delta W}{\delta \theta} = 0$ (Check for q)
    \item Resulting $q$ provides the divergence
    \item $U = 1/2\cdot k_\theta \theta^2 = \int F(\theta) d\theta$ (For system with only one spring)
\end{itemize}

\subsubsection*{Section with more than 1 DOF}

\begin{enumerate}
    \item Number of DOF = Dimensions of Stiffness Matrix and number of equations needed (full rank) $K$
    \item Define a potential energy matrix for mechanical system $K_{i,j} = \frac{\partial U}{\partial x_i \partial x_j} = K_{j,i}$
    \item Define aerodynamic matrix $K_a$ based on aerodynamic forces (independent on $e$ for example)
    \item If a matrix is not symmetric = Non-conservative forces
    \item Similar to before, instead of asking if the system is stable if the denominator is zero, we must know if the determinant of the transfer function is zero
    \item Transfer Function: $[K-qK_a] = K_{ael} $ is `Aeroelastic K'
    \item Find a q for which the transfer function determinant becomes zero, which is divergence dynamic pressure. The solution (forces acting) is the so called divergence mode
    \item If all eigenvalues are $>0 \Leftrightarrow$ stable, if one is at least $<0 \Leftrightarrow$ unstable (for sections)
\end{enumerate}

\subsubsection*{System (more than 1 section) with multiple DOF}

\begin{itemize}
    \item Define a system with multiple $\theta_i$, whereas the calculations become similar to when when calculating one section with multiple DOF
    \item Make an Ansatz with the lagrange equations and define stiffness matrix K
    \item The aerodynamic matrix becomes the identity matrix (if we only speak about $\theta$)
    \item This implies that the solutions for $q$ are the eigenvalues of $K$
\end{itemize}

\subsubsection*{Comment on eigenvalues}

\begin{itemize}
    \item The eigenvalues of the Aeorelastic $K_{ael}$ cannot guarantee that $q$ are always the eigenvalues, since $K_a$ is sometimes
    non-symmetric (most of the times, only if rotational degrees of freedom present)
    \item In general, following statement holds true: \newline $det(K - qK_a) = 0$
    \item $det(K_a^{-1} K -qI) = det(A-\lambda I) = 0$ 
    \item If A is not symmetric, we can say: There are less eigenvectors and values than the order (n), can be complex and come in complex conjugate pairs
\end{itemize}

\subsection*{Active conrol on sections}

\begin{itemize}
    \item With active control, the behaviour of the elastic twist $\theta$ can be controlled with for example a trailing edge flap
    \item As described in the script, a trailing edge flap can influence the lift and the moment as follows:
    \item $l = qc C_{l/\delta} \delta$, $m = qcC_{m/\delta} \delta$ ($C_{m/\delta} < 0$)
    \item Both forces contribute to the overall moment, hence will be added to the calculations we did previously
    \item With a so called `Gain' $G$, the controller controls $\delta$ proportional to $\theta$, hence a new linear equation system is formed
    \item Assuming the nose-down motion of controlling the the edge flap, we have to simplify terms, the end result is 
    \item $q_{div,flap}= \frac{k_\theta}{cae - Gca^*}$, with $a^* = -(cC_{m/\delta}+eC_{l/\delta})/(c\delta)$
\end{itemize}

\subsection*{Ritz Method}

... is a energy variational method whereas an equilibrium occurs in correspondence of an extreme of potential energy. A general application is the virtual work.
According to the Hamilton's principle and Lagrange equations, we can define a set of equations.

\begin{itemize}
    \item $\frac{\partial V}{\partial x_i} = 0$ for all $i$ ($x_i$ degrees of freedom)
    \item From mechanics, we need the bending stiffness ($I$) and the torsional stiffness ($J$)
    \item $I = 1/12*b*h^3$ (w.r.t $x$) and $J = \frac{4 A^2}{\int ds/t}$ (Integral is perimeter (Umfang) divided by thickness)
    \item For circular shapes: $I = \frac{\pi}{4}r^4$, $J= \frac{\pi}{2}r^4$
\end{itemize}


\subsubsection*{Derivation for a beam section (mechanical part, torsion)}

\begin{itemize}
    \item Torsional Strain Energy: $U = \frac{1}{2}\int_0^l GJ (\frac{\partial \theta}{\partial x})^2 dx$
    \item Given aerodynamic forces are non-conservative, we use the concept of virtual work
    \item $\delta U = \delta W$ (Virtual work due to non-conservative forces)
    \item $\sum_{i=1}^n \frac{\partial U}{\partial x_i} \delta x_i = \sum_{i=1}^n \delta W_i$ (reformulated for small variations of one DOF)
    \item $\frac{\partial U}{\partial x_i} - \frac{\delta W}{\delta x_i} = 0$ (reformulated)
    \item The work done by the external forces (aerodynamic) can be rewritten:
    \item $\delta W = \int_0^l m(x) \delta \theta(x) dx$ 
    \item $m(x)$ generated by aerodynamic forces
    \item Without going into further detail, there are 2 distinct cases from which one has to go on in the calculation, either the functions are given in a generalised form or in matrix form
    \item $\theta(x) = \sum_{i=1}^N \phi_i(x)a_i = [\Phi] \{a\}$
    \item $[\Phi]$ is a row vector with elements $\phi_i$!
    \item $\phi_i(x)$ are shape functions and $a_i$ are coefficients and the linear combination of those make up $\theta(x)$
    \item Finally, the overall equations result in $[K] \{a\} = \{f\}$ 
    \item $K_{i,j} = \int_0^l GJ \phi_{i_x} \phi_{j_x} dx = GJ \frac{\partial^2 U}{\partial a_i \partial a_j}$ (Stiffness matrix entries, partial derivatives w.r.t. to $x$ and $a$)
    \item $[K] = GJ \int_0^l [\Phi_x]^T[\Phi_x] dx$
    \item $f_i = \int_0^l m(x) [\Phi]_i(x) dx$ (index $i$ for each element)
    \item $\{f\} = \int_0^l m(x) [\Phi]^T dx$ (in matrix form)
\end{itemize}

\subsubsection*{Derivation for a beam section (aerodynamic part, torsion)} 

Following assumptions are drawn:

\begin{itemize}
    \item Elastic axis is perfectly straight
    \item Aerodynamic center of all sections on a straight line
    \item External moments as before by aerodynamic forces: $m(x) = qcea(\theta(x)+\alpha_0)$
\end{itemize}

Replace all $\theta(x)$ with the above solutions and insert insert $m(x)$ into generalised forces vector

\begin{itemize}
    \item $\{f\} = q \int_0^l cea\alpha_0 [\Phi]^T dx + q \int_0^l cea [\Phi]^T [\Phi] dx \{a\}$
    \item $\{f\} = \{f_0\} + q [K_A] $
    \item $[K_A] = \int_0^l cea [\Phi]^T [\Phi] dx$
    \item $\{a\} = ([K]-q[K_A])^{-1}\{f_0\}$ solves for all $a$
    \item $a$ gives us the beam gives us the response of the system (pitch $\theta$ at all points along $x$)
    \item Stability: $det([K]-q[K_A]) = 0$
    \item Hence the basis of the solution of the eigenvalue problem
    \item Eigenvalues $q$: Dynamic pressure where zero stability 
    \item Eigenvectors: Corresponding divergence modes
    \item Attention: $a \neq \{a\}$! (lift slope vs. coefficients)
\end{itemize}

\subsubsection*{One single shape function (1-DOF) and $[\Phi] = \phi(x)$}

\begin{itemize}
    \item $\theta(x) = a \cdot \phi(x)$ is one dimensional, we assume $a=1$ because we can define it within $\phi$
    \item $U = \frac{1}{2} \int_0^l GJ \phi_x^2 dx$
    \item $K = \int_0^l GJ \phi_x^2 dx = [K]_{torsion}$
    \item $K_a = \int_0^l ce C_{l\alpha} \phi^2 dx$ ($C_{l\alpha} = a$ lift curve slope)
    \item $f_0 = q \int_0^l c e C_{l\alpha} \alpha_0 \phi dx$
    \item $(K-qK_a)a= f_0$
    \item $a = f_0 / (K-qK_a)$ gives us the response by which $\theta$ is multiplied
    \item $q_d = K/K_a$ gives us the divergence
\end{itemize}



\subsubsection*{One single shape function, Bending and Twisting}

\begin{itemize}
    \item For bending, the potential energy is: $U = \frac{1}{2}\int_0^l EI \psi^2_{xx} dx$
    \item $K = \int_0^l EI \psi_{xx}^2 dx = [K]_{bending} $
    \item If we assume bending takes place and torsion as well, we assume both to be decoupled
    \item The stiffness matrix reads
    \item $ K = \begin{bmatrix}
        [K]_{bending} & 0 \\
        0 & [K]_{torsion}
    \end{bmatrix}$ 
    \item The aerodynamic stiffness matrix is of shape (always for bending and twisting):
    \item $ K_a = \begin{bmatrix}
        0 & \int_0^l c C_{l\alpha} \phi \psi dx \\
        0 & \int_0^l ec C_{l\alpha} \psi^2 dx
    \end{bmatrix}$
    \item The vector $x$ includes the coefficients for the shape functions $\psi$ and $\phi$:
    \item $x = (K-qK_a)^{-1} f$
    \item $f = \begin{Bmatrix}
        q \int_0^l c C_{l\alpha} \phi\psi dx \\
        q \int_0^l c C_{l\alpha} \psi^2 dx
    \end{Bmatrix}$
    \item $det(K-qK_a) = 0 \Leftrightarrow$ then $q = q_{div}$
\end{itemize}

\subsubsection*{Shape functions}

Shape functions have to be chosen. In FEA, shape functions are local for each finite element. 

\begin{itemize}
    \item Orthonormal modes: $\int \psi_i \psi_j dx = 0$ for different shape functions in $i$ and $j$
    \item Simple polynomials are great shape functions $x/l$, $(x/l)^n$
    \item Natural vibration modes or normal modes (eigenvectors of the problem): $K-\lambda M$ with $K$ the stiffness and $M$ the mass matrix (Important for dynamic systems later)
\end{itemize}

\subsubsection*{Bending / twisting coupling}

In class multiple examples have been shown whereas following are the key takeaways:

\begin{itemize}
    \item Out of plane bending can exist if for example the shear centre and the principle axes (centre of gravity) are apart from each other significantly
    \item The conventions for positive and negative $e$ eccentricities: positive if aerodynamic centre in front of shear centre and hence negative if the other way around
    \item Positive e is detrimental for aeroelastic stability, negative is beneficial
    \item Helicopter blades have D-spars to shift the elastic axis forward
    \item Gurney flaps at the end of the wing with length $1\%$ of $c$ make the wing virtually longer 
\end{itemize}

\subsubsection*{Control effectiveness, typical section}

Control effectiveness is described by following term:

\begin{itemize}
    \item $\frac{L_{elastic}}{L_{rigid}} = \frac{1-\frac{q}{q_r}}{1-\frac{q}{q_{div}}} = $ Control Effectiveness
    \item $q_r = -q_{div}\frac{e}{c}\frac{C_{l\delta}}{C_{m\delta}}$ ($C_{m\delta}$ is negative!)
    \item If the control effectiveness is zero, the aileron deflection does not contribute to more lift
    \item If the control effectiveness is is negative, this means that $0 \leq q_r< q < q_{div}$, the control system pushes the system
    into the contrary direction of intended use
    \item Possible goal: As close to $q_{div}$ and below $q_r$
    \item Another solution: Outboard ailerons (less stiff due to smaller torsional stiffness) and inboard ailerons ($GJ/l$)
    \item The overall equations for future equations will be for equilibria:
    \item $[K]\{\phi \} =  q [K_a] \{\phi\} + \{m_0\} + q \{f_c\} \delta $ (Aileron Equation)
    \item For mulitple segments: $q [f_c] \delta$
\end{itemize}

\subsubsection*{Effects of Sweep Angle on Divergence}

\begin{itemize}
    \item In this course, a 2-DOF model is used (pitching and flapping)
    \item Spring stiffness for moments on a beam: $k_\theta$ and $k_\phi$
    \item $k_\theta = \frac{G J}{l}$ (l = length of lever/beam $b$, torsion)
    \item $k_\phi = \frac{E I}{l}$ (flapping / bending)
    \item $G/E = \frac{1}{2(1+\nu)}$ (Poisson ratio)
    \item The angle of attack will be reintroduced:
    \item $\tan(\alpha) = \frac{V\perp}{V\parallel} = \frac{-V \sin(\Lambda) \sin(\phi)}{V \cos(\Lambda)} = -\tan(\Lambda)\sin(\phi)$
    \item $\alpha = -\tan(\Lambda) \phi$ (small angle approx)
    \item In the script the approach given results are as follows:
    \item $[K] = \begin{bmatrix}
        k_\phi & 0 \\ 0 & k_\theta
    \end{bmatrix}$
    \item Simplifications: $Q = q_n c b C_{l\alpha}$ and $t = \tan(\Lambda)$ ($Q$ is $q$ redefined)
    \item $[K_a] = \begin{bmatrix}
        -tb/2 & b/2 \\ -te & e
    \end{bmatrix}$
    \item $\{f\} = \frac{Q \alpha_0}{\cos(\Lambda)} \begin{Bmatrix}
        b/2 \\ e
    \end{Bmatrix}$
    \item $[K]_{ael} = [K]-Q[K_a]$
    \item Condition for divergence:
    \item $det (K_{ael}) = \Delta = 0 \Leftrightarrow Q_D = \frac{k_\phi k_\theta}{k_\phi e - k\theta bt /2}$ 
    \item $\Leftrightarrow q_D = \frac{k_\theta / (Se C_{l/\alpha})}{\cos^2(\Lambda) [1-(b/e)(k_\theta/k_\phi)(\tan(\Lambda)/2)]}$
    \item This gives us a uniqe solution for two degrees of freedom. If we try to push $q_D \rightarrow \infty$, we can do so by setting the denominator of $q_D = 0$
    \item This allows us to model our wing with geometrical and material parameters such that the system never becomes unstable
    \item Divergence can already be avoided with small $\Lambda$ sweep angles
    \item Because the angle creates a coupling between wing bending and torsion (deformation), and the angle of attack
    \item $\alpha_{new} = \alpha_0 / \cos(\Lambda) + \theta - \phi \tan(\Lambda)$ 
    \item Assuming small angles: $\alpha_{new} \approx \alpha_0 + \theta - \phi \Lambda$ whereas the negative part is larger
    \item Other approaches: Build wing with unbalanced comosite laminates such that bending/twisting coupling is generated by material
    \item Adding aerodynamic control surfaces (see flaps) with active control
    \item Negative sweep angles reduce divergence speed
\end{itemize}

\subsubsection*{Sweep Angles and Ritz Method (Class Notes)}

\begin{itemize}
    \item Shape functions are chosen such that one has dependencies along the section
    \item For twisting: $\theta(y) = f_\theta(y) \Theta$ ($\Theta$ constant)
    \item For bending: $w(y) = f_w(y) B$
    \item $\phi = \frac{\partial w}{\partial y} = B \cdot \frac{\partial f_w}{\partial y}$
    \item This will make the angle of attack $\alpha$ dependent on $y$
    \item $\alpha(y) = -\tan(\Lambda) \frac{\partial f_w(y)}{\partial y}B + f_\theta(y) \Theta$
    \item Hence lift and moment become also dependent on $y$ (we also assume $C_{m,0} = 0$ or const):
    \item $l(y) = q c C_{l,\alpha} (\alpha(y))$, $m(y) = l(y)\cdot e$
    \item Principle of virtual work:
    \item $\delta W = \int (l(y) \delta w+ m(y) \delta \theta) dy$
    \item Expanding $\delta w$ with $f_w \delta B$ and $\delta \theta$ with $\delta \theta$ with $f_\theta \delta \Theta$ will yield an integral whereas  
    \item The aerodynamic stiffness $K_a$ will be of following shape and hence non-symmetric:
    \item $[K_a] = \begin{bmatrix}
        \int \dots \Theta \delta \Theta & \int \dots B \delta \Theta \\
        \int \dots \Theta \delta B & \int \dots  B \delta B
    \end{bmatrix}$, $x = \begin{Bmatrix}
        \Theta \\ B
    \end{Bmatrix}$
\end{itemize}

\section*{Unsteady Aeroelasticity}

\subsubsection*{Dynamic Systems (Repetition from Bachelor Level)}

\begin{itemize}
    \item $\dot{x} = Ax + B u$
    \item $y = Cx + Du$
    \item $x$ are state variables, $u$ are input variables and $y$ are output
    \item Aeroelastic System:
    \item $m \ddot{z} + k_z z = $ Mechanical force in lunge direction 
    \item $I \ddot{\theta} + k\theta = $ Mechanical force in pitch direction
    \item In compact form:
    \item $M \ddot{x} + C \dot{x} + K x = $ Mechanical Forces $ = f = $Input
    \item Canonical Form:
    \item $A \dot{r} = g$
    \item $r = \begin{Bmatrix}
        x \\ \dot{x}
    \end{Bmatrix}$  and $g = \begin{Bmatrix}
        0 \\ M^{-1}f
    \end{Bmatrix}$ 
    \item $r$ is of dimension $2n$ (twice the No. of DOF) and ordered such that the definition of $A$ is valid
    \item The roots of the characteristic polynomial of $A$ tell us if the system is stable 
    \item The eigenvectors are the modes of the system and the response of the system is a linear combination of these nodes
    \item The topic will be looked at again later on, however one can say in general for the mass, damping and stiffness:
    \item $M = \begin{bmatrix}
        m & m x_{cg} \\ mx_{cg} & I + mx_{xg}^2
    \end{bmatrix}$ 
    \item $C = \begin{bmatrix}
        c_z & 0 \\ 
        0 & c_\theta
    \end{bmatrix}$
    \item $K = \begin{bmatrix}
        k_z & 0  \\ 
        0 & k_{\theta} 
    \end{bmatrix}$
\end{itemize}

\subsubsection*{Quasi-steady approach}

Quasi-steady approaches take into account that the angle of attack changes due to the velocity component normal to the free stream direction

\begin{itemize}
    \item The aerodynamic equations for 2-DOF systems is in general:
    \item $l(\theta,\dot{\theta},\dot{z}) = q c C_{l/\alpha} \left( \theta - \frac{\dot{z}}{V}+(c/2-e)\frac{\dot{\theta}}{V}\right)$
    \item $m(\theta,\dot{\theta},\dot{z}) = l(\theta,\dot{\theta},\dot{z})\cdot e$
    \item This aerodynamic forces hence modify the damping and stiffness matrices of the overall equations of motion:
    \item $[C_a] = \frac{1}{V}c b C_{l/\alpha}\begin{bmatrix}
        -1 & c/2-e\\
        -e & e(c/2-e)
    \end{bmatrix}$ 
    \item $[K_a] = c C_{l/\alpha} \begin{bmatrix}
        0 & 1   \\
        0 & e
    \end{bmatrix}$
    \item For state vector $x = \begin{Bmatrix}
        z           \\ 
        \theta 
    \end{Bmatrix}$
    \item  The mechanical forces do not change, hence:
    \item $[M_{ael}] = [M]$, $[C_{ael}] = - q [C_a]$, and $[K_{ael}] = [K]-q [K_a]$
    \item To check stability: Check eigenvalues of A from created from the matrices (dynamical)
    \item $det(A-I\lambda) = det(-M^{-1} K - I\lambda) \overset{!}{=} 0$ From linear algebra
    \end{itemize}

\subsubsection*{Comment on Stability}
    
    \begin{itemize}
        \item If a system is \emph{statically unstable}, it is also \emph{dynamically unstable}
        \item If a system is \emph{dynamically unstable}, it must not be necessarily \emph{statically unstable}
    \end{itemize} 

Check first if following conditions hold true:

    \begin{enumerate}
        \item Static stability via determinant of $[K_{ael}]$ (if unstable, also dynamically unstable)
        \item If statically stable, check if dynamically unstable (eigenvalues of A)
        \item If both are stable, then overall stable at all times
    \end{enumerate}

 
\subsubsection*{Approach for different models than airfoil in quasi-steady approach}  

\begin{itemize}
    \item For a flapping wing with 1-DOF $\rightarrow$ model a high stiffness $k_z \rightarrow \infty$ and check response
    \item For a helicopter, we have contributions from a 2-DOF model (pitching $\theta$ and flapping $\beta$)
    \item Additionally, the velocity V is a function of the blade's radius ($V(r) = \Omega r$, sometimes $\omega$)
    \item The angle of attack is hence $\alpha = \theta - \dot{\beta} / \Omega $
    \item Derivation: $\frac{\dot{\beta}r}{V} = \frac{\dot{\beta} r}{\Omega r}$ and $q(r) = \frac{1}{2}\rho V^2 = \frac{1}{2} \rho \Omega^2 r^2$
    \item The aerodynamic forces can be calculated first as a function of $r$
    \item $l(r) = q(r) c C_{l/\alpha} \left( \theta - \dot{\beta}/\Omega \right)$ with $l = [N/m]$
    \item $m(r) = l(r) \cdot e $
    \item The overall moments have to be integrated over the whole radius (span) for the momentum contributions from flapping and pitching
    \item Eqution of motion:
    \item $I_\beta \ddot{\beta} + I_\beta \Omega^2 \beta = M_{\beta}^{aero}$ with centrifugal force (par. axis)
    \item $I_\theta \ddot{\theta} + k_\theta \theta = M_{\theta}^{aero}$ as usual
    \item $M_{\beta}^{aero} = \int_0^R l(r) \cdot \emph{r} dr = \frac{1}{8} c C_{l/\alpha} \Omega^2 R^4 \left( \theta - \frac{\dot{\beta}}{\Omega} \right)$
    \item $M_{\theta}^{aero} = \int_0^R m(r) \cdot dr = \frac{1}{6} ec C_{l/\alpha} \Omega^2 R^3 \left( \theta - \frac{\dot{\beta}}{\Omega} \right)$ 
    \item Natural frequency is $\omega = \sqrt{K/M} = \sqrt{I_\beta\Omega^2/I_\beta} = \Omega$
    \item Reduced frequency is $ k = \frac{\omega c}{2 V} = \frac{c}{2r}$ in this case
    \item The reduced frequency checks corrects the natural frequency by the ratio between the half-chord and the velocity, making it dimensionless
\end{itemize}

\section*{Unsteady Aerodynamics}

\subsubsection*{Important Parameters for Unsteady Aerodynamics}   

\begin{itemize}
    \item The natural frequency $\omega$ is the the \emph{highest eigenfrequency} of the system which can be excited in a given situation
    \item The reduced frequency $k$ is a dimenionless frequency and used in the frequency domain
    \item If $k<0.05 \rightarrow$ quasi-steady approach
    \item If $k>0.05 \rightarrow$ unsteady approach
    \item The reduced time $\tau = t \frac{2V}{c}$ is a dimensionless time used to describe transient phenomena in the time domain
    \item Critical damping $\overline{c} = 2 \sqrt{M K}$
    \item Damping ratio $\zeta = C / \overline{c}$
    \item Unsteady aerodynamics: The reduced frequency measures the unsteadiness of aerodynamics: Air is not able to adjust instantly to a change in $\alpha$
\end{itemize}

\subsubsection*{Dynamic Systems in the Frequency Domain}

    \begin{itemize}
        \item In Fourier space $F(\omega)$:
        \item $x = X e^{i\omega t}$,  $y = Y e^{i \omega t}$ and $u = U e^{i \omega t}$
        \item $X = (i\omega I - A)^{-1} B U$
        \item $Y = C((i\omega I -A)^{-1}B + D ) U = H(\omega) U$
        \item We can define Transfer Functions = Operators for the structure and for the aerodynamics:
        \item $H_S(\omega) = -\omega^2 m + k_z = -\omega^2 [M] + [K]$
        \item $H_A(\omega) = i\omega q c_a + q k_a = i\omega q [C_a] + q [K_a]$
        \item $Z = (H_S - H_A)^{-1} F_z(\omega)$ 
        \item mapping the input $F_z$ (aero forces, disturbances) to the output $Z$ (state variable) from before. 
    \end{itemize}

 \subsubsection*{Example for frequency domain "Flying Door"}
 
    \begin{itemize}
        \item Assuming only flapping $\beta$ DOF:
        \item The angle of attack $\alpha_G (t) = v_G(t)/V$ 
        \item The angle of attack directly depends on the gust velocity normal to the free stream velocity
        \item Same calculation as before in the example in quasi-steady approach where $\beta$ becomes the only variable
    \end{itemize}

 \subsubsection*{Modal Decomposition}   

\begin{itemize}
    \item Free vibrations of the system are solutions of the simple system:
    \item $M \ddot{x} + K x = 0$
    \item whereas the eigenvalues $\omega$ and eigenvectors $\Phi$ can be obtained
    \item According to linear algebra, we can define new coordinates:
    \item $r = \Phi x$ hence a coordinate transform gives us following equation:
    \item $\Phi^T M \Phi \ddot{r} + \Phi^T M \Phi r = \Phi^T f = $ GAF
    \item GAF are the Generalized Aerodynamic Forces (if only aero forces present in $f$)
    \item The behavious of the system is given by a linear combination of its modes. For wings, we will consider
    mainly two modes (torsion and bending), which we deem most important and hence the entire calculations are simplified
    \item However, the main criteria of deciding which to keep and which not to keep are based on:
    \item Frequency: Will certain oscillations be reached? Consider also multiple higher frequencies (for example during buffeting)
    This is when shockwaves hit our system or we experience airflow separation
    \item Modeshape: Certain modes must be within the system to represent something physical (Fuselage must be able to move) 
    \item For problems of this type, we have to calculate the new matrices such that it has the same shape as in the beginning:
    \item $\overline{M} = \Phi^T M \Phi$,  $\overline{K} = \Phi^T K \Phi$,  $\overline{f} = \Phi^T f$
\end{itemize}

\subsubsection*{Ritz Method for dynamic Systems}

Since we have non-conservative forces, we have to write:

    \begin{itemize}
        \item $\frac{d}{dt} \frac{\partial L}{\partial \dot{x}} - \frac{\partial L}{\partial x} = \frac{\delta W}{\delta x}$
        \item $\delta W = \int_0^b (l(y) \delta w + m(y) \delta \theta) dy$
        \item The trial/shape functions are linear combinations with time depending weights
        \item $w(y,t) = \sum \psi_i(y) a_i(t)$, $\theta(y,t) = \sum \phi_i(y) b_i(t)$
        \item We can use static solutions, polynomials, normal modes or a combination
    \end{itemize}

\subsubsection*{Theodorsen (Unsteady Aerodynamics in frequency domain)}
Unsteady aerodynamics stems from the fact that a delay is found for $C_{l/\alpha}$ if we change the angle of attack $\alpha$ very quick. The reason behind this is that the distribution of the airfield 
has to adjust since the angle changes quickly in time (instationary airflow)
\newline \newline We want to quantify \emph{aerodynamic lag and reduction of magnitude} by introducing some functions depending on reduced frequency $k$:

    \begin{itemize}
        \item $C(k) = \frac{H_1^{(2)}}{H_1^{(2)}(k) + i H_0^{(2)}(k)}$
        \item Hankel functions of second kind are denoted as $H$ (on basis of Bessel functions)
        \item We can use Theodorsen only for harmonic motions
        \item $H_n^{(2)} = J_n(k)-i Y_n(k)$
        \item We will use following simplification/approximation (Theodorsen Function):
        \item $C(k) = 1 - \sum_{j=1}^2 \frac{A_j}{1-B_j/(ki)}$  
        \item Where we use them for the reduced frequency ($i$ imaginary)
        \item Theodorsen follows from thin airflow theory and exploits ideal non-rotational aerodynamics (no viscosity and vortices)
        \item The matrices are corrected via (in the frequency domain):
        \item $[C_a] \rightarrow C(k) [C_a]$, $[K_a] \rightarrow C(k)[K_a]$
        \item A system with following characteristic equation (2-DOF):
        \item $\begin{Bmatrix}
            L \\ M
        \end{Bmatrix} = q [K_a] \begin{Bmatrix}
            h \\ \alpha
        \end{Bmatrix} + q [C_a] \begin{Bmatrix}
            \dot{h} \\  \dot{\alpha}
        \end{Bmatrix} + q [M_a] \begin{Bmatrix}
            \ddot{h}  \\ \ddot{\alpha} 
        \end{Bmatrix}$
        \item $K_a = \overline{C_{l/\alpha}} S \begin{bmatrix}
            C(k) & 0 \\ C(k)c(1/4-\gamma) & 0
        \end{bmatrix}$
        \item $C_a = \frac{\overline{C_{l/\alpha}} S}{V}  \begin{bmatrix}
            c(\frac{1}{4}+ C(k)(\frac{1}{4}+\gamma)) & C(k) \\ -\frac{c}{4}(\frac{1}{4}+\gamma)+C(k)(\frac{1}{16}-\gamma^2) & C(k)c(\frac{1}{4}-\gamma)
        \end{bmatrix}$
        \item $M_a = \frac{\overline{C_{l/\alpha}} S c}{V^2} \begin{bmatrix}
            c \gamma/4 & 1/4 \\ -c(1/32+\gamma^2) & -c\gamma/4
        \end{bmatrix}$
        \item $\gamma = 1/4-e/c$ 
        \item $\gamma$ is the distance of the shear centre from the mid-chord point
        \item $\overline{C_{l/\alpha}} = $ steady-state lift curve slope 
        \item The aerodynamic damping and stiffness matrices are closely related to the steady-state case 
        \item The aerodynamic mass matrix is the so-called ``\emph{added mass}'' which stem from the fluid's inertia 
        \item Theodorsen function is mainly used for the frequency domain, not time domain (hence the interpretation of them being a correction of time domain system is misleading)
        \item The matrices are now depending on $k$, hence the response can be different for different $k$'s
    \end{itemize}

 \subsubsection*{Analysis of unsteady aeroelastic systems}   

 A system is unsteady: We excite the system by unsteady forces (forced oscillations) or we model the system as unsteady in order to assess the dynamic stability

 \begin{itemize}
     \item In time domain: We check the response to a given excitation, through dynamic stability, ABCD Matrices, Numerical (Direct time integration) with response in unsteady excitation or canocial inputs and lastly the the flutter stabiltiy is given where the first value of q where the system diverges
     \item In frequency domain: Same as time domain, however later through transfer function from ABCD, poles form the transfer function and excitation response in frequency domain
 \end{itemize}

 Unsteady aerodynamics: We correct the quasi steady formulation with Theodorsen, in time domain there is another formulation, the so called Wagner function

 \begin{itemize}
     \item We mostly add a time dependency into our system for the equations of motion by adding a non-conservative, signal input depending on time $t$:
     \item $\alpha = \alpha_0 + \theta + \dot{\theta}/V + ... + w(t)/V$
     \item $w(t)/V$ could be for example a gust of a magnitude $m$ at time $t$
 \end{itemize}

 \subsubsection*{Wagner function (Unsteady Aerodynamics in time Domain)}

 In practice, the response depends on the Aspect Ratio $AR$, the $Ma$ Mach number. We assume the behaviour as follows:

 \begin{itemize}
     \item $C_l \approx C_{l/\alpha} \Delta \alpha (1- A_1 e^{-B_1 \tau} - A_2 e^{-B_2 \tau})$
     \item $A_1 = 0.165$ and $A_2 = 0.335$ 
     \item $B_1 = 0.0455$ and $B_2 = 0.3$
     \item The solution of a differential equation:
     \item $\dot{y} = -B y$ with initial condition $y(0) = A$
     \item The corresponding system:
     \item $\frac{dy_1}{dt} = -2V/c B_1 y_1+\alpha(t)$ with b.c. $y_1(0) = A_1$
     \item $\frac{dy_2}{dt} = -2V/c B_2 y_2+\alpha(t)$ with b.c. $y_2(0) = A_2$
     \item The states are artificial ``lag states'' which we introduce (like degrees of freedom )
     \item We use as many as needed to approximate the response in the time domain (training data)
     \item $C_l(t) = \overline{C_{l/\alpha}} (2V/c)(A_1B_1y_1 + A_2B_2y_2)$ \\ with steady-state lift curve slope $\overline{C_{l/\alpha}}$
     \item Transfer Function in Laplace Domain:
     \item $H(s) = \frac{(A_1B_1+A_2B_2)s + B_1B_2}{s^2+(B_1+B_2)s + B_1B_2}$
     \item To derive $A_i$ and $B_i$, one does wind tunnel or numerical experiments und derive ``best values'' (= ROM or Reduced Order Model)
     \item Wagner function and Theodorsen are 100\% consistent with each other (different domains)
 \end{itemize}

 \subsubsection*{Flutter Tracking}
    We search for the ``flutter dynamic pressure'' $q_f$, which is where the stability of the system is not given anymore. At first one has to find the eigenvalues and vectors as a function of $q$ and the frequencies in the system's modes
    \begin{itemize}
        \item To begin with let's define the system matrix $A$:
        \item $A = \begin{bmatrix}
            0 & I \\ -M^{-1}(K-qK_a) & -M^{-1}(-qC_a)
        \end{bmatrix} = A(q)$
        \item The eigenvalues of $A$ and the modes (eigenvectors) are hence also depending on $q$ (or one could also vary $V$)
        \item Two plots are created, where the eigenvalues (frequencies) are plotted against $V$ or $q$
        \item For the real and imaginary part of the eigenvalues $Re(\lambda)$ and $Im(\lambda)$
        \item If the real part of the eigenvalues become positive, the system reaches its flutter speed for a given $q_{f}$
        \item On the imaginary axis, one can see that the eigenvalue associated to the diverging one is converging to another one (coalescence)
        \item The other mode's real part becomes more negative
        \item Both modes (associdated to the eigenvalues) are coupled (or start to interacti with each other). One becomes heavily damped, the other one becomes unstable
        \item Stable modes dissipate energy into the flow
        \item Unstable modes absorbs energy from the flow 
        \item The overall energy of the system increases with $q$ due to the non-conservative aerodynamic forces
    \end{itemize}

\subsubsection*{Introduction of the Centre of Gravity}

By introducing the centre of gravity, one can see that the kinetic energy of the system changes: 

\begin{itemize}
    \item $T= 1/2 \cdot m \dot{z}^2 + 1/2 \cdot I \dot{\theta}^2$ ($x_{cg} = $ shear centre)
    \item $T = 1/2 \cdot m (\dot{z}+\dot{\theta} x_{cg})^2 + 1/2 \cdot I \dot{\theta}^2$ ($x_{cg} \neq $ shear centre)
    \item The mass matrix will be therefore modified (when we differentiate w.r.t. to $\theta$ and $z$)
    \item It follows that the bigger the changes in $z$ and $\theta$, the larger the effect of $M$ becomes on the entire system
    \item Also from a heuristic point of view, if the centre of gravity is further in front to the wing, perturbations will cause the system to react and create moments such that the system becomes unstable
\end{itemize}

\subsubsection*{Effect of Spring Positions}

When placing the springs at to different positions with the longitudinal spring being placed by $e_1$ in front of the shear centre (where the torsional spring lies), then we can say that:

    \begin{itemize}
        \item The potential energy is effected, which is different to the case of before with the centre of gravity
        \item $U = 1/2 k_z z^2 + 1/2 k_\theta \theta^2$
        \item with shear centre and longitudinal spring on top of each other
        \item $U = 1/2 k_z (z+e_1 \theta)^2 + 1/2 k_\theta \theta^2$
        \item while the long. spring is in front shifted by $e_1$
        \item $K = \begin{bmatrix}
            k_z & e_1\cdot k_z \\ e_1 \cdot k_z & k_\theta
        \end{bmatrix}$
        \item There will be an off-diagonal term, which indicates that for a positive change in $z$ located at the shear axis, that the AoA $\alpha$ becomes negative
        \item This is favourable, since a positive bending is coupled to negative twisting (same case for the centre of gravity being shifted to the front)
    \end{itemize}


 \subsection*{Windtunnel Testing}

 \subsubsection*{Limit Cycles}

    \begin{itemize}
        \item Even though we can model systems with degrees of freedom as linear combinations of eigenvectors 
        \item If the eigenvalues have positive real part, the eigenvectors diverge to infinity and are unstable
        \item In the windtunnel however, this simple formulation cannot always be observed
        \item Ther is a certain limit, from which on the oscillations become constant 
        \item The reason for this stems from the following:
        \begin{enumerate}
            \item The system is no longer linear
            \item Either the structural or aerodynamic non-linearity appear
            \item For example, the torsional moment tends to vary proportionally to the first and third power of twist (instead of only first):
            \item $M_t = k\theta + k_3 \theta^3$
            \item At low $Re$ numbers, the aerodynamics becomes non-linear (the lift curve slope as well) and even can lower lift 
            \item Also boundary layer separation can occur
        \end{enumerate}
        \item For so called $LCO$-curves, one can now say that at flutter speed (read on flight speed axis), instead of the LCO Amplitude going to infinity, the amplitude grows non-linearly
        \item For strong nonlinearity, the LCO amplitude can converge to a certain value
        \item \emph{In conclusion,} flutter is only the case where we describe the velocity or dynamic pressure when the stability is not given anymore
        \item In the linear system, the amplitude of the mode diverges (flutter)
        \item In the nonlinear system, the amplitude reaches a certain limit cycle (oscillating)
        \item In dynamic systems with time delay, the system jumps after reaching flutter speed (unsteady systems), in steady or quasi-steady systems, they follow the nonlinear curve after reaching flutter speed instantaniously
        \item When we throttle the syst4em above flutter speed, it follows a similar curve to the instantanious one 
    \end{itemize}

 \subsubsection*{Mathematical Appendix}

    \begin{itemize}
        \item $e^{[A]t} = T^{-1} e^{[D]t} T$ for the Matrix Exponential
        \item $T$ is the basis of eigenvectors in the diagonal matrix $D$
        \item $y(t) = C e^{At}x_0+ \int_0^t C e^{A(t-\tau)}Bu(\tau)d\tau + D u(t)$ (time-domain)
        \item $Y(s) = \sum(s)U(s) = (C(sI-A)^{-1}B+D)U(s)$ (frequency domain)
    \end{itemize}

\end{multicols*}
\end{document}

