\documentclass[8pt, landscape, fleqn]{scrartcl}
\setlength{\parindent}{0pt}
\usepackage[ngerman]{babel}
\usepackage[applemac]{inputenc}
\usepackage[dvips]{geometry}
\usepackage{latexsym}
\usepackage{multicol}
\usepackage{amsmath}
\usepackage{graphicx}
\usepackage{array}
\usepackage{booktabs}
\usepackage{amsmath}
\usepackage{mathtools}
\usepackage{ulem}
\usepackage{amsfonts}
\usepackage{dsfont}
\usepackage{charter} %%% Schreibart
%\renewcommand{\familydefault}{\sfdefault}



%%%%%%%%%%Paket für Chemische Formeln
\usepackage{chemformula} 
\usepackage[version=3]{mhchem}
%%%%%%%%%%%%%%%%% Farbe
\usepackage{color}

\pagestyle{plain}
\typearea{20}
\columnsep 30pt
\columnseprule .4pt
\setlength{\extrarowheight}{0.9em}

\renewcommand{\arraystretch}{0.8}

\makeatletter
\renewcommand{\section}{\@startsection{section}{1}{0mm}%
{-2\baselineskip}{0.8\baselineskip}%
{\hrule depth 0.2pt width\columnwidth\hrule depth1.5pt
width0.25\columnwidth\vspace*{1.2em}\Large\bfseries\rmfamily}}
\makeatother


\makeatletter
\renewcommand{\subsection}{\@startsection{subsection}{1}{0mm}%
{-2\baselineskip}{0.8\baselineskip}%
{\hrule depth 0.2pt width\columnwidth\hrule depth0.75pt
width0.25\columnwidth\vspace*{1.2em}\large\bfseries\rmfamily}}
\makeatother

\makeatletter
\renewcommand{\subsubsection}{\@startsection{subsubsection}{1}{0mm}%
{-2\baselineskip}{0.8\baselineskip}%
{\hrule depth 0.2pt width\columnwidth\vspace*{1.2em}\normalsize\bfseries\rmfamily}}
\makeatother

\newcommand{\Mx}[1]{\begin{bmatrix}#1\end{bmatrix}}
\begin{document}
\part*{\LARGE\textrm{Aeroelasticity $\hfill$ Xeno Meienberg}}
\begin{multicols*}{3}

\section*{Parameters}
\begin{itemize}
\item Aerodynamic Force $\overline{F_A} = \overline{L}+\overline{D}$
\item Lift Force $L$ $[N]$
\item Drag Force $D$ $[N]$
\item Aerodynamic Moment $M_A$ $[Nm]$
\item Dynamic Pressure $q = 1/2 \rho V^2$ (Bernoulli) $[Pa]$
\item Chord Length $c$ $[m]$
\item Surface Area $S = b\cdot c$ $[m^2]$ (Rectangular)
\item Wing Span $b$ $[m]$ 
\item Lift Coefficient $C_{l}=L/(1/2\rho V^2\cdot S) $
\item Drag Coefficient $C_d = D/(1/2\rho V^2 \cdot S)$
\item Moment Coefficient $C_m = M_A/(1/2\rho V^2 \cdot c \cdot S)$
\item Angle of Attack $\alpha$ $[rad]$ (positive in clockwise direction)
\item Lift curve slope $a = C_l / \alpha \approx tan(angle\,x-axis\,to\,curve)$
\end{itemize}

\section*{Conventions throughout Course}

\begin{itemize}
    \item If L and D absolute $\rightarrow$ use calculations above
    \item If L and D per span unit $\rightarrow$ correct via dividing by $b$
    \item Sign conventions: Lift positive, Drag positive in x and y direction
    \item Moments and angles positive in clockwise direction
    \item Our system coordinate system is defined by the wing. The angle of attack is defined relative to it
    \item The variables which describe the airfoil motion are the pitch $\theta$ and the plunge $h$ which act at the shear centre of the wing
\end{itemize}

\section*{Mathematical Basics}

\begin{itemize}
    \item A = $\begin{pmatrix} a & b \\ c & d
    \end{pmatrix}$
    \item Inverse of Matrix (2D): $A^{-1} = \frac{1}{ad-bc} \begin{pmatrix} d & -b \\ -c & a
    \end{pmatrix}$
    \item Inverse of Matrix (3D): 
        \subitem 1. det (A), then transpose A
        \subitem 2. Find the ajunct matrix (minors) of
        \subitem (cover row and column of element) of $A^T$ 
        \subitem and multiply with +- matrix
        \subitem $A^{-1} = 1/det(A)\cdot Adj(A^T)$
    \item The solutions of $A x = 0$ for a matrix $A$, $x$ cannot be just the trivial solution if A is not invertible
    \item $[rad] = \frac{\pi}{180} [deg]$
\end{itemize}



\section*{Steady Aerofoil and Wing Section Aerodynamics}

\begin{itemize}
    \item Aerofoil = 2-D wing section with goal to generate lift force perpendicular to the relative airspeed
    \item Convention: Lift is up, Drag is in direction of windspeed and Aerodynamic moment in clockwise direction acting on the aerodynamic center. Aerodynamic center is normally at the quarter chord position $c_{m,c/4}$ for syymetric airfoils. $x_{ac} = -m_0/2\pi +0.25$ with $m_0$ as a shape constant 
    \item Further assumptions: No viscosity, incompressible fluid, $Ma < 0.2,0.3$, no vortices, potential flow (Navier-Stokes)
    \item Another centre is the shear center (elastic axis) from mechanics
    \item $L = 1/2\rho V^2 c a \alpha$, with $a$ from tables (CFD and Wind Tunnel) $[N/m]$
    \item $M_A = 1/2\rho V^2c^2 c_{m0}$ with $c_{m0}$ also from tables $[N]$
\end{itemize}

\subsubsection*{Lift curve $C_l(\alpha)$ and drag curve $C_d(\alpha)$}

\begin{itemize}
    \item At small ranges of $\alpha$, both lift and drag increase with: $C_l \propto \alpha$ and $C_d \propto \alpha^2$
    \item In aeroelasticity and this course, $\alpha$ will be very small, hence drag will be negligble small
\end{itemize}

\subsubsection*{The aerodynamic moment $M_A$}

\begin{itemize}
    \item The aerodynamic moment is much more important than drag $C_d$
    \item $M_A$ varies with $\alpha$ in the small ranges of the angle of attack (very small, p. 7)
    \item {\bf Important to note:} There exist a point at which the aerodynamic moment does not depend on $\alpha$. This is the the aerodynamic centre
    \item The aerodynamic centre is not the same as the centre of pressure, which is defined as the point where the aerodynamic moment is zero given a certain angle of attack $\alpha$
    \item Symmetric airfoils at $\alpha=0$ have no aerodynamic moment at all times ($M_A = 0 = const$). At the aerodynamic centre for symmetric foils results into no moment
    \item Asymmetric airfoils at $\alpha=0$ have a non-zero aerodynamic moment at all times (all angles $\alpha$)
\end{itemize}


\subsubsection*{Assessment of $C_l/\alpha$ (Correction of value through Mach Number)}

\begin{itemize}
    \item The linear part of the lift curve is characterised by the slope $a = C_l / \alpha(M) = \frac{C_l/\alpha_{M=0}}{\sqrt{1-M^2}}$
    \item The Prandtl-Glauert factor is $1/\sqrt{1-M^2}$
    \item The factor is depending on the Mach number. The slop increases with increasing $M$ (between $0$ and $1$)
    \item The dependence on $Re$ is more subtle (p. 8)
\end{itemize}    

\subsection*{Extension to wing aerodynamics (p. 8)}

Aerofoil dynamics (2D) refer to the previous topics, however the 3-D case can be also modeled by through a couple examples. A finite wing is less stable and efficient than the airfoil since the tips have vortices
on at the wing tips. These ``induce'' a velocity, which locally reduces the angle of attack. An important parameter is the so called {\bf{Aspect Ratio} $AR = b^2/S$}. 
If the wing is assumed to be of surface $S = b\cdot c$, it follows $AR = b/c$.

\begin{itemize}
    \item The lift curve can become a function of $AR$ if due to the different tips. Approximately, the lift slope $a_0$ is adjusted via 
    following formula: 
    \item $a = a_0 \frac{AR}{AR+4}$
    \item The values $a$ and $c_{m0}$ will hence be corrected with a $a^*$ and $c_{m0}^*$
\end{itemize}

\subsubsection*{Strip Theory (p.9)}

\begin{itemize}
    \item If $AS$ is very small (delta wings), the integral of multiple airfoils
    \item Define multiple airfoils stacked next to each other along the span $b$
    \item Example, the wing is an elliptical $f(y) = \sqrt{1-(\frac{y}{b/2})^2} \cdot \overline{f}_{\phi}$
    \item $f(y) = a \alpha c = C_l c$ with $c = $ chord length. 
\end{itemize}

\section*{Steady-state (static) Aeroelasticity}
\subsection*{Typical Section = 1DOF model}
2-D problem with a rigid wing. We can have multiple typical sections stacked onto each other, which would be later adding dimensionality to the variable $\theta$.
The idea is later on to model the torsional spring to be a torsional stiffness of a beam (since a real wing is actually a beam with a certain stiffness).

\begin{itemize}
    \item The torsion acts in a beam section on the shear centre, however in aeroelasticity on the elastic axis
    \item The goal of engineering is always to move the shear center to the front (comes with risk to thin out
    the rear longeron and thicken the front longeron)
    \item In equilibrium, we know that the aerodynamic forces are equal to the spring forces
    \item $M_t + L \cdot e = \theta k_\theta = (q c^2 c_{m0}+ qca\theta e)\cdot b$ (moment equations)
    \item Pitching moment $M_t$ acting on section with $k_\theta$ stiffness
    \item $k_\theta \theta = qc (C_{l/a} e (\theta + \alpha_0)+c C_{m0})$ 
\end{itemize}

\subsection*{Static Instability or Divergence}

\begin{itemize}
    \item If the elastic twist $\theta$ would become infinity for a given stiffness if the denominator of equation
    \item $\theta = qc \frac{C_{l,a} e \alpha_0 + c C_{m,0}}{k_\theta - cqC_{l,a}e}$, $\theta = \infty \Leftrightarrow$ denominator = 0
    \item If the dynamic pressure $q = \frac{k_\theta}{c C_{l,a}e} = q_{div} \rightarrow$ instable (divergence)
    \item Divergence = Static Instability
    \item $M_{tot} = (k_{\theta} -q Sae) \theta - qSa (e\alpha_0 + C_mc)$ (In equilibrium $M_{tot} = 0$) ($M_{tot} > 0$ if in anti-clockwise direction)
    \item Different interpretation: $\frac{\partial M_{tot}}{\partial \theta} \geq 0 \Leftrightarrow $ increasing total moment in section for increasing $\theta$ $ \Leftrightarrow \Delta \theta > 0$
    \item $\frac{\partial M_{tot}}{\partial \theta} \geq 0 \Leftrightarrow$ overall moment brings blade section back to original position
    \item The divergent dynamic pressure can be found by differentiating w.r.t. $\theta$ 
\end{itemize}

\subsubsection*{Lagrange Equation (Energy interpretation)}

\begin{itemize}
    \item $L = T-U$ (Kinetic and potential energy)
    \item $\frac{d}{dt}\frac{\partial L }{\partial \dot{x}_i} - \frac{\partial L}{\partial x_i} = 0$
    \item In statics: $\frac{\partial U}{\partial x_i} = 0$ (Potential energy conservative)
    \item Here: $\frac{\partial U}{\partial x_i} = \frac{\partial}{\partial x}\frac{\delta W}{\delta x}$ (Virtual work), hence for $\theta$
    \item $\frac{\partial U}{\partial \theta}= \frac{\partial}{\partial \theta}\frac{\delta W}{\delta \theta} = 0$ (Check for q)
    \item Resulting $q$ provides the divergence
    \item $U = 1/2\cdot k_\theta \theta^2 = \int F(\theta) d\theta$ (For system with only one spring)
\end{itemize}

\subsubsection*{Section with more than 1 DOF}

\begin{enumerate}
    \item Number of DOF = Dimensions of Stiffness Matrix and number of equations needed (full rank) $K$
    \item Define a potential energy matrix for mechanical system $K_{i,j} = \frac{\partial U}{\partial x_i \partial x_j} = K_{j,i}$
    \item Define aerodynamic matrix $K_a$ based on aerodynamic forces (independent on $e$ for example)
    \item If a matrix is not symmetric = Non-conservative forces
    \item Similar to before, instead of asking if the system is stable if the denominator is zero, we must know if the determinant of the transfer function is zero
    \item Transfer Function: $[K-qK_a] = K_{ael} $ is `Aeroelastic K'
    \item Find a q for which the transfer function determinant becomes zero, which is divergence dynamic pressure. The solution (forces acting) is the so called divergence mode
    \item If all eigenvalues are $>0 \Leftrightarrow$ stable, if one is at least $<0 \Leftrightarrow$ unstable (for sections)
\end{enumerate}

\subsubsection*{System (more than 1 section) with multiple DOF}

\begin{itemize}
    \item Define a system with multiple $\theta_i$, whereas the calculations become similar to when when calculating one section with multiple DOF
    \item Make an Ansatz with the lagrange equations and define stiffness matrix K
    \item The aerodynamic matrix becomes the identity matrix (if we only speak about $\theta$)
    \item This implies that the solutions for $q$ are the eigenvalues of $K$
\end{itemize}

\subsubsection*{Comment on eigenvalues}

\begin{itemize}
    \item The eigenvalues of the Aeorelastic K $K_{ael}$ cannot guarantee that $q$ are always the eigenvalues, since $K_a$ is sometimes
    non-symmetric (most of the times, only if rotational degrees of freedom present)
    \item In general, following statement holds true: \newline $det(K - qK_a) = 0$
    \item $det(K_a^{-1} K -qI) = det(A-\lambda I) = 0$ 
    \item If A is not symmetric, we can say: There are less eigenvectors and values than the order (n), can be complex and come in complex conjugate pairs
\end{itemize}

\subsection*{Active conrol on sections}

\begin{itemize}
    \item With active control, the behaviour of the elastic twist $\theta$ can be controlled with for example a trailing edge flap
    \item As described in the script, a trailing edge flap can influence the lift and the moment as follows:
    \item $l = qc C_{l/\delta} \delta$, $m = qcC_{m/\delta} \delta$ ($C_{m/\delta} < 0$)
    \item Both forces contribute to the overall moment, hence will be added to the calculations we did previously
    \item With a so called `Gain' $G$, the controller controls $\delta$ proportional to $\theta$, hence a new linear equation system is formed
    \item Assuming the nose-down motion of controlling the the edge flap, we have to simplify terms, the end result is 
    \item $q_{div,flap}= \frac{k_\theta}{cae - Gca^*}$, with $a^* = -(cC_{m/\delta}+eC_{l/\delta})/(c\delta)$
\end{itemize}

\subsection*{Ritz Method}

... is a energy variational method whereas an equilibrium occurs in correspondence of an extreme of potential energy. A general application is the virtual work.
According to the Hamilton's principle and Lagrange equations, we can define a set of equations.

\begin{itemize}
    \item $\frac{\partial V}{\partial x_i} = 0$ for all $i$ ($x_i$ degrees of freedom)
    \item From mechanics, we need the bending stiffness ($I$) and the torsional stiffness ($J$)
    \item $I = 1/12*b*h^3$ (w.r.t $x$) and $J = \frac{4 A^2}{\int ds/t}$ (Integral is perimeter (Umfang) divided by thickness)
\end{itemize}


\subsubsection*{Derivation for a beam section}

\begin{itemize}
    \item Torsional Strain Energy: $U = \frac{1}{2}\int_0^l GJ (\frac{\partial \theta}{\partial x})^2 dx$
    \item Given aerodynamic forces are non-conservative, we use the concept of virtual work
    \item $\delta U = \delta W$ (Virtual work due to non-conservative forces)
    \item $\sum_{i=1}^n \frac{\partial U}{\partial x_i} \delta x_i = \sum_{i=1}^n \delta W_i$ (reformulated for small variations of one DOF)
    \item $\frac{\partial U}{\partial x_i} - \frac{\delta W}{\delta x_i} = 0$ (reformulated)
    \item The work done by the external forces (aerodynamic) can be rewritten:
    \item $\delta W = \int_0^l m(x) \delta \theta(x) dx$ 
    \item $m(x)$ generated by aerodynamic forces
    \item Without going into further detail, there are 2 distinct cases from which one has to go on in the calculation, either the functions are given in a generalised form or in matrix form
    \item $\theta(x) = \sum_{i=1}^N \phi_i(x)a_i = [\Phi] \{a\}$
    \item $[\Phi]$ is a row vector with elements $\phi_i$!
    \item $\phi_i(x)$ are shape functions and $a_i$ are coefficients and the linear combination of those make up $\theta(x)$
    \item Finally, the overall equations result in $[K] \{a\} = \{f\}$ 
    \item $K_{i,j} = \int_0^l GJ \phi_{i_x} \phi_{j_x} dx = GJ \frac{\partial^2 U}{\partial a_i \partial a_j}$ (Stiffness matrix entries, partial derivatives w.r.t. to $x$ and $a$)
    \item $[K] = GJ \int_0^l [\Phi_x]^T[\Phi_x] dx $
    \item $f_i = \int_0^l m(x) [\Phi]_i(x) dx$ (index $i$ for each element)
    \item $\{f\} = \int_0^l m(x) [\Phi]^T dx$ (in matrix form)
\end{itemize}

Following assumptions are drawn:

\begin{itemize}
    \item Elastic axis is perfectly straight
    \item Aerodynamic center of all sections on a straight line
    \item External moments as before by aerodynamic forces: $m(x) = qcea(\theta(x)+\alpha_0)$
\end{itemize}

Replace all $\theta(x)$ with the above solutions

\subsubsection*{One single shape function (1-DOF)}



\subsubsection*{One single shape function, Bending and Twisting}

\begin{itemize}
    \item For bending, the potential energy is: $U = \frac{1}{2}\int_0^l EI \psi^2_{xx} dx$
    \item $\psi = $
\end{itemize}


\end{multicols*}
\end{document}

