\documentclass[8pt, landscape, fleqn]{scrartcl}
\setlength{\parindent}{0pt}
\usepackage[ngerman]{babel}
%\usepackage[applemac]{inputenc}
\usepackage[utf8]{inputenc}
\usepackage[dvips]{geometry}
\usepackage{latexsym}
\usepackage{multicol}
\usepackage{amsmath}
\usepackage{graphicx}
\usepackage{array}
\usepackage{booktabs}
\usepackage{amsmath}
\usepackage{mathtools}
\usepackage{ulem}
\usepackage{amsfonts}
\usepackage{dsfont}
\usepackage{charter} %%% Schreibart
%\renewcommand{\familydefault}{\sfdefault}



%%%%%%%%%%Paket für Chemische Formeln
\usepackage{chemformula} 
\usepackage[version=3]{mhchem}
%%%%%%%%%%%%%%%%% Farbe
\usepackage{color}

\pagestyle{plain}
\typearea{20}
\columnsep 30pt
\columnseprule .4pt
\setlength{\extrarowheight}{0.9em}

\renewcommand{\arraystretch}{0.8}

\makeatletter
\renewcommand{\section}{\@startsection{section}{1}{0mm}%
{-2\baselineskip}{0.8\baselineskip}%
{\hrule depth 0.2pt width\columnwidth\hrule depth1.5pt
width0.25\columnwidth\vspace*{1.2em}\Large\bfseries\rmfamily}}
\makeatother


\makeatletter
\renewcommand{\subsection}{\@startsection{subsection}{1}{0mm}%
{-2\baselineskip}{0.8\baselineskip}%
{\hrule depth 0.2pt width\columnwidth\hrule depth0.75pt
width0.25\columnwidth\vspace*{1.2em}\large\bfseries\rmfamily}}
\makeatother

\makeatletter
\renewcommand{\subsubsection}{\@startsection{subsubsection}{1}{0mm}%
{-2\baselineskip}{0.8\baselineskip}%
{\hrule depth 0.2pt width\columnwidth\vspace*{1.2em}\normalsize\bfseries\rmfamily}}
\makeatother

\newcommand{\Mx}[1]{\begin{bmatrix}#1\end{bmatrix}}
\begin{document}
\part*{\LARGE\textrm{Einführung in die Flug- und Fahrzeugaerodynamik $\hfill$ Xeno Meienberg}}
\begin{multicols*}{3}

\section{Flugtechnik}

\subsection{Atmosphäre}
\subsubsection{Allgemeine Eigenschaften}
Zusammensetzung: $\sim 78\% N_2$, $\sim 21\% O_2$, $ \sim 1\% He, H, He$
\newline \newline
\textbf{Troposphäre (0-7/17 km):} $\frac{dT}{dH} = -6.5\cdot 10^{-3} \frac{K}{m}$ \newline In ihr findet das Wetter statt
\newline \newline
\textbf{Tropopause (abhängig von Breitengrad und Jahr)}: \newline
"Aquator (17 km): $T = 191 K$ \newline 
Pole (7km): $T = 221 K$
\newline \newline
\textbf{Standardatmosphäre (11 km)}: $T_{11000} = 216.65 K$, $p_{11000} = 226.32 HPa$, $\rho_{11000} = 0.3639 km/m^3$
\newline \newline
\textbf{Stratosphäre (bis $\sim$ 50 km)}: $T = 217 K$ (direkt "uber Tropopause, max. bei 50 km)
\newline \newline
\textbf{Stratopause ($\sim$ 50 km)}: $T = 273 K$
\newline \newline
\textbf{Mesosphäre (bis $\sim$ 80 km)}: $T = 173 K$ (negativer Temp. gradient)
\newline \newline
\textbf{Thermosphäre und Ionosphäre (bis $\sim 800km$)}: $T = 1270 K$ bei $480 km$
\newline \newline
\textbf{Exosphäre (ab $800 km$)}: Führt gleitend in den Weltall
\newline \newline
\textbf{Physikalischen Eigenschaften}:
\begin{itemize}
\item $p = \rho R T$ mit $R = 287.3 J/(kg K)$ 
\item Bernoulli: $p + \frac{\rho}{2}V^2 = const$
\item Schallgeschwindigkeit: $a = \sqrt{\gamma R T}$ mit $\gamma = c_p / c_v$
\item Luft: $\gamma = 1.405$
\item $\frac{\Delta \rho}{\rho} \approx \frac{1}{2} M^2$, Machzahl $M = V/a$
\end{itemize}

\subsection{Standardatmosphäre}
\begin{itemize}
    \item $H = 0m$
    \item $T_0 = 288.15 K$, $p = 1013 HPa$, $\rho = 1.225 kg/m^3$, $g = 9.806 m/s^2$
    \item $H < 11000 m$
    \item $\frac{T}{T_0} = \Theta(H)= 1 + \frac{a}{T_0} H = 1-22.558\cdot 10^{-6} \cdot H$
    \item $\frac{p}{p_0} = \delta = \Theta^{5.2561}$
    \item $\frac{\rho}{\rho_0} = \sigma = \Theta^{4.2561}$
    \item $H = 11000 m$:
    \item $\frac{T_{11000}}{T_0} = 0.7519$, $\frac{p_{11000}}{p_0} = 0.2234$, $\frac{\rho_{11000}}{\rho_0} = 0.2971$
    \item $11000 m < H < 25000m$: 
    \item $\frac{T}{T_0}= 0.7519$, $\frac{p}{p_0} = 0.2234 \cdot e^{-\frac{H-11000}{6341.9}}$, $\frac{\rho}{\rho_0} = 0.2971 \cdot e^{-\frac{H-11000}{6341.9}}$
    \item Dynamische Zähigkeit der Luft:
    \item $\mu = (1.458\cdot 10^{-6}\cdot T^{1.5})/(T+110.4) Ns/m^2$
    \item $\mu_0 = 17.894 \cdot 10^{-6} Ns/m^2$
\end{itemize}

\includegraphics[width=8cm]{meteorologie-wetter-atmosphaere-102}

\subsection{Auftrieb}
\subsubsection{Flügelgeometrie}

\begin{itemize}
    \item Zuspitzung: $\lambda = \frac{c_t}{c_0}$
    \item Flügelfläche: $F = \int_{-b/2}^{b/2} c(y) dy$ 
    \item Streckung: $\Lambda = b^2 /F$
    \item Mittl. geome. Flügeltiefe: $\overline{c} = \frac{1}{b} \int_{-b/2}^{b/2} c(y) dy = F/b$ 
    \item Mittl. aero. Flügeltiefe: $l_\mu = \frac{1}{F} \int_{-b/2}^{b/2} c^2(y) dy $
    \item Geometrischer Neutralpunkt = Ort wo die Änderung des Anstellwinkels keine Auswirkung auf Kraft und Moment hat
    \item $x_{N25} = \frac{1}{F} \int_{-b/2}^{b/2} c^2(y) x_{25}(y) dy \approx x_{c_0/4}$, $y_{N25} = 0$
\end{itemize}

Achtung: $b_{ges} = 2\cdot b_{flugel}$!

\begin{center}
    \includegraphics[width = 6cm]{Fluegelgeometrie-1.png}
\end{center}
\begin{tabular}{ l | l l l l}
    & Recht & Trapez & Dreieck & Ellipse \\ 
    \hline
    F & $bc$ & $\frac{c_0+c_t}{2}b$ & $\frac{c_0}{2}b$ & $\frac{\pi}{4}b c_0$ \\ 
    \hline 
    $\Lambda$ & $b/c$ & $2b/(c_0+c_t)$ & $2b/c_0$ & $4b/(\pi c_0)$ \\
    \hline
    $\lambda$ & $1$ & $c_t/c_0$ & 0 & - \\
    \hline
    $\overline{c}$ & c & $(c_0+c_t)/2$ & $c_0/2$ & $\pi/4$ \\
    \hline
    $l_\mu$ & c & $\frac{2}{3}\frac{c_0^2+c_0c_t+c_t^2}{c_0+c_t}$ & $2c_0/3$ & $\frac{8}{3\pi}c_0$ \\
    \hline
    $x_{25}$ & $c/4$ & $\frac{c_0}{4}+\frac{c_0 b}{6(c_0+c_t)}(1+\frac{2c_t}{c_0})tg(\phi_{25})$ & $\frac{c_0}{4}\frac{b}{6}tg(\phi_{25})$ & $\frac{c_0}{4}\frac{b}{6}tg(\phi_{25})$ \\
    \hline
\end{tabular}

\subsubsection{Flügelprofile}

\begin{tabular}{l | l l }
    & Flügel (3D) & Profil (2D) \\ 
    \hline
    Auftrieb & $c_A = \frac{A}{\frac{1}{2}\rho V^2 F}$& $c_a = \frac{A'}{\frac{1}{2}\rho V^2 c}$  \\
    \hline
    Widerstand & $c_W = \frac{W}{\frac{1}{2}\rho V^2 F}$ & $c_w = \frac{W'}{\frac{1}{2}\rho V^2 c}$ \\
    \hline
    Nickmoment & $c_M = \frac{M}{\frac{1}{2}\rho V^2 F l_\mu}$ & $c_m = \frac{M'}{\frac{1}{2}\rho V^2 c^2}$\\
    \hline
\end{tabular}
\newline \newline
Hierbei sind Grössen mit Apostroph pro Spannbreite berechnet (Kraft/Moment pro $b$)

\begin{itemize}
    \item \textbf{Auftriebspolaren:} Nullauftriebswinkel $\alpha_0$ (Winkel wo aerodyn. Auftrieb verschwindet)
    \item $\alpha_0 = 0$ für symmetrische Profile
    \item $\alpha_0 < 0$ für gewölbte Profile
    \item \textbf{Linearbereich}
    \item $c_a = \frac{dc_a}{d\alpha}(\alpha-\alpha_0)$ mit Auftriebsgradient 
    \item $\frac{dc_a}{d\alpha}$: Konstant im Linearbereich
    \item \textbf{Maximaler Auftriebsbeiwert:} $c_{a,max}$ bestimmt die Abrissgeschwindigkeit
    \item \textbf{Minimaler Auftriebsbeiwert:} $c_{a,min}$ analog wie $c_{a,max}$ im Rückenflug
    \item \textbf{Minimaler Widerstandsbeiwert:} $c_{w,min}$ 
    \item $c_{w,min} = 0$ für symmetrische Profile
    \item $c_{w,min} > 0$ für gewölbte Profile ungefähr beim stossfreien Eintritt (tangentialle Umströmung)
    \item \textbf{Sturzflug-Momentenbeiwert} $c_{m_0} = c_m(c_a = 0)$
    \item \textbf{Bester Gleitwinkel}: $\tan(p) = \frac{1}{(\frac{c_a}{c_w})_{max}}$
    \item \textbf{Grössmögliche Reichweite:} $(\frac{c_a}{c_w})_{max}$
    \item \textbf{Beste Steigzahl / Profilsinkzahl:}
    \item $(\frac{c_a^3}{c_w^2})_{max}$ resp. $\sqrt{\frac{c_a^3}{c_w^2}}$ längste Flugdauer 
\end{itemize}

\begin{center}
    \includegraphics[width = 8cm]{Aufstiegspolaren.png}
\end{center}

\begin{itemize}
    \item \emph{Druckpunkt / Neutralpunkt}
    \item Druckbeiwert: $c_p = \frac{p-p_\infty}{1/2 \rho V^2}$
    \item Moment um beliebigen Punkt am Profil:
    \item $c_{m_x}= \frac{x-x_{DP}}{c}(c_a \cos\alpha + c_w \sin\alpha) \approx \frac{x-x_{DP}}{c}c_a$
    \item $x_{DP}$: Lage des Druckpunktes
    \item Am \textbf{Druckpunkt}: $c_{m,DP} = 0$
    \item \textbf{Neutralpunkt}: $\frac{dc_{m_x}}{d\alpha}\vert_{NP} = 0$ und $\frac{dc_{m_x}}{dc_a}\vert_{NP} = 0$
    \item $\frac{x_{NP}-x_R}{c} = -\frac{c_{mR}-c_{m0}}{c_a}$ mit $R$ als Referenzpunkt
\end{itemize}

\subsubsection{Profileigenschaften}

\begin{itemize}
    \item Auftrieb [N/m]: $A= \rho V\Gamma$ 
    \item Zirkulation $[m^2/s]$: $\Gamma = \int_0^c \gamma dx~ [m^2/s]$
    \item \textbf{Einwirbel-Modell}
    \item $A = \rho V^2 \pi c \alpha$, $\frac{dc_a}{d\alpha} = 2\pi$, $c_a = 2\pi\alpha$                                                                                                                                                                                             
\end{itemize}

\subsubsection{Profilsystematik siehe p. 3.34}

\subsubsection{Tragflügel endlicher Spannweite}

\textbf{Aerodynamische Kraft auf Flügel}

\begin{itemize}
    \item $A = \int_{-b/2}^{b/2} \rho V \Gamma(y) dy$
\end{itemize}

\textbf{Induzierter Widerstand}




\begin{itemize} 
    \item $W_i = \frac{A^2}{2\rho V^2F^*} = \frac{2}{\rho V^2 \pi}\left(\frac{A}{b}\right)^2$
    \item mit Prandtl'schem Ansatz $F^* = \frac{\pi}{4}b^2$
    \item $c_{w_i} = \frac{c_a^2}{\pi \Lambda}$ und $\alpha_i = \frac{c_a}{\pi \Lambda}$
\end{itemize}

\textbf{Einfaches Wirbelmodell (Hufeisenwirbel)}

\begin{itemize}
    \item Abwind im Hufeisenwirbel mit $-x \gg y$
    \item $w = w_{re}+ w_{li} = \frac{\Gamma}{2\pi}\left(\frac{1}{b/2-y}+\frac{1}{b/2+y}\right)$
    \item Abwind im Hufeinsenwirbel auf Flügellinie
    \item $w = w_{re}+ w_{li} = \frac{\Gamma}{4\pi}\left(\frac{1}{b/2-y}+\frac{1}{b/2+y}\right)$
    \item Auftrieb über ganze Spannweite
    \item $A = \rho V \Gamma b$
\end{itemize}
\textbf{Allgemeine induzierte Geschwindigkeit}
\begin{itemize}
    \item Halbunendlicher Wirbelfaden: $w_i = \frac{\Gamma}{4\pi a}$
    \item Unendlicher Wirbelfaden: $w_i = \frac{\Gamma}{2\pi a}$
\end{itemize}

\begin{center}
    \includegraphics[width = 5.5cm]{Druckverteilung_am_Profil.png}
\end{center}


\subsubsection{Prandtl'sche Traglinientheorie}
Zirkulationsverteilung für \textbf{elliptische Auftriebsverteilung}
\begin{itemize}
    \item $\Gamma(y) = \Gamma_0 \sqrt{1-\left(\frac{2y}{b}\right)^2}$, ind. Anstellwinkel: $\alpha_i = \frac{\Gamma_0}{2bV}$
    \item Elliptische Flügel erzeugen ... eine elliptische Auftriebsverteilung 
    \item ... einen in Spannweitenrichtung kontanten Abwind
    \item ... in Spannweitenrichtung konstanten lokalen Auftriebsbeiwert
    \item $\alpha_i = \frac{c_A}{\pi \Lambda}$ Abwindwinkel des Ellipsenflügels am Flügel selbst
    \item $c_A = c_{a_\alpha}\frac{\Lambda}{\Lambda+2}\alpha~$ Auftriebsbeiwert des Ellipsenflügels
    \item $\frac{d c_A}{d \alpha} = c_{a_\alpha}\left[\frac{1}{1+\frac{c_{a_\alpha}}{\pi \Lambda}}\right]$ Auftriebsderivativ des Ellipsenflügels
    \item $\frac{d c_A}{d \alpha} = c_{a_\alpha}\frac{\Lambda}{\Lambda+2}$ Auftriebsderivativ des Ellipsenflügels (potentialtheoretisch)
    \item $c_{W_i}=\frac{c_A^2}{\pi \Lambda}$ Induzierter Widerstand des Ellipsenflügels
\end{itemize}
\textbf{Beliebige Auftriebsverteilung}
\begin{itemize}
    \item \emph{Methode von Schrenk}: Aufteilung von Auftriebsverteilung auf Basisauftrieb und Zusatzauftrieb (A/2 elliptische Form, A/2 proportional zu Flügelgrundriss)
    \item $\frac{dc_A}{d\alpha} = c_{a_\alpha}\left[\frac{\Lambda}{\Lambda+\frac{2(\Lambda+4)}{\Lambda+1}}\right]$ (McCormick Näherung)
    \item $\frac{dc_A}{d\alpha} = c_{a_\alpha}\left[\frac{\Lambda}{\frac{c_{a_\alpha}}{\pi}+\sqrt{\left(\frac{c_{a_\alpha}}{\pi}\right)^2+\Lambda^2}}\right]$, falls $c_{a_\alpha} = 2 \pi $:
    \item $\frac{dc_A}{d\alpha} = c_{a_\alpha}\frac{\Lambda}{2+\sqrt{4+\Lambda^2}}$ (Lowry+Polhermus Näherung)
\end{itemize}

\begin{center}
    \includegraphics[width=5.5cm]{Wirbelsystem_einfaches_Hufeisenmodell.png}
    \includegraphics[width=5.5cm]{Prandtl_Wirbelmodell.png}
\end{center}

\textbf{Prandtl-Glauert Faktoren}

\begin{itemize}
    \item Prandtl-Glauert Faktoren $\tau$ und $\delta$ geben Abweichungen zum idealen Ellipsenflügel an 
    \item $\alpha_i = \frac{c_A}{\pi A}(1+\tau)$
    \item $c_{W_i} = \frac{c_A^2}{\pi A}(1+\delta)$
\end{itemize}

\subsubsection{Strömungsabriss am Flügel}
\begin{itemize}
    \item Abbrissverhalten kritischer je ausgeprägter der Auftriebsabfall nach erreichen von $c_{A,max}$
    \item Abriss bemerkbar durch Schütteln (Buffeting)
    \item Abriss erkennbar wenn Innenflügel im abgerissenen Zustand und Aussenflügel gesund umströmt
    \item Bei Trapezflügel: Abrissverhalten aussen kritischer
    \item Bei gepfeilter Flügelform: Zusätzlich kan ein Längsmoment (Pitch Up) eintreten, was zu einer Verstärkung des Abriss führt
    \item \textbf{Flügelverwindung}: Aeroelastische Antwort wobei Flügel nach aussen unten verwunden werden (-3°) damit Pilot länger Kontrolle auf Steuerruder hat
    \item \textbf{Stall Control Devices}:
    \item Absenkung der Profilnase im Ausenflügel (Drop Nose)
    \item Nasenklappen im Aussenflügel
    \item Sägezahn (bei Pfeilflügeln) zwecks Aufbau einer Grenzschicht
    \item Grenzschichtzaun, verhindert Strömungsabfluss gegen Flügelspitze
    \item Vortex-Generatoren, verzögern Ablösung im Querruder
    \item Abrisskanten (Stall Strips) am Innenflügel, lösen früher ab, Pilot wird durch Buffeting gewarnt ohne das Querruderwirksamkeit verloren geht
\end{itemize}



\subsubsection{Auftriebserhöhende Klappen}

\begin{itemize}
    \item Schnellflug: $c_{W,min}$ möglichst klein
    \item Reiseflug: $c_a/c_w$ resp. $c_a^3/c_w^2$ möglichst gross
    \item Langsamflug $c_{A,max}$ möglichst gross
    \item Um alle Anforderungen zu erfüllen, werden Klappen gebraucht
    \item Die Profilwölbung führt zu einem grösseren $c_{A,max}$ 
    \item $\frac{dc_a}{d\alpha}$ bleibt ungefähr gleich
    \item $c_a$ verschiebt sich zu grösseren Auftriebsbeiwerten
\end{itemize}


\subsection{Widerstand}

\subsubsection{Widerstandsarten}

\begin{center}
    \includegraphics[width=6cm]{Widerstandsarten.png}    
\end{center} 

\textbf{Gesamtwiderstand = Induzierter Widerstand + Restwiderstand} \newline \newline
\textbf{Restwiderstand}
\begin{itemize}
    \item Reibungswiderstand: Reibungswiderstand auf benetzter Oberfläche
    \item Formwiderstand: Druckwiderstand auf Oberfläche parallel zur Strömung
    \item Interferenzwiderstand: Widerstand durch zwei Körper nahe beieinander
    \item Trimmwiderstand: Zusatzwiderstand durch Komponenten welche zum Momentengleichgewicht benötigt sind
    \item Profilwiderstand: Reibungs- und Formwiderstand eines 2-D Profils
    \item Kühlungswiderstand: Widerstand durch Impulsverlust beim Durchströmen von Kühleinrichtungen
    \item Heckwiderstand: Druckwiderstand eines stumpfen Hecks 
    \item Wellenwiderstand: Bei Überschallströmungen, durch Schockwellen
\end{itemize}

\textbf{Induzierter Widerstand}
\begin{itemize}
    \item $c_W = c_{W_0} + c_{W_i}$ wobei $c_{W_i} = kc_A^2$ (elliptisch: $k = \frac{1}{\pi \Lambda}$)
    \item $W = \frac{\rho}{2} V^2 F c_{W_0} + \frac{\rho}{2}V^2Fkc_A^2$ (dimensionsbehafte Form)
    \item Stationärer Horizontalflug ($A = mg$ und $c_A = \frac{2mg}{\rho V^2 F}$ )
    \item $W = \frac{\rho}{2} V^2 F c_{W_0} +  \frac{k(mg)^2}{\frac{\rho}{2}V^2F}$
\end{itemize}

\subsubsection{Restwiderstand des Flügels}
\textbf{Profilwiderstand}(Widerstand des Flügels) - p. 4.47 ff 


\begin{itemize}
    \item $W_{Fl"ugel} = W_{Rest,Fl} + W_{induziert}$
    \item $W_{Rest,Fl} = 2 \frac{\rho}{2}V^2 \int_{b_R/2}^{b/2} c_{W_\infty}(y)c(y)dy$  
    \item $c_{W_\infty}$ Profilwiderstand 2D
\end{itemize}

Unverwundener Ellipsenflügel:

\begin{itemize}
    \item $W_{Fl"ugel} = c_{W_\infty} \frac{F^*}{F}+c_{W_i}$ 
    \item $F^*$ benetzter Anteil Flügelfläche
\end{itemize}

Oswald-Faktor

\begin{itemize}
    \item $c_W = c_{W_0}+ \frac{c_A^2}{\pi \Lambda e}$ mit 
    \item $e = \frac{1}{1+\delta+\pi\Lambda k}$ Oswald-Wirkungsfaktor
    \item Gilt nur im Linearbereich der Auftriebspolaren!
    \item Flügel mit elliptischer Auftriebsverteilung: $e=1$
    \item Flügel $0.85<e<0.95$
    \item Flugzeug $0.6<e<0.9$
\end{itemize}

Bester Gleitwinkel $(c_A/c_W)_{max}$

\begin{itemize}
    \item $c_W = 2c_{W_0}$, $c_A = \sqrt{\pi \Lambda e c_{W_0}}$, $c_{W_0} = c_W(c_A = 0)$
\end{itemize}

\subsubsection{Restwiderstand des Flugzeugs}

\textbf{\emph{Reibungswiderstand} $W_R$}

\begin{itemize}
    \item $c_f = \frac{W_R}{\frac{\rho}{2}V^2F_W}$ mit $F_W$: benetzte (überstr.) Oberfläche
    \item Lokale Reynoldszahl: $Re_x = \frac{V x}{\nu}$
\end{itemize}

\textbf{Umschlag von laminar-turbulent}

\begin{itemize}
    \item $Re_{krit} = (V x / \nu)_{krit} \approx 3\cdot 10^5-3\cdot 10^6$
    \item laminar $Re < Re_{krit}$
    \item turbulent $Re > Re_{krit}$
\end{itemize}

\textbf{Ebene Platte mit glatter Oberfläche}

\begin{itemize}
    \item $Re = (V l /\nu) $ mit $l$ Plattenlänge
    \item laminar: $c_f = 1.328 \frac{1}{\sqrt{Re}}$
    \item turbulent: $c_f = 0.074 Re^{-1/5}$
\end{itemize}

\textbf{Ebene Platte mit rauher Oberfläche}
\newline Rauhigkeit $k_s$
\begin{itemize}
    \item $k_s = 0 mm$ - Aerodynamisch/hydraulisch glatt
    \item $k_s = 0.5\cdot10^{-3} - 2 \cdot 10^{-3} mm $ - Metall/Holz poliert 
    \item $k_s = 6\cdot 10^{-3} mm$ - Farboberfläche, glänzend
    \item $k_s = 0.01-0.03 mm$ - Tarnfarbe, unpoliert
    \item $k_s = 0.15 mm$ - Metalloberfläche, galvanisierend
\end{itemize}
Rauhigkeitsbereiche:
\begin{itemize}
    \item $\frac{u_\tau k_s}{\nu} < 5$ hydraulisch glatt
    \item $5 <\frac{u_\tau k_s}{\nu} < 70$ Übergangsbereich
    \item $\frac{u_\tau k_s}{\nu} > 70$ rauh 
\end{itemize}
Zulässige Rauhigkeitshöhe $k_{s,zul}$ für Grenzschichten
\begin{itemize}
    \item laminare GS: $k_{s,zul} \leq 15 \frac{u_\tau}{\nu} = k_{s,krit} = 26.03 \frac{\nu \sqrt[4]{Re_x}}{V}$
    \item turbulente GS: $k_{s,zul} < 100 \frac{\nu}{V} = 100 \frac{l}{Re}$
    \item $c_f = (1.89 + 1.62 \log(\frac{1}{k_s}))^{-2.5}$ für $10^2 < \frac{1}{k_s} < 10^6$
\end{itemize}

\textbf{Plattenförmige Körper ohne grosse Ablösungssgebiete}

\begin{itemize}
    \item $c_W^* = c_f \frac{F_W}{F_F}$
    \item Benetzte Oberfläche $F_W$, Frontfläche $F_F$
\end{itemize}

\textbf{Reibungswiderstand für profilierte Flächen (empirische Beziehung)}

\begin{itemize}
    \item $c_{W_0} = c_f \frac{F_F}{F}\left[1 + L\left(\frac{d}{c}\right)+100\left(\frac{d}{c}\right)^4\right]$
    \item mit $L = 1.2$: Falls Profil max. Dicke $x/c > 0.3$
    \item mit $L = 2.0$: Falls Profil max. Dicke $x/c < 0.3$
    \item $F$: Referenzfläche, $F_W$: Benetzte Oberfläche
    \item $c$: Profiltiefe, $d$: Profildicke
\end{itemize}

\textbf{Formwiderstand}

\begin{itemize}
    \item $c_{W,Ru}(\alpha) = c_{W_{0,Ru}}+c_{W_{\alpha,Ru}}+c_{W_{H,Ru}}$
    \item $c_{W,Ru} = 0.05-0.15$ \newline $0.15$ für kleine, gedrungene Flugzeuge
    \item $c_{W,Ru} = \left( 1 + \frac{D}{2l}\right)c_{f,pl}\frac{F_W}{F}$ \newline $c_{f,pl}$ Reibungsbeiwert Platte
    \item $c_{\alpha,Ru} \approx k_R \left(\frac{\alpha}{10}\right)^3 c_{W_{0,Ru}}$ \newline $k_R \approx 0.3$ (gedrungen), $k_R \approx 0.9$ (schlank)
    \item $c_{W_{H,Ru}} = 0.029 \left(\frac{D_H}{D}\right)^3 \frac{1}{\sqrt{c_{W_{0,Ru}}}}\frac{\pi D^2}{4}\frac{1}{F}$ \newline $D$: max. Rumpfdurchmesser, $D_H$: Heckdurchmesser
\end{itemize}

\textbf{Interferenzwiderstand}

\begin{itemize}
    \item $\approx 5\%$ des Rumpfwiderstands bei kleinen Anstellwinkeln, durch Messungen zu bestimmen
\end{itemize}

\textbf{Trimmwiderstand}

\begin{itemize}
    \item $\approx max. 1-2 \%$ des Gesamtwiderstands im stationären Reiseflug
\end{itemize}

\textbf{Abschätzung des Restwiderstands}

\begin{enumerate}
    \item Einzelteile auflisten
    \item Geometrie der Einzelteile bestimmen
    \item Referenzfläche $F_N$ bestimmen und Widerstandsbeiwert $c_{W_n}$ abschätzen
    \item Widerstandsfläche der Einzelteile berechnen: $f_n = c_{W_n}F_n$
    \item Widerstandsfläche des Flugzeugs bestimmen: $f = \sum_{i=1}^n f_i$
    \item Abschätzen von allfälligen Zusatzwiderständen (Interferenzen, Kühlung)
    \item Gesamtwiderstand: $W_{Rest} = \frac{1}{2}\rho V^2 f +W_{zusatz}$
\end{enumerate}

\subsubsection{Gesamtwiderstand des Flugzeugs}

\begin{itemize}
    \item $W = \frac{1}{2}\rho V^2 \left(f + F \frac{c_A^2}{\pi \Lambda e}\right)$
    \item Im stationären Horizontalflug: $A = mg$
    \item $W = \frac{1}{2}\rho V^2 f + \frac{2}{\rho \pi e}\left(\frac{mg}{b}\right)^2\frac{1}{V^2}$
\end{itemize}

\textbf{Minimaler Widerstand}

\begin{itemize}
    \item $W_{min} = \frac{2mg}{b}\sqrt{\frac{f}{\pi e}}$
    \item $V(W_{min}) = \left[\frac{4}{\pi e f}\left(\frac{mg}{\rho b}\right)^2\right]^{0.25}$
\end{itemize}

\subsubsection{Widerstandsverminderung}

\textbf{Reduktion des induzierten Widerstands}

\begin{itemize}
    \item $c_{W_i} = \frac{c_A^2}{\pi \Lambda e} \rightarrow$ möglichst grosse Streckung $\Lambda$
    \item Möglichst elliptischer Auftrieb $e = 1$
    \item Durch Beeinflussung der Ausgleichsströmung (Flügelend-Tanks, Winglets etc.)
\end{itemize}

\textbf{Reduktion des Restwiderstand (p. 4.36)}

\begin{itemize}
    \item Reduktion der Oberflächenreibung durch Laminarhaltung der Strömung
    \item Reduktion der Oberflächenreibung durch Reduktion der Rauheit 
    \item Grenzschichtbeeinflussung durch passive oder aktive Mittel (Grenzschichtabsaugung, Zusatzinstallation etc.)
    \item Beeinflussung der Grenzschicht durch Riblets
    \item Verringerung des Kleinteilewiderstands (Drag clean up), Beschränkung von störenden Teilen auf der Oberfläche auf ein Minimum
\end{itemize}



\section{Einführung in die Fahrzeugaerodynamik}
\subsection{Grundlagen}

\subsubsection{Kräfte am Fahrzeug}

    \begin{itemize}
        \item Widerstandskraft: $W= -(F_{xa}\cos{\beta}- F_{ya}\sin{\beta})$
        \item $F_i = c_{Fi}\frac{1}{2}\rho V^2 A $
        \item Seitenkraft: $F_y = (F_{xa} \sin{\beta}+ F_{ya}cos{\beta})$ 
        \item Auftrieb: $A = A_y + A_H$
        \item Rollwiderstand: $W_R = W_{RV} + W_{RH}$
        \item $F_R = \mu_R F_Z$ mit $\mu_R = \mu_{R, Basis} / \mu(V=0)$
        \item Beschleunigungswiderstand: $W_B = \dot{V}m(1+\varepsilon_i)$
        \item $\varepsilon_i =$ rotierende Massen (Gänge), $\varepsilon_1 = 0.25$, $\varepsilon_2 = 0.15$, $\varepsilon_3 = 0.1$, $\varepsilon_4 = 0.075$
        \item Steigungslasten $W_S = mg \sin\theta$
        \item Anhängelasten $F_{zx}$ und $F_{zz}$ horizontal und vertikal
    \end{itemize}


    \begin{center}
        \includegraphics[width=4cm]{Koordinaten_Schraegstroemung.png}
    \end{center}


\subsubsection{Fahrleistungen}
\textbf{Kräftegleichgewicht am beschleunigten Fahrzeug}
\begin{itemize}
    \item $m a_x = F_T - \mu_r (mg - \frac{1}{2}\rho V^2 c_A A- mg \sin\theta)$
\end{itemize}
\textbf{Traktionskraft}
\begin{itemize}
    \item $F_T = \mu_{tan} (m_i g \cos\theta -1/2 \rho V^2 c_{A,i}A)$
    \item Index $i$ für Vorder- und Hinterachse
\end{itemize}

\textbf{Traktionsleistung}

\begin{itemize}
    \item $P_T = F_T V$
\end{itemize}

\textbf{Kurvenfahrleistung}

\begin{itemize}
    \item Max vom Fahrreifen übertragbare Seitenkraft: \newline $F_{lat} = \mu_{lat}(mg - 1/2 \rho V^2A c_A)$
\end{itemize}

\textbf{Kräftegleichgewicht im beschleunigten Fahrzeug}

\begin{itemize}
    \item $mV^2 / R = F_{lat,max} = \mu_{lat,max}(mg- 1/2 \rho V^2 A c_A)$
\end{itemize}

\textbf{Maximale fahrbare Geschwindigkeit (Kurve mit Radius $r$)}

\begin{itemize}
    \item $V_{max} = \sqrt{\mu_{lat,max} R(g-\rho V^2Ac_A/(2m)} =$ \newline $= \sqrt{\mu_{lat,max} R(1-F_A/(mg))}$
\end{itemize}

\textbf{Fahrstabilität}
\newline \newline
Anzustreben ist eine ausgeglichene Auftriebsverteilung an Vorder- und Hinterachse. Bei deutlich grösserem Auftrieb auf der Hinterachse
kann es zu Übersteuern kommen. Auftriebsminderung an der Hinterachse ist deutlicher schwieriger als an der Vorderachse.
\newline \newline
\textbf{Bremsverhalten}

\begin{itemize}
    \item Bremskraft: \newline $F_{b,tot} = ma_x = F_{b,mech} - W_R - W - mg \sin{\theta}$
    \item Sinnvolle Vorzeichen, $F_{b,mech}$ meist negativ da in $-x$-Richtung
    \item Abbremsung: $Z = F_{b,tot} / (mg)$
\end{itemize}

\subsubsection{Treibstoffverbrauch}

\textbf{Treibstoffverbrauch} in unbeschleunigter Fahrt auf ebener Strasse

\begin{itemize}
    \item $B = (P_{aero} + P_{roll} + \Delta P_{mech})(SFC)(1000/\rho_{fuel})(100/V) [l/100km]$
    \item $P_{aero} = W\cdot V$, $P_{roll} = W_{roll}\cdot V$, $\Delta P_{mech}$: mech. Verluste
    \item $SFC$: Spezifischer Treibstoffverbrauch (Euromix: 44\% Luft, 44\% Roll, 12\% Beschleunig.) 
\end{itemize}

\subsection{Personenwagen}

\textbf{Hauptziele der Aerodynamik}:

\begin{itemize}
    \item Reduzierung Kraftstoffverbrauch, Emissionen
    \item Reduzierung Seitenwindempfindlichkeit
    \item Sicherstellung der Motorkühlung, Limitierung der Bauteiltemperaturen
    \item Verminderung von Schmutzablagerungen im Durchsichtbereich
    \item Fahrgastraumkonditionierung (A/C) und Geräuschminderung
\end{itemize}

\subsubsection{Aerodynamik der Grundform}

Die widerstandsgünstigste Form bei kleinem Bodenabstand ein Halbkörper. Anfügen von unverschalten Rädern verdoppelt ca. den Widerstand \newline \newline
\textbf{Fahrzeugformen}

\begin{itemize}
    \item Stufenheck: siehe 3.9
    \item Schrägheck: Erzeugt einen grossen Auftrieb, Giermoment wächst grösserer Seitenanströmung nicht mehr an, grosses negatives Längsmoment
    \item Vollheck: Reagiert etwas empfindlicher bezüglich Widerstandsanstieg durch Schräganströmung, wesentlich kleinerer Auftrieb, grösster Anstieg der Seitenkraft mit Schräganströmung, kleinster Hinterachsauftrieb
\end{itemize}


\textbf{Druckverteilung}

\begin{center}
        \includegraphics[width=4.5cm]{Druckverteilung_Personenwagen.png}
\end{center}


\subsubsection{Komponentengestaltung}
\textbf{Farzeugsfront}: Annäherung - Quader / Tiefliegender Staupunkt = weniger Widerstand \newline \newline
\textbf{Frontscheibe}: Je flacher desto günstiger für Widerstand und Auftrieb. Grenzen durch Aufheizung des Innenraums und Sonnereflexionen \newline \newline
\textbf{Dach}: Übergang Front-Heck, mit leichter Verwölbung sinkt der Widerstand, bei starker Verwölbung steigt Anströmfläche \newline \newline
\textbf{Fahrzeugheck}: Geprägt von Ablösung, verschiedene Hecktypen: Stufenheck, Schrägheck, Vollheck

\begin{center}
    \includegraphics[width=8cm]{Fahrzeugheck_Personenwagen.png}
\end{center}

\textbf{Unterboden}: Durch glatte Unterbodengestaltung kann eine Widerstandsreduktion von bis zu $\Delta c_W \approx -0.05$ erreicht werden. Mit einem Heckdiffusor kann der Hinterachsauftrieb gesenkt werden \newline \newline
\textbf{Frontspoiler}: Widerstandsverkleinerung, Vorderachsauftriebsverkleinerung, Verbesserung der Kühlluftströmung \newline \newline
\textbf{Heckspoiler}: Widerstandverringerung, Hinterachsauftrieb, Reduzierung der Heckverschmutzung \newline \newline
\textbf{Durchströmung}: Versorgt Motor und Fahrgastraum mit Luft, erzeugt Widerstand und Auftrieb (p. 3.22 ff) \newline \newline
\textbf{Dachaufbauten}: Verursachen immer einen Zusatzwiderstand. Sekundäreffekte können grosse Wirkung haben (Beeinflussung der Umströmung). Dachlasten können Widerstände im Bereich um ein Drittel des Gesamtwiderstands erhöhen. Meist tritt durch Dachlasten eine Verringerung des Auftriebs durch gestörte Umströmung ein.

\subsubsection{Verschmutzung und Benetzung}

Können Fahrkomfort (Ästhetik und Verschmutzung beim Be- und Entladen) und Sicherheit (Sicht, Scheinwerfer) beeinträchtigen, werden durch aerodynamisch günstige Gestaltung vermieden

\subsubsection{Windgeräusche}

Werden erzeugt durch schnelle Druckschwankungen, die Intensität hängt vom quadratischen Wert der Schwankung auf (RMS-Wert/ root-mean square): $RMS = \sqrt{\frac{1}{n}\sum x_i^2}$ \newline \newline

\textbf{Schalldruckpegel}: $L_p = 10 \log(p/p_0)^2 = 20 \log(p/p_0) [dB]$ \newline $p_0 = 2\cdot 10^{-4}$

\begin{itemize}
    \item Hörbarer Bereich: $0$ bis ca. $140 dB$ 
    \item Addition: Zwei gleiche Lärmquellen - Erhöhung um $3 dB$ 
    \item Verdopplung der Distanz: $-6 dB$
\end{itemize}

\textbf{Strouhal-Zahl}: $S = f\cdot D/V_{\infty}$ ($f$: Frequenz, $D$: Dimension) \newline
Im Bereich $1000 \leq Re \leq 10^5$ gilt $S \approx 0.2$ \newline \newline

\emph{A-Säulen-Strömungslärm}: Stärker auf der Wind abgewandten Seite \newline \newline 

\emph{Kavitätslärm}: Entsteht an Kavitäten wie Türübergängen oder Schiebedächern \newline \newline

\emph{Wirbellärm}: Durch Heckwirbel oder Anbauten, Lärm ungefähr linear in Anströmgeschwindigkeit. Grössenordnung von $0.002$ bis $0.03 m$

\subsection{Nutzfahrzeuge}


\subsubsection{Lastkraftwagen}


\textbf{Symmetrische Antrömung}: Starke Wechselwirkung zwischen Fahrerhaus und Nutzlastaufbau. Druckverlauf siehe $Fig.~4.8, p.~4.6$

\begin{center}
    \includegraphics[width=8cm]{LKW_Druck.png}
\end{center}

\textbf{Schräganströmung}: Grosser Widerstand und Seitenkraft durch Durchströmung zwischen Fahrerhaus und Nutzlastaufbau. Reduktion um $~10\%$ Widerstand un $~30\%$ Seiten- oder Trennwänden erreicht werden. Widerstandsanteil siehe $Fig.~4.12, p.~4.9$. \newline \newline
\textbf{Formgestaltung}: 
\begin{itemize}
    \item Geringe Raddurchmesser: Verminderung Widerstand
    \item Verrundung der Frontscheibe: Verminderung Widerstand (Verrundung der Kanten $> 150 mm$)
    \item Im Bereich der Strömungsumlenkung keine Störquellen wie Türfugen. 
    \item Seitenspiegel aerodynamisch ausgelegt oder durch Kameras ersetzt 
    \item Oberflächenwiderstand bis zu $20\%$, deshalb benetzte Oberfläche glatt halten
    \item Gute Gestaltung von Stossstangen, Frontspoiler und Lufteinöffnungen
    \item Aerod. Optimierung des Fahrerhauses nicht isoliert betrachen 
\end{itemize}

\textbf{Widerstandsreduzierte Anbauteile}: Auswirkungen auf den $c_W$ siehe $Fig.~4.16$, $p.~4.12$

\subsubsection{Busse}

\textbf{Typische Strömungs- und Druckverhältnisse}: Meist quaderähnliche Formgebung aus Platzanspruchsgründen, trotzdem kleinere Optimierungen können $c_W$ im Bereich der Personenwagen ($c_W \approx 0.3$) erreicht werden, siehe $p.~4.15~ff$ 

\begin{center}
    \includegraphics[width=8cm]{Druck_Bus_Ohne_Aerodyn.png}
\end{center}

\textbf{Strömung und Formgebung des Fahrzeugbugs}: Abrundung der Stirnradien um $150mm$ genügen um selbst mit vollständig stromlinienförmiger Optimierung kein wesentlicher Gewinn mehr zu erzielen

\begin{center}
    \includegraphics[width=8cm]{Druck_Bus_Mit_Aerodyn.png}
\end{center}

\textbf{Strömung und Formgebung des Fahrzeughecks}: Geprägt durch grossflächige Vollheckablösung. Verminderung durch seitliche und vertikale Formeinzüge. Diese Massnahme ist nur bei sauberer Seiten- und Dachströmung wirksam. Die Frontgestaltung beeinflusst diese Umströmung wesentlich!

\subsection{Rennfahrzeuge}

\subsubsection{Grundlagen}

\textbf{Aerodynamische Relevanz}: 

\begin{itemize}
    \item Hohe aerodynamische Abtriebskräfte vergrössern Adhäsion auf der Strasse
    \item Verbessertes Fahrverhalten durch bewusst gesteuerte Abtriebtskräfte auf Vorder- und Hinterachsen 
    \item Kleinerer Widerstand für erhöhte Spitzengeschwindigkeit
    \item Ausreichend Kühlluftzufuhr für Motor, Antriebskomp., Bremsen udn Fahrer 
\end{itemize}

\textbf{Maximal fahrbare Geschwindigkeit} in einer Kurve mit Radius r:
$V_{max} = \sqrt{\mu_{lat,max}R(g-\rho V^2 A c_A / (2m))} = \sqrt{\mu_{lat,max} R (1- F_A/(mg))}$

\subsubsection{Tourenwagen}

Tourenwagen sind von Serien-Personenwagen abgeleitete Rennfahrzeuge. \newline \newline

\textbf{Optimierung}: $(c_A / c_W)_{max}$, analog zur Gleitzahl \newline \newline

\textbf{Typische Werte}: $c_W \approx 0.35 - 0.45$, $c_A \approx -0.5$ bis $-1.0$, $-c_A/c_W \approx 2-3$ \newline \newline

\textbf{Skirts}: Unterdruck im Unterboden bewirkt Suagkraft, kann verstärkt werden durch Abdichtung zum Boden (Skirts) \newline \newline

\textbf{Fahrzeugfront}: Möglichst saubere Umströmung (ablösefrei). Gezielte Ablösung durch Spoiler im unteren Bugbereich. Bugform bestimmt weitgehend die Grösse des Vorderachsenantriebs udn die Eintrittsbedingungen für die Unterbodenströmung.  \newline \newline

\textbf{Unterboden}: Geprägt durch Bugform, seitliche Abdeckung im Heckdruck. Möglichst nahe am Boden oder gut abgedichtet zum Boden, möglichst glatt. Mit Optimierung kann eine Abtriebserhöhung (Hinterachse) bis zu $5\%$ bei gleichzeitig gerigerem Widerstand erzeugt werden \newline \newline

\textbf{Heckflügel}: Umströmter Körper in Bodennähe erzeugt im hinteren Bereich Auftrieb. Heckabtrieb ist daher wesentlich schwieriger als Frontabtrieb. Ein Heckflügel
erzeugt Abtrieb (bei möglichst geringem Widerstand) und kann auch die Unterbodenströmung positiv beeinflussen (kleineres Druckniveau am Heck). Oft sehr komplexe Umströmung durch Störungen! \newline \newline

\subsubsection{Rennfahrzeuge mit abgedeckten Rädern}

\textbf{Typische Werte}: $c_W \approx 0.3 - 0.8$, $c_A \approx -1.0$ bis $-3.0$, $-c_A/c_W \approx 2 - 4$ \newline \newline

\textbf{Fahrzeugfront}: Saubere Strömungsgestaltung, grosser Vorderachsantrieb, kleiner Bodenabstand (meist keine Frontspoiler nötig). Mittels seitlich angebrachten 
Strakes (Leitbleche) dienen oft zur Feinabstimmung bereits ausgeführter Modelle. \newline \newline

\textbf{Unterboden}: Möglichst hoher Unterdruck am Unterboden, glatte Oberfläche. Druckniveau am Unterboden wird weitgehend durch die Austrittsbedingung am Heck geprägt. 
Durch einen Heckdiffusor lässt sich der Druck im Heck weiter absenken.

\begin{itemize}
    \item Aktive Unterdruckerzeugung: Z.B. mittels Absaugen
    \item Passive Unterdruckerzeugung: Z.B. mittels Skirts 
\end{itemize}

\subsubsection{Rennfahrzeuge mit offenen Rädern}

\textbf{Typische Werte}: $c_W \approx 0.8-1.5$, $c_A \approx -2.5$ bis $-3.5$, $-c_A/c_W \approx 2 - 3$ 
Ungefähr die Hälfte des Abtriebs kommt von Front- und Heckflügel. Sehr komplexe Strömungsverhältnisse durch Wirbel im Radbereich.\newline \newline

\textbf{Frontflügel}: Bauvolumen oft beschränkt, trotzdem hoher Abtrieb, deshalb oft Hochabtriebskonfigurationen mit grosser Wölbung und Klappen. Der Frontflügel darf die Luftzufuhr zum Unterboden
und den Kühlluftfluss nicht blockieren, deshalb meist Aussparungen in der Flügelmitte \newline \newline

\textbf{Unterboden}: Weniger uniforme Zuströmung als mit geschlossener Karosserie, nutzbare Bodenfläche ist kleiner. Optimale Einstellung durch Front- und Heckflügel stark 
beeinflusst. Unterboden mit Diffusor liefert grössere Abtriebswerte

\begin{center}
    \includegraphics[width=8cm]{F_1_Wirbel.png}
\end{center}

\textbf{Heckflügel}: Sehr wirksame Heckflügel erforderlich, falls keine Diffusoren im Unterboden zulässig. Üblicherweise ein oberer, dreiteiliger Flügel und unterer, einteiliger Flügel. \newline \newline

\textbf{Räder}: Der Radwiderstand ist ein wesentlicher Bestandteil des Gesamtwiderstands ($~50\%$ der Frontfläche durch Räder). Der Ablösepunkt beim rotierenden Rad verscheibt sich stromaufwärts. Der $c_W$ eines rotierenden Rads wird dadurch kleiner.  

\subsection*{Disclaimer}

Diese Zusammenfassung basiert auf den persönlichen Notizen und Zusammenfassungen früherer Jahre. Fehler sind unvermeidbar und es besteht keine Garantie dass diese Zusammenfassung vollständig komplett ist. 

\end{multicols*}
\end{document}

