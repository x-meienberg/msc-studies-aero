\documentclass[8pt, landscape, fleqn]{scrartcl}
\setlength{\parindent}{0pt}
\usepackage[ngerman]{babel}
%\usepackage[applemac]{inputenc}
\usepackage[utf8]{inputenc}
\usepackage[dvips]{geometry}
\usepackage{latexsym}
\usepackage{multicol}
\usepackage{amsmath}
\usepackage{graphicx}
\usepackage{array}
\usepackage{booktabs}
\usepackage{amsmath}
\usepackage{mathtools}
\usepackage{ulem}
\usepackage{amsfonts}
\usepackage{dsfont}
\usepackage{charter} %%% Schreibart
%\renewcommand{\familydefault}{\sfdefault}



%%%%%%%%%%Paket für Chemische Formeln
\usepackage{chemformula} 
\usepackage[version=3]{mhchem}
%%%%%%%%%%%%%%%%% Farbe
\usepackage{color}

\pagestyle{plain}
\typearea{20}
\columnsep 30pt
\columnseprule .4pt
\setlength{\extrarowheight}{0.9em}

\renewcommand{\arraystretch}{0.8}

\makeatletter
\renewcommand{\section}{\@startsection{section}{1}{0mm}%
{-2\baselineskip}{0.8\baselineskip}%
{\hrule depth 0.2pt width\columnwidth\hrule depth1.5pt
width0.25\columnwidth\vspace*{1.2em}\Large\bfseries\rmfamily}}
\makeatother


\makeatletter
\renewcommand{\subsection}{\@startsection{subsection}{1}{0mm}%
{-2\baselineskip}{0.8\baselineskip}%
{\hrule depth 0.2pt width\columnwidth\hrule depth0.75pt
width0.25\columnwidth\vspace*{1.2em}\large\bfseries\rmfamily}}
\makeatother

\makeatletter
\renewcommand{\subsubsection}{\@startsection{subsubsection}{1}{0mm}%
{-2\baselineskip}{0.8\baselineskip}%
{\hrule depth 0.2pt width\columnwidth\vspace*{1.2em}\normalsize\bfseries\rmfamily}}
\makeatother

\newcommand{\Mx}[1]{\begin{bmatrix}#1\end{bmatrix}}
\begin{document}
\part*{\LARGE\textrm{Satelitengeodäsie - Zusammenfassung $\hfill$ Xeno Meienberg}}
\begin{multicols*}{3}

\section{Bezugssysteme}

\subsection{Einführung}

\subsubsection{Zeit und Bezugssystem}

\begin{itemize}
    \item Newtonsche Mechanik: absolute Zeit $t$ $\rightarrow$ gleiförmig und unabhängig vom Koord.system
    \item Mathematisch: Die unabhängige Variable im Raum sämtlicher Bewegungen ist die Zeit
    \item Koordinatensystem: Definiert durch einen Urpsprung $O$ + drei orthogonale Einheitsvektoren $e_1$, $e_2$, $e_3$
    \item $\vec{e_1}$,$\vec{e_2}$, $\vec{e_3}$ = Basisvektoren, wobei $\vec{x}$ eine Linearkombination aller drei Komponenten ($x_1$,$x_2$,$x_3$) des Punktes $\vec{x}$ sind 
    \item Die Bezeichnung ``Vektor'' ist eigentlich falsch, da ein Vektor unabhängig von Koordinatensystem ist. Koordinaten sind von einem Bezugssystem abhängig
    \item Post-newtonsche Formalismen sind Bezugs- und Zeitsysteme, Frequenzen, Phasen- und Laufzeitdifferenzen, welche ebenfalls wichtig sind
\end{itemize}

\subsubsection{Ortsvektor, Bahnkurve, Weltlinie}

Die \textbf{Kinematik} eines Massenpunktes wird beschrieben durch den Ortsvektor $\underline{x}(t)$ als Funktion der Zeit: $\underline{x}(t) = (x_1(t),x_2(t),x_3(t))$. \\


Die Gesamtheit aller Endpunkte der Ortsvektoren nennt man \textbf{Bahnkurve}. Die Funktionen $x_1$,$x_2$,$x_3$ sind die Koordinaten des Massenpunktes im gewählten \textbf{Bezugssystems} $\{O,~e_1,~e_2,~e_3\}$ \\

Anstatt Parameterdarstellung $\underline{x}=\underline{x}(t)$ kann auch die Kurve im vierdimensionalen Raum dargestellt werden ($t =$ 4. Achse). Daraus ergibt sich die \textbf{Weltlinie} $\underline{x}(ct,~x_1,~x_2,~x_3)$

\subsubsection{Referenzsysteme und Referenzrahmen}

\begin{itemize}
    \item \textbf{Koordinatensystem}: Alle Grössen, die notwendig sind, um Punkte eindeutige Koordinaten zuordnen zu können (Ursprung, Achsen, Massstab)
    \item \textbf{Referenz- und Bezugssystem}: Erweiterung des Koordinatensystems durch z.B. eines Bezugsellipsoids oder Modellschwerefeld oder Gravitationskonstante
    \item \textbf{Datumsdefinition}: Festlegung von DOF (Freiheitsgrade), die nicht aus den Messungen selbst abgeleitet werden können. Hinreichende Menge von ausgewählten Grössen muss vorhanden sein 
    \item \textbf{Koordinatenrahmen}: Wahl des Systems (Koordinatensystem) + Datumsdefinition $\rightarrow$ Koordinatenlisten für Referenzpunkte
    \item \textbf{Referenzrahmen oder Bezugsrahmen}: Realisierung des Koordinatensystems + Datumsdefinition
    \item \textbf{Referenzsystem}: Konzept eines idealen Referenzsystems basierend auf abstrakten Prinzipen (Mathematik)
    \item \textbf{Konventionelles Ref.system}: Mathematik reicht nicht aus, da Faktoren wie Gezeiten, Lovesche Zahlen (Erdrigidität) berücksichtigt werden müssen
    \item \textbf{Konventioneller Ref.rahmen}: Konkrete Realisierung durch Punkte wie Stationen, Sterne, Quasare mit Koordinaten
\end{itemize}

In der Geodäsie gibt es drei Arten von Bezugssystemen:

\begin{itemize}
    \item \textbf{Raumfeste Systeme}: Inertialsystem (gute Annäherung) $\rightarrow$ zälestisch = celestial
    \item \textbf{Erdfeste Systeme}: Fest mit Erde verbunden und rotiert mit dieser
    \item \textbf{Lokale Systeme}: An ein Messinstrument gekoppelt oder orientiert sich an einer Referenzfläche
\end{itemize}

\subsection{Raumfeste Bezugssysteme}

Inertialsystem, welche keinen Scheinkräften ausgesetzt sind, es gelten allgemeine Kräfte wie: $F = m\cdot a$. Es gibt zwei Ansätze hierzu:

\begin{itemize}
    \item \textbf{Dynamischer Ansatz}: Trajektorien von Planeten udn Ssystem sind Differentialgleichungssysteme. Die Gleichungen gelten nur in einem Inertialsystem
    \item \textbf{Kinematischer Ansatz}: Annahme: Universum rotiert nicht (fixe Punkte). Aus diesem Grund ist es ein Inertialsystem. Galaktische Objekte bewegen sich nicht allzu fest
\end{itemize}

Es ist nicht ausschliessbar, dass unser Sonnensystem sich bewegt, und sich linear im Raum bewegt. Man nennt dies in diesem Fall ein \textbf{Quasiinertialsystem}

\subsubsection{Grundbegriffe}
\begin{itemize}
    \item \textbf{Epoche}: Zeitpunkt, auf den sich Koordinaten, Bahnelemente oder Ephemeriden (tabellierte Positionen von Sternen etc. beziehen
    \item \textbf{Äquatorebene:} Ausdehnung des Äquators in den Raum und senkrecht zur Rotationsachse der Erde
    \item \textbf{Ekliptik:} Umlaufebene der Erde um die Sonne. Diese ist relativ zur Äquatorebene verschoben um die Schiefe der Ekliptik
    \item \textbf{Rotationspol der Erde:} $P_N$: Schnitt der Koordinate Z im Raumfesten Bezugssystem (Rotationsachse)
    \item \textbf{Pol der Ekliptik:} $P_\Pi$: Pol , welcher sich mit der Ekliptik verschiebt
    \item \textbf{Schiefe der Ekliptik:} $\epsilon \approx 23.4^{\circ}$
    \item \textbf{Frühlingspunkt}: Ist der Punkt, wo sich Ekliptik und und Äquatorebene schneidet bevor die Ekliptik positiv wird (von Süden nach Norden). Bei der Nordhalbkugel ist dies der 19., 20. oder 21. März
    \item \textbf{Herbstpunkt}: Analog zu Frühlingspunkt, jedoch wird die Ekliptik hier von Norden nach Süden verlaufen (22. oder 23. September)
    \item \textbf{Äquinoktien}: Sind Frühlings- und Herbstpunkt der Erde (analog bei anderen Himmelskörpern) so kommt es zur Tag- und Nachtgleiche. Die Ekliptik und Äquatorebene gleich und die Halbkugel Nord und Süd wird komplett von der Sonne bestrahlt
\end{itemize}

\subsubsection{International Celestial Reference System}

Das \textbf{International Celestial Reference System (ICRS)} ist kinematisch definiert und ist ein raumfestes System. Die Achsenorientierung ist ausgelegt auf sehr weite Objekte im Universum, da diese praktisch kaum eine Eigenbewegung aufweisen. \\

Die Hauptebene des Systems entspricht dem mittleren Erdäquator zur Epoche J2000.0. Diese Ebene wird auch \textbf{Himmelsäquator} genannt. Eine Achse des Systems zeigt dementsprechend zum mittleren Rotationspol $\overline{P}_{N_0}$. Dieser Pol wird \textbf{Conventional Reference Pole} CRP gennant.

\subsubsection{Internation Celestial Reference Frame}

\subsection{Erdfeste Bezugssysteme}

\subsubsection{International Terrestrial Reference System}

\subsubsection{International Terrestrial Reference Frame}

\subsubsection{GNSS-spezifische Referenzsysteme}

\subsection{Erdrotation}

\subsubsection{Präzession}

\subsubsection{Nutation}

\subsubsection{Erdrotation}

\subsubsection{Polbewegung}

\subsubsection{Zusammenfassung Transformation raumfest $\leftrightarrow$ erdfest}

\subsection{Schwerefeldbezogene Bezugssysteme}

\subsubsection{Beschreibung der örtlichen Lotlinie}

\subsubsection{Lokale astronomische Systeme}

\subsection{Zeitsysteme}

\subsubsection{Sonnenzeit und Sternzeit}
 
\subsubsection{Atomzeit}

\section{Geometrie der Erde}

\end{multicols*}
\end{document}

