\documentclass[9pt, landscape, fleqn]{scrartcl}
\setlength{\parindent}{0pt}
\usepackage[ngerman]{babel}
%\usepackage[applemac]{inputenc}
\usepackage[utf8]{inputenc}
\usepackage[dvips]{geometry}
\usepackage{latexsym}
\usepackage{multicol}
\usepackage{amsmath}
\usepackage{graphicx}
\usepackage{array}
\usepackage{booktabs}
\usepackage{amsmath}
\usepackage{mathtools}
\usepackage{ulem}
\usepackage{amsfonts}
\usepackage{dsfont}
\usepackage{charter} %%% Schreibart
%\renewcommand{\familydefault}{\sfdefault}



%%%%%%%%%%Paket für Chemische Formeln
\usepackage{chemformula} 
\usepackage[version=3]{mhchem}
%%%%%%%%%%%%%%%%% Farbe
\usepackage{color}

\pagestyle{plain}
\typearea{45}
\columnsep 30pt
\columnseprule .4pt
\setlength{\extrarowheight}{0.9em}

\renewcommand{\arraystretch}{0.8}

\makeatletter
\renewcommand{\section}{\@startsection{section}{1}{0mm}%
{-2\baselineskip}{0.8\baselineskip}%
{\hrule depth 0.2pt width\columnwidth\hrule depth1.5pt
width0.25\columnwidth\vspace*{1.2em}\Large\bfseries\rmfamily}}
\makeatother


\makeatletter
\renewcommand{\subsection}{\@startsection{subsection}{1}{0mm}%
{-2\baselineskip}{0.8\baselineskip}%
{\hrule depth 0.2pt width\columnwidth\hrule depth0.75pt
width0.25\columnwidth\vspace*{1.2em}\large\bfseries\rmfamily}}
\makeatother

\makeatletter
\renewcommand{\subsubsection}{\@startsection{subsubsection}{1}{0mm}%
{-2\baselineskip}{0.8\baselineskip}%
{\hrule depth 0.2pt width\columnwidth\vspace*{1.2em}\normalsize\bfseries\rmfamily}}
\makeatother

\newcommand{\Mx}[1]{\begin{bmatrix}#1\end{bmatrix}}
\begin{document}
\part*{\LARGE\textrm{Satelitengeodäsie - Zusammenfassung $\hfill$ Xeno Meienberg}}
\begin{multicols*}{3}

\section{Bezugssysteme}

\subsection{Einführung}

\subsubsection{Zeit und Bezugssystem}

\begin{itemize}
    \item Newtonsche Mechanik: absolute Zeit $t$ $\rightarrow$ gleiförmig und unabhängig vom Koord.system
    \item Mathematisch: Die unabhängige Variable im Raum sämtlicher Bewegungen ist die Zeit
    \item Koordinatensystem: Definiert durch einen Urpsprung $O$ + drei orthogonale Einheitsvektoren $e_1$, $e_2$, $e_3$
    \item $\vec{e_1}$,$\vec{e_2}$, $\vec{e_3}$ = Basisvektoren, wobei $\vec{x}$ eine Linearkombination aller drei Komponenten ($x_1$,$x_2$,$x_3$) des Punktes $\vec{x}$ sind 
    \item Die Bezeichnung ``Vektor'' ist eigentlich falsch, da ein Vektor unabhängig von Koordinatensystem ist. Koordinaten sind von einem Bezugssystem abhängig
    \item Post-newtonsche Formalismen sind Bezugs- und Zeitsysteme, Frequenzen, Phasen- und Laufzeitdifferenzen, welche ebenfalls wichtig sind
\end{itemize}

\subsubsection{Ortsvektor, Bahnkurve, Weltlinie}

Die \textbf{Kinematik} eines Massenpunktes wird beschrieben durch den Ortsvektor $\underline{x}(t)$ als Funktion der Zeit: $\underline{x}(t) = (x_1(t),x_2(t),x_3(t))$. \\


Die Gesamtheit aller Endpunkte der Ortsvektoren nennt man \textbf{Bahnkurve}. Die Funktionen $x_1$,$x_2$,$x_3$ sind die Koordinaten des Massenpunktes im gewählten \textbf{Bezugssystems} $\{O,~e_1,~e_2,~e_3\}$ \\

Anstatt Parameterdarstellung $\underline{x}=\underline{x}(t)$ kann auch die Kurve im vierdimensionalen Raum dargestellt werden ($t =$ 4. Achse). Daraus ergibt sich die \textbf{Weltlinie} $\underline{x}(ct,~x_1,~x_2,~x_3)$ \\

Die Skalen für die Erde mit $R=6378~km$ sind wie folgt:

\begin{itemize}
    \item $1^\circ = 110~km$ 
    \item $1'' = 30~m$
    \item $1 mas = 3~cm$
    \item $1 \mu as = 0.03~mm$
\end{itemize}

\subsubsection{Referenzsysteme und Referenzrahmen}

\begin{itemize}
    \item \textbf{Koordinatensystem}: Alle Grössen, die notwendig sind, um Punkte eindeutige Koordinaten zuordnen zu können (Ursprung, Achsen, Massstab)
    \item \textbf{Referenz- und Bezugssystem}: Erweiterung des Koordinatensystems durch z.B. eines Bezugsellipsoids oder Modellschwerefeld oder Gravitationskonstante
    \item \textbf{Datumsdefinition}: Festlegung von DOF (Freiheitsgrade), die nicht aus den Messungen selbst abgeleitet werden können. Hinreichende Menge von ausgewählten Grössen muss vorhanden sein 
    \item \textbf{Koordinatenrahmen}: Wahl des Systems (Koordinatensystem) + Datumsdefinition $\rightarrow$ Koordinatenlisten für Referenzpunkte
    \item \textbf{Referenzrahmen oder Bezugsrahmen}: Realisierung des Koordinatensystems + Datumsdefinition
    \item \textbf{Referenzsystem}: Konzept eines idealen Referenzsystems basierend auf abstrakten Prinzipen (Mathematik)
    \item \textbf{Konventionelles Ref.system}: Mathematik reicht nicht aus, da Faktoren wie Gezeiten, Lovesche Zahlen (Erdrigidität) berücksichtigt werden müssen. Physikalische Modelle sind gegeben.
    \item \textbf{Konventioneller Ref.rahmen}: Konkrete Realisierung durch Punkte wie Stationen, Sterne, Quasare mit Koordinaten
\end{itemize}

Somit ist ein Rahmen eine Realisierung eines Systems mittels Koordinaten, Referenz- oder Bezugssystemen, sowie einer Datumsdefinition. \\

In der Geodäsie gibt es drei Arten von Bezugssystemen:

\begin{itemize}
    \item \textbf{Raumfeste Systeme}: Inertialsystem (gute Annäherung) $\rightarrow$ zälestisch = celestial
    \item \textbf{Erdfeste Systeme}: Fest mit Erde verbunden und rotiert mit dieser (terrestrische Systeme)
    \item \textbf{Lokale Systeme}: An ein Messinstrument gekoppelt oder orientiert sich an einer Referenzfläche (topozentrische Systeme)
\end{itemize}

\subsection{Raumfeste Bezugssysteme}

Inertialsystem, welche keinen Scheinkräften ausgesetzt sind, es gelten allgemeine Kräfte wie: $F = m\cdot a$. Es gibt zwei Ansätze hierzu:

\begin{itemize}
    \item \textbf{Dynamischer Ansatz}: Trajektorien von Planeten udn Ssystem sind Differentialgleichungssysteme. Die Gleichungen gelten nur in einem Inertialsystem
    \item \textbf{Kinematischer Ansatz}: Annahme: Universum rotiert nicht (fixe Punkte). Aus diesem Grund ist es ein Inertialsystem. Galaktische Objekte, wie z.B. Quasare bewegen sich nicht allzu fest
\end{itemize}

Es ist nicht ausschliessbar, dass unser Sonnensystem sich bewegt, und sich linear im Raum bewegt (Expansion des Universums). Man nennt dies in diesem Fall ein \textbf{Quasiinertialsystem}

\subsubsection{Grundbegriffe}
\begin{itemize}
    \item \textbf{Epoche}: Zeitpunkt, auf den sich Koordinaten, Bahnelemente oder Ephemeriden (tabellierte Positionen von Sternen etc. beziehen
    \item \textbf{Äquatorebene:} Ausdehnung des Äquators in den Raum und senkrecht zur Rotationsachse der Erde
    \item \textbf{Ekliptik:} Umlaufebene der Erde um die Sonne. Diese ist relativ zur Äquatorebene verschoben um die Schiefe der Ekliptik
    \item \textbf{Rotationspol der Erde:} $P_N$: Schnitt der Koordinate Z im Raumfesten Bezugssystem (Rotationsachse)
    \item \textbf{Pol der Ekliptik:} $P_\Pi$: Pol , welcher sich mit der Ekliptik verschiebt
    \item \textbf{Schiefe der Ekliptik:} $\epsilon \approx 23.4^{\circ}$
    \item \textbf{Frühlingspunkt}: Ist der Punkt, wo sich Ekliptik und und Äquatorebene schneidet bevor die Ekliptik positiv wird (von Süden nach Norden). Bei der Nordhalbkugel ist dies der 19., 20. oder 21. März
    \item \textbf{Herbstpunkt}: Analog zu Frühlingspunkt, jedoch wird die Ekliptik hier von Norden nach Süden verlaufen (22. oder 23. September)
    \item \textbf{Äquinoktien}: Sind Frühlings- und Herbstpunkt der Erde (analog bei anderen Himmelskörpern) so kommt es zur Tag- und Nachtgleiche. Die Ekliptik und Äquatorebene gleich und die Halbkugel Nord und Süd wird komplett von der Sonne bestrahlt
\end{itemize}

\subsubsection{International Celestial Reference System}

Das \textbf{International Celestial Reference System (ICRS)} ist kinematisch definiert und ist ein raumfestes System. Die Achsenorientierung ist ausgelegt auf sehr weite Objekte im Universum, da diese praktisch kaum eine Eigenbewegung aufweisen. \\

Die Hauptebene des Systems entspricht dem mittleren Erdäquator zur Epoche J2000.0. Diese Ebene wird auch \textbf{Himmelsäquator} genannt. Eine Achse des Systems zeigt dementsprechend zum mittleren Rotationspol $\overline{P}_{N_0}$. Dieser Pol wird \textbf{Conventional Reference Pole} CRP gennant.
Die Referenzrichtung entspricht der Richtung zum Frühlingspunkt zur Epoche J2000.0.\\

Die Festlegung der Epoche ist wichtig, da Äquatorebene, Ekliptik, und damit Frühlingspunkt aufgrund Nutation und Präzession sich langsam verändern.
Die Epoche J2000.0 entspricht dem 1. Januar 2000, 11:58:55.816 (UTC). Je nachdem, wo der Ursprung gelegt wird, spricht man vom Barycentric Celestial Reference System (BCRS, Sonne) oder Geocentric Celestial Reference System (GCRS, Erde). Anwendungen liegen in der Raumfahrt. Das GCRS kommt meistens zur Anwendung bei Satelitenbahnen.
Die Umrechnung berücksichtigt auch relatistische Effekte.

\subsubsection{International Celestial Reference Frame}
Das ICRF benutzt Quasare, BL-Lacertae-Objekte und aktive Galaxienkerne. Die ausgewählten Objekte sind in drei Klassen eingeteilt:
\begin{itemize}
    \item 1. Klasse: Defining Group: Kennt man lange
    \item 2. Klasse: Candidate Group: Bisher nicht lange gekannt oder nicht bisher so lange beobachtet
    \item 3. Klasse: Nicht geeignet, jedoch immer noch interessant
\end{itemize}

Das aktuelle ICRF3 nutzt 303 definierte und 4500 Objekte (Genauigkeit $30~\mu as = 1~mm$, Achsenstabilität $10~\mu as$). Die Objekte werden mittels VLBI ausgemessen. Es gibt auch optische Realisierungen, basierend auf dem Hipparcos Sternkatalog (100'000 Sterne aus zur Epoche 1991.25). Mittels Sonden konnte man jedoch auch bereits optische Realisierungen des ICRS verbessern. Die Rekaszension und Deklination Genauigkeit liegt hier bei 0.77 mas bzw. 0.64 mas. Der Nachfolger wurde im Jahr 2013 mit GAIA gemacht, welche eine Genauigkeit von $10 \mu as $ aufweist und 40-fach Genauer ist.\\

Die dynamische Realisierung des ICRS ist durch die Ephemeriden von Planeten und des Mondes gegeben. Diese werden durch die Einstein-Infeld-Hoffmann-Gleichungen erzeugt (DGL-System), durch Punktmassen unter gegenseitiger Gravitationsanziehung mit berücksichtigung allgmein-relativistischen Effekten.

\subsection{Erdfeste Bezugssysteme}
Fest mit der Erde verbunden. Die Positionen sind fest mit der Erdoberfläche verbunden nur zu kleinen zeitlichen Variationen. Durch Tektonik oder Gezeiten verschieben diese sich wenig. Zudem definierte man ein Bezugsellipsoid, welches sich gut an den Erdkörper anschmiegt. Zudem kann ein Bezugsschwerefeld festgelegt werden.

\subsubsection{International Terrestrial Reference System}

Das IERS definierte ITRS ist ein Koordinatensystem mit gleich skalierten Koordinatenachsen. Massenzentrum ist die Erde (inkl. Ozeane und Atmosphäre). Die Z-Achse entspricht der mittleren Rotationsachse der Erde, 
die X-Achse zeigt zum Greenwich Meridian. Die Summe aller Im Netz liegenden Stationen aufgrund Plattenbewegungen ergibt null (no-net-rotation), dies ist eine Bedingung des Systems. Regularisierte Stationskoordinaten führen dazu, dass Stationskoordinaten des ITRS sich auf eine feste Erdkruste beziehen.

\subsubsection{International Terrestrial Reference Frame}

Basierend auf VLBI, SLR, GPS und DORIS. Die Berechnung der Koordinaten findet zweistufig statt:

\begin{enumerate}
    \item Einzelne Zeitserien zu technik-spezifischen Langzeiglösungen: Koordinaten, Geschwindigkeiten und Rotationsparameter werden berechnet
    \item Vier Einzellösungen werden zur endgültigen ITRF Lösung kombiniert.
\end{enumerate}

Das ITRF2014 bestimmt den Ursprung (Geozentrum) mittels SLR, Der Massstab wird definiert durch SLR und VLBI; und die Orientierung basierend auf dem alten System realisiert. \\
Die Stationen erhalten Koordinaten, inklusive eine Geschwindigkeit: $\mathbf{x} = \mathbf{x_0} + \mathbf{x}(t-t_0)$. Die Koordinaten sind regularisiert durch konventionelle Korrekturen, um hochfrequente zeitliche Variationen zu eliminieren (Gezeiten und Ozeanauflasten). Der Zusatzterm ist $\sum \Delta \mathbf{x_i}(t)$ \\
Die Koordinaten neuer Punkte im ITRF werden folgendermassen ausgewertet:

\begin{itemize}
    \item Direkte Benutzung einiger ITRF-Stationen als Referenzpunkte
    \item Einführung von IGS Orbit- und Uhrinformationen in eine GNSS-Auswertung (Precise Point Positioning -PPP)
    \item Einführung und fixieren von einigen ITRF-Referenzstationen in eine GNSS-Auswertung
    \item Ähnlichkeitstransformationen der gemesssenen Neupunkte in das ITRF mitttels Referenzstationen
\end{itemize}

Neben kartensischen Koordinaten können auch geographische ellipsoidische Koordinaten miteinbezogen werden. Die Parameter des GRS80-Ellipsoids sind da empfohlen. \\ 

Schwerefelder können genutzt werden um Satelitenbahnen zu modellieren. Das EGM2008 Modell benötigt GRACE (Satelit) Daten, Schweremessungen am Boden und Altimetriedaten

\subsubsection{GNSS-spezifische Referenzsysteme}

GNSS-spezifische Systeme werden oft von broadcast-Orbits genutzt. GPS, GLONASS, GTRF nutzen dieses und stellen diese Daten zu Verfügung. 

\begin{itemize}
    \item GPS nutzt 12 Fundamentalstationen, es werden in der Realität mehr genutzt. Diese Stationen werden regelmässig vermessen und an das ITRF angeschlossen. Das WGS84 ist auf das ITRF2008 bis zu 10cm konsistent
    \item GLONASS: Nutzt das russiche System PZ-90, und beruht auf 26 Bodenstationen. Zwischen PZ-90.11 und ITRF2008 gibt es nur eine Translation
    \item GTRF (GALILEO Terrestrial Reference Frame): nutzt Daten von Referenzstationen mit ausgewählten GPS-Stationen des IGS-Netzes und transformiert in das ITRF. Hierzu gibt es auch Transformationsparameter.
\end{itemize}

\subsection{Erdrotation}
Die Transformation vom raumfesten System in das konventionelle erdfeste System ist im Prinzip eine räumliche Rotation, die durch drei Rotationswinkel vollständig beschrieben ist. Diese wird in mehrere Einzelschritte aufgetrennt.
\subsubsection{Präzession - $\overline{S}_{i_0} \rightarrow \overline{S}_{i}$}

Im ersten Schritt wird die Präzessionsbewegung der Erde berücksichtigt. Diese wird hauptsächlich durch die Gezeitenkräfte von Mond und Sonne verursacht. Aufgrund der Form der Erde, angenähert durch eines abgeplatteten Ellispoids, wird auf die Äquatorwulst ein Drehmoment aufweist. Die Erdachse versucht sich aufzurichten, womit die Präzession entsteht. Zur Referenzepoche J2000.0 wird das mittlere Inertialsystem zum \textbf{mittleren Inertialsystem zur Beobachtungsepoche} umgewandelt. Es handelt sich hierbei um die Drehung um drei Winkel um die Achsen 3-2-3. Die Präzessionsperiode ist 28400 Jahre. Die Magnitude der Präzession liegt bei $50''$ pro Jahr (1.5km)

\begin{equation*}
    \mathbf{\overline{e}}_i = \mathbf{P} \mathbf{\overline{e}}_{i_0} = \mathbf{R_3}(-z_A)\mathbf{R_2}(\theta_A) \mathbf{R_3}(-\zeta_A) \mathbf{\overline{e}}_{i_0}
\end{equation*}

Die Winkel werden angenähert durch ein Zeitargument $t$, wobei dieses folgendermassen definiert ist:
\begin{equation*}
    t = (t_{obs}-J2000.0) / 36525
\end{equation*}

Der Zeitunterschied in Tagen zwischen der Zeit der Observation zum J2000.0 wird normiert und so in Jahrhunderten gerechnet.

\subsubsection{Nutation - $\overline{S}_{i} \rightarrow S_i$}

Nach der Präzession werden kurzperiodeische Nutationsperioden berücksichtigt. Die mit der Präzession eingehende Nickung ist die Nutation, welche mit der Bewegung des Mondes einhergeht (T=18.6 Jahre). Somit wandelt man das System vom mittleren intertialsystem der Beobachtungsepoche zum \textbf{Wahren Inertialsystem der Beobachtungsepoche}.
Somit wird mittels der mittleren Schiefe der Ekliptik (Rotation um Ekliptik), der Nutation in der Länge (Rotation um Erdachse) der Nutation in der Schiefe (Rotation um Ekliptik) die Transformation durchgeführt. Die Nutationswinkel sind schwieriger zu berechnen (viel Trigonometrie). Vom Konzept her wird die mittlere Äquatorebene zuerst in die Ekliptik verschoben, danach durch zwei Rotationen korrigiert in die wahre Äquatorebene. Die Magnitude der Nutation liegt bei $9.2''$ (300met)
\begin{equation*}
    \mathbf{e}_i = \mathbf{N} \mathbf{\overline{e}}_{i} = \mathbf{R_1}(-\epsilon_A-\Delta \epsilon)\mathbf{R_3}(-\Delta \psi) \mathbf{R_1}(\epsilon_A) \mathbf{\overline{e}}_{i_0}
\end{equation*}
\subsubsection{Erdrotation - $S_i \rightarrow S_e$}
Durch das Anbringen der Präzession und Nutation wird die dritte Achse angepasst, damit die Rotationsachse der Erde entspricht und somit das \textbf{wahre erdfeste System} erhalten wird.
\begin{equation*}
    \mathbf{e}_e = \mathbf{R}_3(\Theta_0) \mathbf{e}_{i} 
\end{equation*}

\begin{align*}
    \Theta_0 = GAST &= GMST + \Delta \psi \cos \epsilon_A + 0.00264'' \sin \Omega + \\ 
    &+ 0.000063'' sin(2\Omega)
\end{align*}

Der Rotationswinkel $\Theta_0$ ist GAST (Greenwich apparent siderial time), der wahren Sternzeit von Greenwich. GMST (mittlere Sternzeit) berücksichtigt die Nutation nicht. Der Winkel ist der wahre Frühlingspunkt relativ zum Greenwichmeridians. Durch die schwankende Unregelmässigkeit der Tageslänge muss man die Differenz $UT1-UTC$ mitberücksichtigt werden. Die Nutation hat ebenfalls einen Einfluss auf GAST, also muss man diese mitberücksichten. UT1 beschreibt die Rotation der Erde, da es sich hier um die exakte Zeit handelt. Der Sterntag ist 23h56min, die Sonnenzeit exakt 24h. Somit muss diese Korrektur mitberücksichtigt werden.

\subsubsection{Polbewegung - $S_e \rightarrow S'_e$}

Schliesslich muss noch die Polbewegung der Erde (Bewegung der Rotationsachse um Figurenachse) im erdfesten System berücksichtigen werden. 
\begin{equation*}
    \mathbf{e}'_e = \mathbf{W} \mathbf{e_e} = \mathbf{R}_2(-x_p) \mathbf{R}_1(-y_p) \mathbf{e}_e
\end{equation*}

Die Polbewegung ist allerdings unregelmässig. Sie ist nicht periodisch, weshalb man auf tabellierte Werte zurückgreift. UT1-UTC, die Nutationswinkel werden jeweils immer vom IERS gestellt. Jedoch lässt sich eine allgemeine Schwebung durch die Überlagerung zweier Frequenzen: Der Jahreslänge (360 Tage) und der Chandlerperiode (430 Tage). Die maximale Amplitude ist hierbei 0.3'' (10m)

\subsubsection{Zusammenfassung Transformation raumfest $\leftrightarrow$ erdfest}

\begin{equation*}
    \mathbf{e}'_e = \mathbf{WRNP} \overline{\mathbf{e}}_{i_0}
\end{equation*}

Da es sich um Rotationsmatrizen handelt, welche orthogonal sind, kann die Rücktransformation mittels der Inversen (transponierte Matrizen) gemacht werden. Seit 2003 gibt es neuere Systeme, wo sogenannte intermediäre Pole, zälistische und terrestiale Referenzsysteme definiert worden sind. Diese Methode ist noch nicht sehr weit verbreitet.

\subsection{Schwerefeldbezogene Bezugssysteme}

Geodätische oder astronomische Beobachtungen an oder nahe der Erdofberfläche orientieren sich an der örtlichen Lotlinie. Diese Beobachtungen werden vorzugsweise in lokalen, auf das Schwerefeld bezogenen Referenzsystemen modelliert. Die Orientierung der lokalen Systeme ist durch die Astronomische Breite und Länge gegeben. Die Orientierungsparameter erlauben eine Hin- und Rücktransformation zwischen lokalen und globalen Bezugssystemen.

\subsubsection{Beschreibung der örtlichen Lotlinie}

Die Richtung des Sschwerevektors wird durch die angabe der beiden Winkel der Breite und Länge gegeben ($\Phi$ und $\Lambda$)

\begin{equation*}
    \mathbf{g} = -|\mathbf{g}|\cdot \mathbf{n} = -|\mathbf{g}| \cdot \begin{pmatrix}
        \cos \Phi \cos \Lambda \\ \cos \Phi \sin \Lambda \\ \sin \Phi
    \end{pmatrix}
\end{equation*}

\subsubsection{Lokale astronomische Systeme}

Der Usrpung lokaler astronomischer System liegt im Beobachtungspunkt $P$. Die Z-Achse ist gegeben durch die Lotrichtung und zeigt zum Zenit. Die x-Achse zeigt in Nord-richtung. Die y-Achse komplettiert ein Linkssystem. Der Zenitwinkel wird vom Zenit ausgemessen. Der Azimut im Uhrzeigersinn von der x-Achse aus.

\begin{equation*}
    \mathbf{x} = \begin{pmatrix}
        x \\ y \\ z
    \end{pmatrix} = s \begin{pmatrix}
        \cos A \sin z \\ \sin A \sin z \\ \cos z
    \end{pmatrix}
\end{equation*}

Die Umrechnung vom lokalen System in ein globales System kann auf zwei verschiedene Arten ablaufen:

\begin{enumerate}
    \item Geometrisch: \begin{itemize}
        \item Richtungen der Koordinatenachsen des lokalen Systems werden abgebildet in den Raum des globalen Systems. Mittels Skalarprodukts werden die Koordinaten in $X$ auf $x$ projiziert. Die Matrix, da orthonormal, kann invertiert werden durch die Transponation, somit ist $M^T$ die Abbildung von Grössen in $x$ nach $X$
    \end{itemize}
    \item Rotationen und Spiegelungen: \begin{itemize}
        \item Durch die Spiegelung der y-Achse (Rechtssystems), druch Drehungen um die y-Achse um den Winkel $90-\Phi$ (Breite) und Drehung um den Winkel $180-\Lambda$ (Länge), ergibt sich durch Matrixmultiplikation $M^T$ 
    \end{itemize}
\end{enumerate}

\subsection{Zeitsysteme}

\subsubsection{Sonnenzeit und Sternzeit}
Die Rotation der Erde stellt ein natürliches Zeitmass da. Hierbei werden zwei periodische Bewegungen in Betracht gezogen:
\begin{itemize}
    \item Die tägliche Rotation der Erde um die polare Achse
    \item Der jährliche Umlauf der Erde um die Sonne
\end{itemize}

Der Sonnentag wird von aufeinanderfolgenden Höchstständen der Sonna an einem Beobachtungspunkt der Erde abgeleitet. Diese ist jedoch nicht konstant, da:
\begin{itemize}
    \item Die Geschwindigkeit im Umlauf um die Sonne sich variiert (2. Kelper Gesetz)
    \item Die Bahnebene nicht rechtwinklig zur Rotationsachse ist
\end{itemize}

Die Einführung einer mittleren Sonnenbahn (Universal Time UT) führt zu einem mittleren Sonnentag. Hierfür gibt es folgende Gleichung, welche die Rektaszension der wahren Sonne gegenüber der Rektaszension der mittleren Sonne beschreibt:
\begin{equation*}
    EQ_T = -2 e \sin M - 5/4 e^2 \sin 2M + \tan^2 \frac{\epsilon}{2} \sin(2\lambda_EK) - 1/2 \tan^4 \frac{\epsilon}{2} \sin 4\lambda_EK
\end{equation*}

Die Exzentrizität der Bahnellipse, die Schiefe der Ekliptik, die Mittlere Anomalie der Sonne, sowie die Länge der Sonne in der Ekliptik gemessen vom Frühlingspunkt werden hierzu benötigt. Die Abweichungen der wahren zur mittleren Sonnenzeit beträgt zwischen -14 und 16 Minuten. UT1 wurde eingeführt, um die aktuelle Erdrotation, die mittlere Sonnenbahn und den mittleren Pol zu beschreiben um somit den wahren Winkel der Rotation des globalen terrestrischen Koordinatensystems darzustellen. \\

Die Sternzeit bezieht sich auf den Meridiandurgang des Frühlingspunktes. Im Vergleich zur Sonnenzeit, wo 50 Stationen die Höchststände der Sonne bestimmen, wird hier vom der Winkelunterschied vom Greenwich Meridian zum mittleren, beziehungsweise wahren Frühlingspunkt gemessen. Lokale Sternzeiten sind ebenfalls definiert, womit die Relation zum Greenwich Apparent Siderial Time (GAST) und Greenwich Median Siderial Time berechnet wird:
\begin{equation*}
    \Lambda = LAST-GAST = LMST - GMST
\end{equation*}
 
\subsubsection{Atomzeit}
Alteranativ zu den auf der Erdrotation basierenden Zeitskalen, werden die Zeiten durch Atomuhren realisiert. Die internationale Atomzeit ATI basiert auf der SI-Sekunde auf dem rotierenden Geoid. Die Frequenzsstabilitäten über Jahre sind bis zu $10^{-13}$. $TT$ ist die dynamische Terrestrische Zeit und wird durch für die Integration von Satelitenbahne verwendet. $TAI$ wurde mittels 250 Atomuhren auf der Erde bestimmt und der Anfangspunkt wurde auf UT1 im Jahre 1958 festgelegt. Da jedoch UT1 und UTC mit der Zeit sich ändern aufgrund der Rotation der Erde.

\section{Geometrie der Erde}
Bereits früh wurde erkannt, dass die Erde eine Kugel ist. Durch die Annahmen der Griechen hatte man über die Zeit den Erdradius immer besser annähern können. Zudem wurde beobachtet, dass die Erde keine Kugel, sondern eher ein Rotationsellipsoid ist. Einerseits durch die Abplattung anderer Planeten,
andererseits durch die veränderungen der Schwerkraft mit der geographischen Breite. 

\begin{equation*}
    \frac{X^2}{a^2}+\frac{Y^2}{a^2} + \frac{Z^2}{b^2} = \frac{W^2}{a^2} + \frac{Z^2}{b^2}= 1
\end{equation*}

Durch die Vereinfachung von $W = \sqrt{X^2+Y^2}$, also dem kürzesten Abstand zwischen Z-Achse und Punkt P auf dem Rotationsellipsoiden können neue Gleichungen hergeleitet werden:
\begin{align*}
    f &= \frac{a-b}{a},~ \textbf{Abplattung}~f \\
    \varepsilon &= \sqrt{a^2-b^2},~\textbf{Lineare Exzentrizität}~\varepsilon \\
    e &= \varepsilon/a \Longleftrightarrow b=a \sqrt{1-e^2},~\textbf{Numerische Exzentrizität}~e \\
    \frac{b}{a}&=1-f = \sqrt{1-e^2}
\end{align*}
Die Erde hat ungefähr eine Abplattung von $f=300$, also beträgt der Unterschied zwischen $a$ und $b$ ungefähr $22~km$. Die Lineare Exzentrizität misst den Abstand der Brennpunkte zum Ursprung. Bei der Erde ist dies ungefähr $522~km$. 

\subsection{Geozentrische kartesische und krummlinige Koordinaten}
\subsubsection{Einleitung und Umrechnungsmethoden}
In diesem Abschnitt wird die Umrechnungsmethode von kartesischen in krummlinige Koordinaten (Breite, Länge) angeschaut. Beim Rotationsellipsoid können drei verschiedene Arten von Breiten berechnet werden, die Länge $\lambda$ ist gleich bei allen:
\begin{itemize}
    \item Geozentrische Breite $\beta$: Hier wird die Ellipsenbahn verwendet, der Abstand ist abhängig von der Breite $\beta$.
    \item Reduzierte Breite $\beta'$: Hier wird ein Referenzkreis mit Radius $a$ genommen, wobei ein Referenzpunkt P runterprojiziert werden kann auf Punkt Q, von dem die Koordinaten bestimmt werden können.
    \item Geodätische Breite $\varphi$: Hier wird die Tangentensteigung der Ellipse bestimmt. Mittels Trigonometrie, Ableitung der Ellipsengleichung von $Z$ nach $W$ nutzt man den Steigungswinkel. Es ergibt sich eine neue Grösse $R_N$, den Querkrümmungsradius im Punkt (ein Kreis, welcher an die Ellipse lokal anschmiegt und auf der Z-Achse liegt).
\end{itemize}

Die verschiedenen Winkel stehen im Verhältnis:
\begin{equation*}
    \tan \varphi = \frac{Z}{W} \frac{b^2}{a^2} \geq \tan \beta' = \frac{Z}{W} \frac{b}{a} \geq \tan \beta = \frac{Z}{W}
\end{equation*}

\subsubsection{Topographie}
Bisher wurden Helmert-Projektionen verwendet. Hier wird der Punkt P senkrecht auf die Ellipsenoberfläche projiziert. In der Pizzetti-Projektion wird Punkt $P$ entlang der Lotlinie auf die Ellipsenoberfläche projiziert. Somit ergibt sich eine Parameterdarstellung des Punktes auf die Oberfläche.
\begin{align*}
    &\mathbf{n}_Q = \begin{pmatrix}
        \cos \varphi \cos \lambda \\
        \cos \varphi \sin \lambda \\
        \sin \varphi
    \end{pmatrix} \\
    &\mathbf{r}_Q = R_N \begin{pmatrix}
        \cos \varphi \cos \lambda \\
        \cos \varphi \sin \lambda \\
        (1-e^2)\sin \varphi
    \end{pmatrix} \\
    & \mathbf{r}_Q = \mathbf{r}_Q + h_P \cdot \mathbf{n}_Q
\end{align*}

Da der Querkrümmungsradius $R_N$ die geodätische Breite $\varphi$ beinhaltet, muss diese Gleichung für $\varphi$ iterativ gelöst werden. Es gibt jedoch Annäherungsmethoden.

\section{Geodätische Raumverfahren}

\subsection{Very Long Baseline Interometry}
Mit grossen Radioteleksopen (bis $D=100~m$) werden bei VLBI extragalaktische Radioquellen, sog. Quasare beobachtet. Diese Objekte sind so weit entfernt, dass sie keine feststellbaren Bewegungen aufweisen. Sie sind daher ideal geeignet, um ein raumfestes Referenzsystem zu realisieren. Das Netz von 2 bis 6 VLBI Stationen registrieren innerhalb von Sessionen à 24 Stunden. Von der gleichen Radioquelle werden zwei Frequenzen registriert. Auf extrem genauen Zeitmarken auf Festplatten werden diese registriert und im Bereich von $10^{-14}$. Durch Kreuzkorrelation werden die Signale zur Interferenz gebracht und die Unterschiede in der Ankunftszeit der Signale durch den Time Delay und die zeitlichen Differenzen der Laufzeitunterschiede durch Maximierung der Korrelation ermittelt. 
\begin{equation*}
    \Delta \tau = -\frac{1}{c}\mathbf{b}\cdot \mathbf{e}_s
\end{equation*}
Durch Messungen vieler Radioquellen ergeben sich die Basisvektoren $\mathbf{b}$ zwischen Stationen und die Richtungen zu den Quellen $\mathbf{e_s}$ bestimmen und die troposphärische Refraktion, sowie instrumentelle Fehler mitbestimmen und reduzieren. Die Genauigkeit liegt hier im Millimeterbereich für Koordinaten bzw. Millimeter/Jahr in der Geschwindigkeit. Die ionosphärische Refraktion wird mittels zwei Frequenzen eliminiert. (2.3 und 8.4 GHz). Die Genauigkeit der Erdrotationsparameter beträgt etwa 0.2 Millibogensekunden bei Polschwankung und 10 $\mu s$ in der Erdrehung (UT1). VLBI ist die einzige Methode, um die Verbindung zum raumfesten Referenzsystem zu realisieren und die Positionen der Radioquellen relativ zueinander zu bestimmen.

\subsection{Satelite and Lunar Laser Ranging}
Laser Ranging zu künstlichen Erdsateliten oder Mond ist vom Konzept her sehr einfach. Sehr kurze Laserpulse werden zu einem Sateliten geschickt, der mit Retroreflektoren augestattes ist und zur Station zurückschickt. Aus der Messung der Lichtlaufzeit hin und zurück ergibt sich die Distanz zw. Station und Satelit. Neben der Distanz muss in der Beobachtungsleichung nebst der geometrischen Distanz auch Korrekturterme mitberücksichtigt werden.
\begin{align*}
    \Delta t_E^S &= \frac{2}{c}\left( \underbrace{\rho_E^s}_{Distanz} + \underbrace{\delta \rho_{atm}}_{Refraktion} + \underbrace{\delta_{rel}}_{Relativitaet} \right)  \\
    &+ \underbrace{\frac{1}{c} \delta \rho_{sys}}_{Signalverz.}+ \underbrace{\epsilon_E^S}_{Messfehler}
\end{align*}
Eigentlich müsste die rein geometrische Laufzeit des Pulses aus zwei Einzellaufzeiten berechnet werden (hin und zurück). Für einen Sateliten in der Höhe von 20'000 km (GPS, oder GLONASS) beträgt die Lichtlaufzeit rund 0.13 Sekunden. Der quadratische Term aus einer Reihenentwicklung liegt im Fehler bei etwa $10^{-4}m$ Bereich. Für Messungen zum Mond muss die Gleichung erweitert werden. Der Sender und Empfänger müssen im gleichen Koordinatensystem vorliegen (Transformation von Erdfest zu Raumfestes System). \\

Zur Zeit gibt es etwa 40 Sateliten, welche mit SLR verfolgt werden können. Die wichtigsten sind LAGEOS I und II. Diese sind etwa 60 cm gross und etwa bei 6000 km Höhe. Die Distanzmessungen sind etwa 1cm genau. Die Parameter von Satelitenbahnen, Koeffizienten des Erdschwerefeldes, Staionsgeschwindigkeiten und Erdrotationsparameter werden mittels SLR bestimmt. SLR hilft am genauesten, um den Schwerpunkt der Erde zu bestimmen. \\

LLR ist 2cm genau und benötigt starke Laser, um diese wirklich zu messen. Erdrotationsparameter können mittels LLR bestimmt werden (UT1 z.B.), sowie Parameter für die Mondbahn und für Tests in der Relativitätstheorie.

\subsection{Doppler Orbitography by Radiopositioning Integrated on Satelite (DORIS)}
Bei DORIS handelt es sich um ein Zweifrequenz-Dopplersystem (2.03 GHz und 0.4 GHz). Das vor allem für die Bestimmung von Satelitenbahnen eingesetzt wird. Bei DORIS ist nebst Sendesystem auch ein Empfänger integriert, dessen Bahn man mitverfolgen kann. DORIS kann also einem Sateliten zusätzlich mitgegeben werden. \\

Die Doppler-Beobachtungsdaten werden zentral im Sateliten gesammelt und an das Betriebszentrum übermittelt. DORIS kann auch an entlegene Gebiete aufgestellt werden wo die Übermittlung von Daten z.B. für GPS, ein Problem darstellen würde. \\

Die 50 Stationen des DORIS Systems sind homogen über die Erde verteilt und bilden ein gutes Referenzsystem für die Untersuchung von Plattenbewegungen. Die Genauigkeit liegt hier für Stationskoordinaten 2cm, und 1-2 Millibogensekunden für die Stellung der Erdachse im erdfesten System. Zur Zeit sind 6 Sateliten mit DORIS ausgestattet.

\subsection{Satelitenaltimetrie}
Das Prinzip basiert auf der Laufzeitmessung eines Mikrowellenpulses im Radarfrequenzbereich (14 GHz) vom Sateliten zur Erdoberfläche und zurück. Die Genauigkeit liegt hier bei einigen Zentimetern bis zu einem Meter. 
\begin{equation*}
    \rho = c \frac{\Delta t}{2}
\end{equation*}
Die Pulslänge beträgt wenige Nanosekunden. Besonders gute Messungen liefert dieses Verfahren wegen den guten Reflexionseigenschaften über Wasseroberflächen. Der Durchmesser des Strahles am Boden wird folgendermassen berechnet:
\begin{equation*}
    D = 2 \sqrt{2 \rho c \tau + (c\tau)^2}
\end{equation*}
Hier ist $c\tau$ die Pulslänge. Die Fläche die hier abgedeckt wird sind oft einige Kilometer. Somit lässt sich die Distanz zwischen Satelit und Meeresoberläche bestimmen. Nimmt man an, man kenne die Höhe der Meeresoberfläche, wie die die lokale Höhe des Geoids sowie die Abweichung $H$ der Meeresoberfläche vom Geoid, lassen sich für verschiedene Disziplinen die Ozeangebiete in kurzer Zeit flächenhaft vermessen.
\section{Das Schwerefeld der Erde}
Hier werden die Grundlagen der Physikalischen Geodäsie erarbeitet, insbesondere die Grundlagen, welche mit dem Schwerefeld der Erde sich befassen. Hierbei liefert die Potentialtheorie die Grundlagen der Physikalischen Geodäsie. Die meisten geodätischen Messungen beziehen sich auf das Schwerefeld und deren Richtungen (Lotrichtung, lokale Niveaufläche), d.h. viele Messinstrumente werden beispielsweise horizontiert. Andere Messungen wie Nivellelement, Schweremessungen, Trägheitsnavigation und Satelitenbahnen hängen ebenfalls vom Schwerefeld ab.
\subsection{Newtonsches Gravitationsgesetz}
\begin{equation*}
    F = G \frac{m_A m_B}{l^2_{AB}}
\end{equation*}
Die Gravitationskraft wird bestimmt durch eine Gravitationskonstante $G$, den zwei punktförmigen, homogenen Massen $A$ und $B$ sowie deren Abstand. Die Graviationskonstante für sich alleine kennt man nur auf 6 Stellen genau. Das Produkt von Gravitationskonstante und Erdmasse kennt man jedoch auf ca. 10 Stellen sehr genau, da man viele künstliche Sateliten hat. \\

Die erste Bestimmung von $G$ wurde mittles einer Torsionswaage bestimmt. Um die Richtung der Gravitationskraft zu bestimmen, wird das Gesetz oft in vektorieller Form bestimmt.
\begin{equation*}
    \mathbf{F}_A = -G \frac{m_A m_B}{l^3_{AB}}\cdot \mathbf{l}_{AB} = -G -G \frac{m_A m_B}{|\mathbf{x}_A - \mathbf{x}_B|^3}\cdot (\mathbf{x_A} - \mathbf{x_B})
\end{equation*}
Dabei ist $F_A$ die Gravitationskraft, die von Masse $B$ auf $A$ ausgeübt wird. Die Massen sind positiv. Im Vergleich zur elektrischen Coulomb Kraft ist diese sehr ähnlich, wobei jedoch die Ladungen positiv und negativ sein können. \\

Nach dem 3. Newtonschen Gesetz gilt auch , dass die entgegengesetzte Kraft auf $B$ exakt der Kraft von $A$ entspricht. Nach dem 2. Newtonschen Gesetz gilt auch $F_A = m_A \mathbf{\ddot{x}}_A$. Hierbei ist jedoch die Masse als Trägheitsmasse zu verstehen, und im Gegensatz zur schweren Masse $m_A$ einen unteschieldichen Charakter aufweist. Newton postulierte, dass beide Massen identisch sind, aus diesem Grund ergibt sich folgende Gleichung für die Gravitationsbeschleuningung:
\begin{equation*}
    \mathbf{a}_A = \ddot{\mathbf{x}}_A = -G \frac{m_B}{|\mathbf{x}_A - \mathbf{x}_B|^3}(\mathbf{x}_A - \mathbf{x}_B) = - G \frac{m_B}{l^3_{AB}}\mathbf{l}_{AB}
\end{equation*}
Bei der Gravitationsbeschleuningung handelt es sich um ein Vektorfeld. Somit gilt auch das Überlagerungs, bzw. Superpositionsprinzips - wo mehrere Massen im Raum dieses verändern und auf einen jeweiligen Bezugspunkt $i$ eine Kraft ausübt. \\

Sobald man von der Punktmasse in geschlossene, kontinuierliche Volumen übergeht, dann ergibt sich die Gravitationsbeschleuningung als eine Riemannsche Summe von Massen, beziehungsweise Volumen mittels der Dichte.

\begin{equation*}
    \mathbf{a}_A = -G \int \int \int_M \frac{\mathbf{l}_{AQ}}{l^3_{AQ}}dm = -G \int \int \int_\Sigma \rho(Q) \frac{\mathbf{l}_{AQ}}{l^3_{AQ}} d\Sigma
\end{equation*}

Die Gravitationskraft veränderts sich instantan. Die Relativitätstheorie widerspricht dem, jedoch dies genügt unseren Anforderungen.
\subsection{Gravitationspotential}
Aufgrund der Annahme, dass das Vektorfeld instantan ist und zeitunabhängig ist, ist das Feld stationär. Zudem kann man das definierte Vektorfeld auch als skalare Funktion beschreiben:
\begin{equation*}
    \mathbf{a} = \nabla V(x,y,z)
\end{equation*}
Das Gravitationspotential V hat einen Gradienten, welches der Gravitationsbeschleuningung entspricht. Das Gravitationspotential ist nicht eindeutig, sondern bis auf eine Konstante festgelegt. Zudem muss jenes auch im Unendlichen Null sein, gegeben durch den Abstand, welcher in der Unendlichkeit den Term 0 macht. Falls das Koordinatensystem so gewählt
ist, dass die Masse B als Quelle des Potentials im Urpsrung liegt, dann kann folgende Gleichung verwendet werden:
\begin{align*}
    & \mathbf{a} = \ddot{\mathbf{r}} = -\frac{G m}{r^3} \mathbf{r} \\
    & V(\mathbf{r}) = \frac{Gm}{r}
\end{align*}
Das Gravitationspotential ist also so definiert.
\subsection{Gravitationspotential einer homogenen Kugel}
\subsubsection{Einführung}
Nun werden Gravitationspotential homogener Kugeln bestimmt. Dies bedeutet, dass diese Kugeln überall die gleiche Dichte $\rho(Q) = \rho = const.$ aufweisen in einem beliebigen Punkt innerhalb der Kugel.
\subsubsection{Potential einer Kugel für einen Punk ausserhalb der Kugel}
Das Potential, welches bei einer Kugelschale $S$ der Dicke $dR$, auf eine Punkt $A$ ausserhalb der Kugel wirkt, wird nun berechnet. Wie bei einer Vollkugel lässt sich durch Transformationnen und Trigonometrie bestimmen, dass die Potentiale ausserhalb von Schalen und Kugeln mit der Distanz $d$ umgekehrt proportional sind zum Abstand, und dass die Massen jeweils einer Schale oder Kugel die Magnitude mitbestimmt.  
\begin{align*}
    &V_{S,aussen} = \frac{G M_S}{d} \\
    &V_{K,aussen} = \frac{G M_K}{d}
\end{align*}
\subsubsection{Potential einer Kugel für einen Punkt innerhalb der Kugel}
Für einen Punkt innerhalb einer Kugelschale erhält man ein konstantes Potential (unabhängig von $d$). Da der Gradient bei einer Konstanten verschwindet, übt die Kugelschale auf einen Massenpunkt im Innern keine Kraft aus.
\begin{equation*}
    V_{S,innen} = 4 \pi G \rho R dR = const.
\end{equation*}
Für einen Punkt innerhalb einer Vollkugel kann folgendes Modell angenommen werden: Es gibt einer äussere Schale, welche aufgrund der Integrationsgrenzen auch einen Effekt auf das Potential hat.
\begin{equation*}
    V_{K,innen} = 2 \pi G \rho \left( R_K^2-\frac{d^2}{3} \right)
\end{equation*}
Somit kann via dem Gradienten folgendes gezeigt werden:
\begin{align*}
    &\mathbf{a} = \nabla V_{K, aussen}(r) = -\frac{GM_K}{r^3} \mathbf{r} \\
    &\mathbf{a} = \nabla V_{K,innen}(r) = -\frac{4}{3}\pi G \rho \mathbf{r}
\end{align*}
Die Gravitationsbeschleunigung nimmt mit $r^2$ ausserhalb der Kugel ab, und nimmt innerhalb mit $r$ ab.
\subsection{Laplacegleichung und Poissongleichung}
Durch die weitere Anwendung des Nabla Operators, erhält man nun beim Potential neue Gleichungen, die erfüllt werden müssen:
\begin{align*}
    &\Delta V_{K,aussen}(r) = 0 \\
    &\Delta V_{K,innen}(r)= - 4\pi G \rho
\end{align*}
Für eine beliebige Dichteverteilung müssen die gegebenen Gleichungen erfüllt sein.
\subsection{Zentrifugalbeschleunigung}
In einem erdfesten Referenzsystem wirken auf einem ruhenden Körper nebst der Anziehungskraft auch die Flieh- oder Zentrifugalkraft, welche durch die Erdrotation verursacht wird. Die Corioliskraft und Eulerkraft kommen ebenfalls hinzu, falls sich die Rotationsgeschindigkeit oder die Richtung der Rotationsachse sich ändert. Diese Kräfte sind Scheinkräfte, welche nur auftreten weil die Erde nicht ein Inertialsystem ist.
\begin{align*}
    &\mathbf{F}_Z = m \mathbf{\omega} \times (\mathbf{r} \times \mathbf{\omega}) = m \omega^2 \mathbf{p} \\
    & p = |\mathbf{p}| = r \cos \beta \\
    &\mathbf{a}_Z = \frac{\mathbf{F}_Z}{m} = \omega^2 \mathbf{p} \\
    &a_z = \omega^2 r \cos \beta, ~a_{z,max} \approx 0.03391 m/s^2
\end{align*}
Die Zentrifugalkraft auf einen Massenpunkt $m$ wirkt in Richtung $p$, welcher senkrecht zur Rotationsachse liegt. Für das Potenzial gilt, dass dieses konstant bei $\Delta V_Z = 2 \omega^2$ liegt.
\subsection{Schwerkraft, Schwerebeschleunigung und Schwerepotential}
Die Schwerkraft ist also eine Vektorsumme aus Gravitation und Zentrifugalkraft. Auf einen Massepunkt hin lassen sich je die Schwerebeschleunigung und Schwerepotential durch Summation ermittlen. Somit kann auch die Gleichungen in Laplace und Poissongleichung umformulieren und hinzufügen.
\begin{align*}
    &\mathbf{F}_S = \mathbf{F}_G + \mathbf{F}_Z \\
    &\mathbf{g} = \mathbf{a}_G + \mathbf{a}_Z \\
    & W = V_G + V_Z \\
    &\Delta W_{K, aussen} = 2 \omega^2 \\
    &\Delta W_{K, innen} = -4 \pi G \rho + 2 \omega^2
\end{align*}

\subsection{Geometrie des Schwerefeldes}
\subsubsection{Niveauflächen, Äquipotentialflächen}
Flächen mit konstantem Schwerepotential $W = W(r) = const.$ werden als Niveauflächen oder Äquipotentialflächen der Schwere bezeichnet. Entlang einer solchen Fläche wird keine Arbeit geleistet. Die Schwerebeschleunigung $g$ besteht aus mehr als nur der Gravitationsbeschleunigung, deshalb kann diese Fläche sich verschiedenartig zusammensetzen.
\subsubsection{Lotlinien}
Die zugehörigen Lotlinen sind definiert durch 
\begin{equation*}
    \mathbf{g} = \nabla W
\end{equation*}
Die Lotlinien stehen senkrecht auf allen Äquipotentialflächen.
\subsubsection{Astronomische Breite und Länge}
Die astronomische Breite und Länge eines Punktes $P$ ist gegeben durch die Richtung des lokalen Schwerevektors.
\subsubsection{Geometrie des Schwerefeldes}
Das Erdschwerefeld lässt sich au neben der physikalischen Deutung auch rein geometrisch definieren. Die Äquipotentialflächen in jedem Punkt horizontal zu den Lotlinien. Diese lassen sich durch einen Satz bestimmen mittels Konstanten $C$. \\

Ausserhalb der Erde sind diese Flächen unendlich oft differenzierbar.Innerhalb der Massen sind sie stetig und stetig differenzierbar. Erst in der zweiten und dritten Ableitung sind diese nicht stetig differenzierbar. Aus diesem Grund sind solche Flächen glatt und schneiden sich nicht. \\

Im erdnahen sind Niveauflächen konvex, das heisst sie besitzen keine Dellen. Alle ungestörten Wasserflächen sind Teil einer Äquipotentialfläche und eine Wasserwaage würde ich an jedem Ort einer Niveaufläche einspielen. \\

Die idealisierte Oberfläche der Weltmeere nennt man Geoid. Dieser ist befreit von Effekten wie Strömungen, Wirbeln, und Wellen - und ist teil einer Äquipotentialfläche. Der Geoid wird als ideale Bezugsfläche verwendet (m. über Meer, Höhen über Normal Null). \\

Natürlich ist es schwierig mit dem Geoid das Bezugsellipsoid zu vergleichen. Messungen z.B. in Indien und Guinea weichen zwischen -100m bis 80m vom idealen Geoid ab. Die Geoidhöhen korrelieren nicht mit der Verteilung der Kontinente. \\

Mit zunehmendem Abstand von der Erde werden die Äquipotentialflächen glatter, deshalb wird auch jenes mit der Zeit immer kugelähnlicher, wenn nur das Gravitationspotential wirken würde. Aufgrund von Zentrifugalspotential und anderen Effekten ist dies leider nicht der Fall. \\

Die Lotlinien durchschneiden in jedem Punkt die Äquipotentialflächen senkrecht. Die Schwerevektoren sind in jedem Punkt tangential zu den Lotlinien. Diese Lotlinien schneiden sich nicht und sind leicht zu den Polen hin gekrümmt.
\begin{equation*}
    dW = \mathbf{g} \cdot \mathbf{dx}
\end{equation*}
Entlang einem Wegelement $dx$ verändert sich das Schwerepotential nicht, wenn $g$ rechtwinklig dazu steht (Skalarprodukt). Somit ist $dx$ tangential zur Potentialfläche. Zeigt $dx$ in Richtung des Zenits, also nach oben, dann ist
\begin{equation*}
    \mathbf{g} \cdot \mathbf{dx} = -g dH = dW,~~ g = -\frac{\partial W}{\partial H}
\end{equation*}
Das Schwerepotential $W$ verändert sich mit dem nivellierten Höhenunterschied $dH$. \\

Das Schwerepotential von einem Punkt $Q$ auf der gleichen Äquipotentialfläche liegend mit $P$, kann angenähert durch eine Taylorentwicklung und schlussendlich eine Differentialgleichung zweiter Ordnung, welche zu einem sogenannten Eotvös-Tensor führt. Dieser kann verwendet werden, um allgemein überall die jeweiligen Schwerepotentiale auf Äquipotentialflächen zu bestimmen
\begin{equation*}
    W_{gf} = \begin{pmatrix}
        W_{xx} & W_{xy} & W_{xz} \\
        W_{yx} & W_{yy} & W_{yz} \\
        W_{zx} & W_{zy} & W_{zz}
    \end{pmatrix} = - g \begin{pmatrix}
        k_1 & t_1 & f_1 \\
        t_1 & k_2 & f_2 \\
        f_1 & f_2 & -2 \overline{H} + \frac{4 \pi G \rho - 2 \omega^2}{g}
    \end{pmatrix}
\end{equation*}
\section{Satelitenbahnen}
\subsection{Das Zweikörperproblem: Die ungestörte Satelitenbahn}
Eine Bewegungsgleichung für je die kleine Masse $m$ und die grosse Masse $M$ kann basierend auf dem Gravitationsgesetz hergeleitet werden. Die Distanz zwischen beiden Körper lässt sich durch $\mathbf{r} = \mathbf{x}-\mathbf{x}_E$ bestimmen. Diese lässt sich ebenfalls nach der Beschleunigung ableiten und somit folgende Bewegungsgleichungen für Sateliten um einen grossen Körper $M$ im Zentrum des Koordinatensystems bestimmen:
\begin{equation*}
    \ddot{r} = \ddot{x}-\ddot{x}_E = -GM \frac{\mathbf{r}}{|r|^3} -  Gm \frac{\mathbf{r}}{|r|^3} = -G(M+m)\frac{\mathbf{r}}{|r|^3}
\end{equation*}
Oft gilt $m<< M$, also wird oft nur die Konstante $GM$ genommen. Diese Gleihchung ist eine DGL 2. Ordnung und auch vektoriell, daher sind 6 Integrationskonstanten benötigt. Diese sind einerseits durch die Angabe der Orts- und Geschwindigkeitsvektoren zum Zeitpunkt $t_0$ gegeben, oder durch ein Set von den 6 Keplerschen Elemente der Bahn. Die Grössen können ineinander umgerechnet werden. \\

Die Lösungen der ungestörten Bewegungsgleichung sind Kegelschnitte (Ellipse, Parabel, Hyperbel). Die Keplerschen Elemente sind Integrationskonstanten des ungestörten Zweikörperproblems. Sie sind zeitlich konstant und beschreiben die ungestörte Bahn eindeutig.
\subsection{Definition der Keplerschen Bahnelemente}
Die Anfangsbedingunen $r_0$ und $\dot{r}$ beziehen sich im Falle einer Erdumflaufbahn gewöhnlich auf ein Äquatorsystem zu einer bestimmten Epoche $J$. Die Lage des Koordinatensystems wird folgendermassen bestimmt.
\begin{enumerate}
    \item Ursprung im Erdmittelpunkt (Massenschwerpunkt)
    \item Äquator zur Epoche J als Fundamentalebene. Die Koordinatenachse $e_3$ steht hierzu senkrecht
    \item Koordinatenachse $e_1$ in Richtung des Frühlingspunktes
    \item $e_2$ ist zu $e_1$ und $e_3$ orthogonal, und bildet ein rechtshändiges System
\end{enumerate}
Die sechs Keplerschen Bahnelemente sind dann wie folgt definiert:
\begin{enumerate}
    \item Parameter $p$ des Kegelschnittes oder grosse Halbachse $a$ mit $a = p/(1-e^2)$ (Ellipse) oder $a = p/(e^2-1)$ (Hyperbel)
    \item Numerische Exzentrizität $e$ (Kreis: $e=0$, Ellipse $0 < e < 1$, Parabel $e=1$, Hyperbel $e>1$) 
    \item Rektaszension des aufsteigenden Knotens $\Omega$ gemessen vom Frühlingspunkt entlang des Äquators zur Schnittlinie zwischen Bahn- und Äquatorebene
    \item Bahnneigung $i$ gegenüber Äquatorebene
    \item Der Perigäumsabstand $\omega$ ist der Winkel vom aufsteigenden Knoten der Bahnebene gemessen zum Perigäum
    \item Perigäumsdurchgangszeit $T_0$: Zeitpunkt, zu dem der erdnächste Punkt durchlaufen wird.
\end{enumerate}
Die Elemente $p$ oder $a$ und $e$ beschreiben die Form der Bahn. Die Elemente $\Omega$, $i$ und $\omega$ beschreiben die Lage der Bahn im Raum und $T_0$ enthält dynamische Information, der Zeitpunkt wann der Massepunkt einen bestimmten Punkt der Bahn durchläuft.
\subsection{Keplersche Bahnelemente $\rightarrow$ Anfangsbedingungen}
Hier werden bei gegebenen Keplerschen Bahnelementen die Anfangsbedingungen $\mathbf{r}$ und $\mathbf{\dot{r}}$ zum Zeitpunkt $t$.
\begin{enumerate}
    \item Zuerst muss die mittlere Bewegung $n$ bestimmt werden, danach die mittlere Anomalie $M(t)$:\\
     $n = \sqrt{\frac{GM}{a^3}}$, $M(t) = n\cdot(t-T_0)$
    \item Für Ellipsen und Hyperbeln wird iterativ die Lösung der Keplergleichung berechnet (Gleichungen implizit): \\
    $E = M + e \sin E$ oder $H=-M +e \sinh H$
    \item Die wahre Anomalie $v$ lässt sich dadurch bestimmen:\\
    $\tan \frac{v}{2} = \sqrt{\frac{1+e}{1-e}}\tan \frac{E}{2}$ oder $\tan \frac{v}{2} = \sqrt{\frac{e+1}{e-1}}\tanh \frac{H}{2}$
    \item Der Radiusvektor $r$ ist definiert als:\\
    $r = \frac{p}{1+e \cos v}$
    \item Die Geschwindikeitsvektoren und Ortsvektoren in der Bahnebene sind somit bestimmt. Die Koordinaten müssen via einer 3-1-3 Umdrehung in in das originale Koordinatensystem umgerechnet werden.
\end{enumerate}
\subsection{Anfangsbedingunen $\rightarrow$ Keplersche Bahnelemente}
Sind die Anfangsbedingungen gegeben, dann können die Keplerschen Bahnelemente bestimmt werden. Zunächst ist das ganze Koordinatensystem auf der Äquatorebene.
\begin{enumerate}
    \item Aus den beiden Vektoren der Geschwindigkeit und Ort ergibt sich die Grösse $h$, welche sozusagen der massenlose Drehimpuls repräsentiert.
    \item Aus den trigonometrischen Beziehungen lassen sich die Winkel $\Omega$ und $i$ bestimmen
    \item Zudem gibt es eine Beziehung zwischen $h$ und p, also ist $p$ bestimmt
    \item Zudem gelten Beziehungen zwischen dem Ortsvekoren, den Geschwindigkeiten $r$ und $\dot{r}$ mit den Grössen $e$ und $v_0$.
    \item Aus den gegebenen Grössen $p$ und $e$ lässt sich die Halbachse bestimmen
    \item Durch eine Rotation um $\Omega$ auf der 3. Achse, sowie einer Rotation auf der ersten um $i$ erhält man nun die Koordinaten in der Bahnebene
    \item Die Koordinaten des Ortsvektors in der Bahnebene ergibt das Argument der Breite $u_0$. Mittels dem vorherbestimmen $v_0$ ergibt sich nun $\omega$, den Perigäumsabstand
    \item Schlussendlich müsste die Pärigäumsdurchgangszeit $T_0$ mittels der Keplergleichung herausgerechnet werden
\end{enumerate}
\subsection{Gestörte Satelitenbahnen}
In wirklichkeit bewegen sich Sateliten nicht gemäss der ungestörten Bewegungsgleichung. Es kommen Störfaktoren hinzu, welche zusätliche Beschleunigungsterme hinzufügen. Faktoren welche zur Störung führen wären:
\begin{itemize}
    \item Erdschwerefeld (muss mitberücksichtigt werden)
    \item Abplattung der Erde
    \item Gravitation Mond
    \item Gravitation Sonne
    \item Luftwiderstand (CHAMP, da nahe an Erde)
    \item Weitere Erdschwerefeldkoeffizienten
    \item Strahlungsdruck der Sonne
    \item Y-Bias (GPS-Spezifischer Yaw, also Eigendrehung)
    \item Feste Erdgezeiten
    \item Meeresgezeiten
    \item Erdalbedo (Rückstrahlung Erde)
    \item relativistische Effekte
    \item Thermische Emmisionen
\end{itemize}
Diese Faktoren fügen sozusagen wirkende Störbeschleunigungen, wobei $q$ Parameter, dynamische Parameter die einzelnen Störbeschreibungen beschreiben und oft nich sehr genau bekannt sind und in der Bahnbestimmung nebst den Anfangsbedingungen mitbestimmt werden müssen. \\

Somit ist die Differenzialgleichung nicht mehr analytisch mehr lösbar, und es müssen numerische Integrationen gemacht werden. Aufgrund der Störkräfte sind Satelitenbahnen nicht mehr rein elliptisch, sondern nur näherungsweise.
\section{Bestimmung des Gravitationsfeldes der Erde mit Satelitenmissionen}
\subsection{Kugelfunktionsentwicklung des Gravitationsfeldes}
\subsubsection{Laplace-Operator in Kugelkoordinaten}
Der Laplace-Operator in Kugelkoordinaten sieht folgendermassen aus:
\begin{equation*}
    \Delta = \frac{\partial^2}{\partial r^2} + \frac{2}{r}\frac{\partial}{\partial r} + \frac{1}{r^2}\frac{\partial^2}{\partial \theta^2} + \frac{\cot \theta}{r^2}\frac{\partial}{\partial \theta} + \frac{1}{r^2 \sin^2 \theta} \frac{\partial^2}{\partial \lambda^2}
\end{equation*}
Diese hilft später, das Gravitationsfeld in einer anderen Form zu beschreiben.
\subsubsection{Lösung der Laplace-Gleichung in Kugelkoordinaten}
Der Laplaceoperator angewandt auf das Gravitationspotential $V$ ergibt Null für Lösungen im Aussenraum. Mittels eines Separationsansatzes macht man das Gravitationspotential zu einem Produkt verschiedenere Funktionen, welche von den einzelnen Koordinatne $r$, $\theta$ und $\lambda$ abhängig sind.
\begin{equation*}
    V(r,\theta, \lambda) = f(r) \cdot g(\theta) \cdot h(\lambda)
\end{equation*}
Somit entstehen drei separate Differentialgleichungen di mit Standardansätzen gelöst werden können. Für die Funktion $f$ wird angenommen, dass es sich um Potenzreihen handelt:
\begin{equation*}
    f(r) = r^n~~, \text{ oder } f(r) = \frac{1}{r^{n+1}}
\end{equation*}
Der Ansatz für die Winkel $\theta$ und $\lambda$ sind die Legendrefunktionen und Fourierreihen. Mit den Randbedingungen nach Dirichlet, nämlich dass mit dem Radius im Unendlichen das Potential Null sein muss, sowie bei $r=R$ exakt das Potential eine Funktion von $\theta$ und $\lambda$ sein muss, ergibt sich folgende Gleichung:
\begin{equation*}
    V(r,\theta, \lambda) = \sum_{n=0}^{\infty} \left( \frac{R}{r}\right)^{n+1} \sum_{m=0}^{n} P_{nm}\left[ A_{nm} \cos m \lambda + B_{nm} \sin m\lambda \right]
\end{equation*}
Zudem lassen sich die Koeffizienten dimensionslos darstellen, sodass folgende Funktion definiert werden kann:
\begin{equation*}
    V(r,\theta, \lambda) = \frac{GM}{R}\sum_{n=0}^{\infty} \left( \frac{R}{r}\right)^{n+1} \sum_{m=0}^{n} P_{nm}\left[ C_{nm} \cos m \lambda + S_{nm} \sin m\lambda \right]
\end{equation*}
Zudem könnten anstelle der Legendre-Funktionen auch die normalisierten zugeordneten Legendre-Funktionen als Koeffizienten verwendet werden.
\subsection{Globale Gravitationsmodelle}
Heute stehen viele Gravitationsmodelle zur Verfügung. Das EGM2008 ist bis zum Grad und Orrdnung 2159 und entählt zusätzliche Koeffizienten bis zum Grad 2190 und der Ordnung 2159. \\

Die Koeffizienten werden mittels einer Kombination von GRACE Daten ermittelten Schwerefeldes, und mittleren Freiluft-Schwereanomalien basierend auf einem globalen Gitter mit fünf Bogenminuten Maschenweite berechnet.
\subsection{Satelitenbasierte Schwerefeldmissionen}
Die Bahn eines Sateliten hängt vom Gravitationsfeld der Erde ab. Umgekehrt kann man anhand der Satelitenbahn Informationen über das Gravitationsfeld gewinnen. Für die Bestimmung des Feldes sind tieffliegende Sateliten am günstigsten, da so der Glättungsfaktor besser bestimmt wird. Ebenso sind Koeffizienten mit höherem Grad und Ordnung tendenziell kleiner. \\

Andererseits muss der höhere Luftwiderstand von tieffliegenden Sateliten ausgeglichen werden. Dies ist möglich, wenn auf eine Prüfmasse im Satelit nur die Gravitation wirkts, aber die Hülle von Oberflächenkräften angegriffen wird. Anhand der wirkenden Kräfte auf der Prüfmasse kann die Antriebsleistung variiert werden. \\

Für die Bestimmung der Satelitenbahn und des Gravitationsfeldes der Erde werden heute in erster Linie drei Beobachtunggrössen benutzt:
\begin{itemize}
    \item (CHAMP) SST high-low: Beobachtungen von GPS-Empfängern and Board von LEO-Sateliten (160-2000km Bahnhöhe)
    \item (GRACE) SST low-high: Distanzmessungen zwischen Sateliten mittels Mikrowellen
    \item (GOCE): Gradiometer-Beobahtung dreier Paare dreidimensionaler Beschleunigungsmesser an Board eines LEo, mit denen der vollständige Gravitations-Gradienttensor gemessen wird.
\end{itemize}
SST heisst Satelite-to-Satelite Tracking. Für Sateliten der Beobachtungsgrössen 2 und 3 sind in der Regel auch GPS-Empfänger auf diesen an Board. \\

Bei der Messung des Gradiententensors sind hohe technische Anforderungen gegeben. Die Diagonalelemente des Tensors in der Höhe von einigen 100km müssen im Bereich von $10^{-9} s^{-2}$ liegen. Die Bestimmung der zweiten Ableitung liegt somit im Bereich $10^{-11}$ bis $10^{-13} s^{-2}$. Diese Schweregradiometer dienen bei der Ermittlung der Koeffizienten mit höherem Grad/Ordnung. \\

Der heutige Fokus in der Schwerefeldbestimmung liegt haupstächlich in der zeitlichen Änderung. Somit können Variationen des Schwerefeldes die Differenzen von kurzzeiteigen Lösungen und Langzeit-Modekllen bestimmt werden. Die Änderungen kommen zumeist im Zusammenhang mit Masseverlagerungen im Erdmantel, der Erdoberfläche oder Atmosphäre. Zudem lassen sich säkulare Änderungen, z.B. das Abschmelzen von Eis in Polarregionen und und periodische Variationen (Wassermenge gespeichert auf den Kontinenten) beobachten. \\

Zeitvariable Schwerefeldkoeffizienten sind mit zunehmender Ordnung auch mit Rauschen behaftet und sind weniger signifikant. Somit nimmt man bei monatlichen Modellen Koeffizienten im Bereich $n \approx 60-100$ und $m \approx 45-60$. 
\end{multicols*}
\end{document}

