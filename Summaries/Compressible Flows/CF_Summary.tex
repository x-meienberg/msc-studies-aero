\documentclass[8pt, landscape, fleqn]{scrartcl}
\setlength{\parindent}{0pt}
\usepackage[ngerman]{babel}
%\usepackage[applemac]{inputenc}
\usepackage[utf8]{inputenc}
\usepackage[dvips]{geometry}
\usepackage{latexsym}
\usepackage{multicol}
\usepackage{amsmath}
\usepackage{graphicx}
\usepackage{array}
\usepackage{booktabs}
\usepackage{amsmath}
\usepackage{mathtools}
\usepackage{ulem}
\usepackage{amsfonts}
\usepackage{dsfont}
\usepackage{charter} %%% Schreibart
%\renewcommand{\familydefault}{\sfdefault}



%%%%%%%%%%Paket für Chemische Formeln
\usepackage{chemformula} 
\usepackage[version=3]{mhchem}
%%%%%%%%%%%%%%%%% Farbe
\usepackage{color}

\pagestyle{plain}
\typearea{20}
\columnsep 30pt
\columnseprule .4pt
\setlength{\extrarowheight}{0.9em}

\renewcommand{\arraystretch}{0.8}

\makeatletter
\renewcommand{\section}{\@startsection{section}{1}{0mm}%
{-2\baselineskip}{0.8\baselineskip}%
{\hrule depth 0.2pt width\columnwidth\hrule depth1.5pt
width0.25\columnwidth\vspace*{1.2em}\Large\bfseries\rmfamily}}
\makeatother


\makeatletter
\renewcommand{\subsection}{\@startsection{subsection}{1}{0mm}%
{-2\baselineskip}{0.8\baselineskip}%
{\hrule depth 0.2pt width\columnwidth\hrule depth0.75pt
width0.25\columnwidth\vspace*{1.2em}\large\bfseries\rmfamily}}
\makeatother

\makeatletter
\renewcommand{\subsubsection}{\@startsection{subsubsection}{1}{0mm}%
{-2\baselineskip}{0.8\baselineskip}%
{\hrule depth 0.2pt width\columnwidth\vspace*{1.2em}\normalsize\bfseries\rmfamily}}
\makeatother

\newcommand{\Mx}[1]{\begin{bmatrix}#1\end{bmatrix}}
\begin{document}
\part*{\LARGE\textrm{Compressible Flows $\hfill$ Xeno Meienberg}}
\begin{multicols*}{3}

\section{General Considerations}

\begin{equation}
    \frac{D\rho}{D t} = \frac{\partial \rho}{\partial t} + u_i \frac{\partial \rho}{\partial x_i} \neq 0
\end{equation}

\begin{itemize}
    \item Wave propagation
    \item Convective flows with buoancy
    \item Flows with variable temperature, friction, sources of heat
    \item High speed flows with Mach numbers $Ma \geq 1$
\end{itemize}

Compressible flows can still be described through the continuum model and conservation laws. The assumption is also that the thermodynamic state of the fluid is in a local equilibrium. 

\subsubsection*{Assumptions}

\begin{itemize}
    \item Length scale of flows \emph{large} compared to molecular scales (mean free path $\lambda$)
    \item Length scale of flows \emph{small} compared to the geometric scales (length $L$)
    \item Time scale $\tau_F$ of the flow \emph{long} compared to the molecular process (relaxation) time constants $\tau_R$
\end{itemize}


\subsubsection*{Description of the ``Continuum'' Flow State}

\begin{itemize}
    \item Three components of flow velocity $\underline{u}(\underline{x},t)$
    \item The fluid density $\rho(\underline{x},t)$
    \item The fluid pressure $p(\underline{x},t)$
    \item The energy $e(\underline{x},t)$
\end{itemize}

The required equations are the conservation laws for mass, momentum and energy together with suitable thermodynamic equations of state. With corresponding initial and boundary conditions, the evolution can then be computed.

\section{Thermodynamic Relations}

\subsubsection*{State Variables}

\begin{itemize}
    \item Density: $\rho = \rho(p,T)$
    \item Pressure: $p = p(\rho,T)$
    \item Temperature: $T = T(\rho, p)$
    \item Internal energy: $e = e(\rho, T)$ $[e] = J/kg$
    \item Enthalpy: $h = h(p,T)$
    \item Entropy: $s = s(\rho, T)$
\end{itemize}

\subsubsection*{Van der Waals Gas}

\begin{equation}
    (p+a\rho^2)\left(\frac{1}{\rho}-b\right) = RT
\end{equation}



\subsubsection*{Incompressible Fluid}

\begin{equation}
    \rho = const. \neq \rho(p,T)
\end{equation}

\section{Conservation Laws for Continuum Flows}

\section{Simplification Strategies}

\section{Conservation Laws for Stream Tubes}

\section{Steady one-dimensional Flow without Friction and Heat}

\section{Unsteady one-dimensional Flows}

\section{Two-dimensional steady supersonic Flow}

\section{Method Characteristics for planar homentropic supersonic Flows}

\section{Homentropic Flow around slender Wings}

\section{Homentropic Flow around axisymmetric slender Bodies}

\section{Similarity Relations}

\end{multicols*}
\end{document}

