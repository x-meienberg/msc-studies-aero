\documentclass[8pt, landscape, fleqn]{scrartcl}
\setlength{\parindent}{0pt}
\usepackage[ngerman]{babel}
%\usepackage[applemac]{inputenc}
\usepackage[utf8]{inputenc}
\usepackage[dvips]{geometry}
\usepackage{latexsym}
\usepackage{multicol}
\usepackage{amsmath}
\usepackage{graphicx}
\usepackage{array}
\usepackage{booktabs}
\usepackage{amsmath}
\usepackage{mathtools}
\usepackage{ulem}
\usepackage{amsfonts}
\usepackage{dsfont}
\usepackage{charter} %%% Schreibart
%\renewcommand{\familydefault}{\sfdefault}



%%%%%%%%%%Paket für Chemische Formeln
\usepackage{chemformula} 
\usepackage[version=3]{mhchem}
%%%%%%%%%%%%%%%%% Farbe
\usepackage{color}

\pagestyle{plain}
\typearea{20}
\columnsep 30pt
\columnseprule .4pt
\setlength{\extrarowheight}{0.9em}

\renewcommand{\arraystretch}{0.8}

\makeatletter
\renewcommand{\section}{\@startsection{section}{1}{0mm}%
{-2\baselineskip}{0.8\baselineskip}%
{\hrule depth 0.2pt width\columnwidth\hrule depth1.5pt
width0.25\columnwidth\vspace*{1.2em}\Large\bfseries\rmfamily}}
\makeatother


\makeatletter
\renewcommand{\subsection}{\@startsection{subsection}{1}{0mm}%
{-2\baselineskip}{0.8\baselineskip}%
{\hrule depth 0.2pt width\columnwidth\hrule depth0.75pt
width0.25\columnwidth\vspace*{1.2em}\large\bfseries\rmfamily}}
\makeatother

\makeatletter
\renewcommand{\subsubsection}{\@startsection{subsubsection}{1}{0mm}%
{-2\baselineskip}{0.8\baselineskip}%
{\hrule depth 0.2pt width\columnwidth\vspace*{1.2em}\normalsize\bfseries\rmfamily}}
\makeatother

\newcommand{\Mx}[1]{\begin{bmatrix}#1\end{bmatrix}}
\begin{document}
\part*{\LARGE\textrm{Compressible Flows $\hfill$ Xeno Meienberg}}
\begin{multicols*}{3}

\section{General Considerations}

\begin{equation}
    \frac{D\rho}{D t} = \frac{\partial \rho}{\partial t} + u_i \frac{\partial \rho}{\partial x_i} \neq 0
\end{equation}

\begin{itemize}
    \item Wave propagation
    \item Convective flows with buoancy
    \item Flows with variable temperature, friction, sources of heat
    \item High speed flows with Mach numbers $Ma \geq 1$
\end{itemize}

Compressible flows can still be described through the continuum model and conservation laws. The assumption is also that the thermodynamic state of the fluid is in a local equilibrium. 

\subsubsection*{Assumptions}

\begin{itemize}
    \item Length scale of flows \emph{large} compared to molecular scales (mean free path $\lambda$)
    \item Length scale of flows \emph{small} compared to the geometric scales (length $L$)
    \item Time scale $\tau_F$ of the flow \emph{long} compared to the molecular process (relaxation) time constants $\tau_R$
\end{itemize}


\subsubsection*{Description of the ``Continuum'' Flow State}

\begin{itemize}
    \item Three components of flow velocity $\underline{u}(\underline{x},t)$
    \item The fluid density $\rho(\underline{x},t)$
    \item The fluid pressure $p(\underline{x},t)$
    \item The energy $e(\underline{x},t)$
\end{itemize}

The required equations are the conservation laws for mass, momentum and energy together with suitable thermodynamic equations of state. With corresponding initial and boundary conditions, the evolution can then be computed.

\section{Thermodynamic Relations}

\subsubsection*{State Variables}

\begin{itemize}
    \item Density: $\rho = \rho(p,T)$
    \item Pressure: $p = p(\rho,T)$
    \item Temperature: $T = T(\rho, p)$
    \item Internal energy: $e = e(\rho, T)$ $[e] = J/kg$
    \item Enthalpy: $h = h(p,T)$
    \item Entropy: $s = s(\rho, T)$
\end{itemize}

\subsubsection*{Van der Waals Gas}

\begin{equation}
    (p+a\rho^2)\left(\frac{1}{\rho}-b\right) = RT
\end{equation}



\subsubsection*{Incompressible Fluid}

\begin{equation}
    \rho = const. \neq \rho(p,T)
\end{equation}



\section{Conservation Laws for Continuum Flows}

\begin{equation}
	\frac{Dm}{Dt} = \frac{D}{Dt}\int_{\tilde V} \rho d\tilde V = 0 ~(material~ volume)
\end{equation}

\begin{equation}
	\int_V \frac{\partial \rho}{\partial t} dV + \int_S \rho(\bold{u}\cdot \bold{n}) dS = 0~(Eulerian~Volume)
\end{equation}

\begin{equation}
	\frac{\partial \rho}{\partial t} + \frac{\partial}{\partial x_i}(\rho u_i) = 0~(material~volume~/~index)
\end{equation}

\begin{equation}
	\frac{D\rho}{Dt} = -\rho \frac{\partial u_i}{\partial x_i}~(Eulerian~Volume / ~index)
\end{equation}

\subsubsection*{Mass Conservation}

Material Volume
\begin{align}
    \frac{Dm}{Dt} = \frac{D}{Dt} \int_{\vec{V}} \rho d\vec{V} = 0 \\
    \frac{\partial \rho}{\partial t} + \frac{\partial \rho}{\partial x_i }(\rho u_i) = 0
\end{align}

Eulerian Volume

\begin{align}
    \int_V \frac{\partial \rho}{\partial t} dV + \int_S \rho (\vec{u}\cdot \vec{n}) dS = 0 \\
    \frac{D\rho}{Dt} = -\rho \frac{\partial u_i}{\partial x_i}
\end{align}

\subsubsection*{Momentum Conservation}

\begin{align}
    \frac{\partial}{\partial t} (\rho u_i) + \frac{\partial}{\partial x_j} (\rho u_i u_j) = \frac{\partial}{\partial x_j} \sigma_{ij} + \rho f_i \\
    \rho \frac{D u_i}{Dt} = \frac{\partial}{\partial x_j} \sigma_{ij} + \rho f_i \\
    \sigma_{ij} = -p \delta_{ij} + \tau_{ij} \\
    \tau_{ij} = \mu\left(\frac{\partial u_i}{ \partial x_j} + \frac{\partial u_j}{\partial x_i}\right) + \left( \mu_v - \frac{2}{3} \mu\right) \delta_{ij} \frac{\partial u_k}{\partial x_k} \\
    \rho \frac{D u_i}{D t} = -\frac{\partial p}{\partial x_i} + \frac{\partial}{\partial x_j}\left[\mu \left( \frac{\partial u_i}{x_j} + \frac{\partial u_j}{\partial x_i}\right) + \left( \mu_v - \frac{2}{3}\mu\right) \delta_{ij} \frac{\partial u_k}{\partial x_k}\right] + \rho f_i
\end{align}

\subsubsection*{Energy Conservation}

\begin{align}
    \rho \frac{D}{Dt}(e + \frac{1}{2} u_1^2 ) = \frac{\partial}{\partial x_j} (\sigma_{ij} u_i)) + \rho f_i u_i - \frac{\partial q_i}{\partial x_i} + \rho q_v \\
    \rho \frac{D}{Dt}(e + \frac{1}{2} u_1^2 ) = -\frac{\partial}{\partial x_i}(pu_i) + \frac{\partial }{\partial x_j}(\tau_{ij} u_i) + \rho f_i u_i - \frac{\partial q_i}{\partial x_i} + \rho q_v \\
    \rho u_i \frac{D u_i}{Dt} = \rho \frac{D}{Dt}\left(\frac{u_i^2}{2}\right) = -u_i \frac{\partial p}{\partial x_i} + u_i \frac{\partial}{\partial x_j} \tau_{ij} + \rho f_i u_i \\
    \rho \frac{D e}{D t} = \rho \frac{D}{Dt}\left(e + \frac{1}{2} u_i^2 \right) - \rho \frac{D}{Dt}\left( \frac{u_i^2}{2}\right) = \\ = -p \frac{\partial u_i}{\partial x_i} + \tau_{ij} \frac{\partial u_i}{\partial x_j} + \rho q_v - \frac{\partial q_i}{\partial x_i}
\end{align}

\subsubsection*{Dissipation Function $ \Phi $}

Insert $h = e + \frac{p}{\rho}$ to obtain Enthalpy equation, introduce $h_t = h + \frac{u_i^2}{2}$ and add kinetic energy (p. 15). For perfect gasses, $h = c_p T$, $q_i = -k \frac{dT}{dx}$, derive the temperature equation.

\subsubsection*{Entropy Equation}

\begin{equation}
    \rho T \frac{Ds}{Dt} = \Phi + \rho q_v - \frac{\partial q_i}{\partial x_i}
\end{equation}

\subsubsection*{Vorticity Equation}
\begin{equation}
    \rho \frac{D}{Dt} \left( \frac{\vec{\omega}}{\rho} \right) = \left( \vec{\omega} \cdot \nabla \right) \vec{u} + \frac{1}{\rho^2} \nabla \rho \times \nabla p + \nabla \times \left( \frac{1}{\rho} \nabla \cdot \vec{\tau}\right)
\end{equation}

\subsubsection*{Crocco Theorem (rewritten momentum equation using Enthalpy and Entropy)} 

\section{Simplification Strategies}

\section{Conservation Laws for Stream Tubes}

\section{Steady one-dimensional Flow without Friction and Heat}

\section{Unsteady one-dimensional Flows}

\section{Two-dimensional steady supersonic Flow}

\section{Method Characteristics for planar homentropic supersonic Flows}

\section{Homentropic Flow around slender Wings}

\section{Homentropic Flow around axisymmetric slender Bodies}

\section{Similarity Relations}

\end{multicols*}
\end{document}

