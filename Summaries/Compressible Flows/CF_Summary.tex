\documentclass[8pt, landscape, fleqn]{scrartcl}
\setlength{\parindent}{0pt}
\usepackage[ngerman]{babel}
%\usepackage[applemac]{inputenc}
\usepackage[utf8]{inputenc}
\usepackage[dvips]{geometry}
\usepackage{latexsym}
\usepackage{multicol}
\usepackage{amsmath}
\usepackage{graphicx}
\usepackage{array}
\usepackage{booktabs}
\usepackage{amsmath}
\usepackage{mathtools}
\usepackage{ulem}
\usepackage{amsfonts}
\usepackage{dsfont}
\usepackage{charter} %%% Schreibart
%\renewcommand{\familydefault}{\sfdefault}



%%%%%%%%%%Paket für Chemische Formeln
\usepackage{chemformula} 
\usepackage[version=3]{mhchem}
%%%%%%%%%%%%%%%%% Farbe
\usepackage{color}

\pagestyle{plain}
\typearea{45}
\columnsep 39pt
\columnseprule .4pt
\setlength{\extrarowheight}{0.9em}

\renewcommand{\arraystretch}{0.8}

\makeatletter
\renewcommand{\section}{\@startsection{section}{1}{0mm}%
{-2\baselineskip}{0.8\baselineskip}%
{\hrule depth 0.2pt width\columnwidth\hrule depth1.5pt
width0.25\columnwidth\vspace*{1.2em}\Large\bfseries\rmfamily}}
\makeatother


\makeatletter
\renewcommand{\subsection}{\@startsection{subsection}{1}{0mm}%
{-2\baselineskip}{0.8\baselineskip}%
{\hrule depth 0.2pt width\columnwidth\hrule depth0.75pt
width0.25\columnwidth\vspace*{1.2em}\large\bfseries\rmfamily}}
\makeatother

\makeatletter
\renewcommand{\subsubsection}{\@startsection{subsubsection}{1}{0mm}%
{-2\baselineskip}{0.8\baselineskip}%
{\hrule depth 0.2pt width\columnwidth\vspace*{1.2em}\normalsize\bfseries\rmfamily}}
\makeatother

\newcommand{\Mx}[1]{\begin{bmatrix}#1\end{bmatrix}}
\begin{document}
\part*{\LARGE\textrm{Compressible Flows $\hfill$ Xeno Meienberg}}
\begin{multicols*}{3}

\section{General Considerations}

\begin{equation*}
    \frac{D\rho}{D t} = \frac{\partial \rho}{\partial t} + u_i \frac{\partial \rho}{\partial x_i} \neq 0
\end{equation*}

\begin{itemize}
    \item Wave propagation
    \item Convective flows with buoancy
    \item Flows with variable temperature, friction, sources of heat
    \item High speed flows with Mach numbers $Ma \geq 1$
\end{itemize}

Compressible flows can still be described through the continuum model and conservation laws. The assumption is also that the thermodynamic state of the fluid is in a local equilibrium. \\

\textbf{Assumptions}

\begin{itemize}
    \item Length scale of flows \emph{large} compared to molecular scales (mean free path $\lambda$)
    \item Length scale of flows \emph{small} compared to the geometric scales (length $L$)
    \item Time scale $\tau_F$ of the flow \emph{long} compared to the molecular process (relaxation) time constants $\tau_R$
\end{itemize}


\textbf{Description of the ``Continuum'' Flow State}

\begin{itemize}
    \item Three components of flow velocity $\underline{u}(\underline{x},t)$
    \item The fluid density $\rho(\underline{x},t)$
    \item The fluid pressure $p(\underline{x},t)$
    \item The energy $e(\underline{x},t)$
\end{itemize}

The required equations are the conservation laws for mass, momentum and energy together with suitable thermodynamic equations of state. With corresponding initial and boundary conditions, the evolution can then be computed.

\section{Thermodynamic Relations}

\textbf{State Variables}

\begin{itemize}
    \item Density: $\rho = \rho(p,T)$
    \item Pressure: $p = p(\rho,T)$
    \item Temperature: $T = T(\rho, p)$
    \item Internal energy: $e = e(\rho, T)$ $[e] = J/kg$
    \item Enthalpy: $h = h(p,T)$
    \item Entropy: $s = s(\rho, T)$
\end{itemize}

\textbf{Van der Waals Gas}
\begin{equation*}
    (p+a\rho^2)\left(\frac{1}{\rho}-b\right) = RT
\end{equation*}


\textbf{Incompressible Fluid}
\begin{equation*}
    \rho = const. \neq \rho(p,T)
\end{equation*}

\section{Conservation Laws for Continuum Flows}
\underline{\textbf{Mass Conservation}} \\
\begin{equation*}
	\frac{Dm}{Dt} = \frac{D}{Dt}\int_{\tilde V} \rho d\tilde V = 0 ~\text{(material~ volume)}
\end{equation*}
\begin{equation*}
	\int_V \frac{\partial \rho}{\partial t} dV + \int_S \rho(\bold{u}\cdot \bold{n}) dS = 0~\text{(Eulerian~Volume)}
\end{equation*}
\begin{equation*}
	\frac{\partial \rho}{\partial t} + \frac{\partial}{\partial x_i}(\rho u_i) = 0~\text{(material~volume~/~index)}
\end{equation*}
\begin{equation*}
	\frac{D\rho}{Dt} = -\rho \frac{\partial u_i}{\partial x_i}~\text{(Eulerian~Volume / ~index)}
\end{equation*}

\underline{\textbf{Momentum Conservation}}
\begin{equation*}
    \frac{\partial}{\partial t} (\rho u_i) + \frac{\partial}{\partial x_j} (\rho u_i u_j) = \frac{\partial}{\partial x_j} \sigma_{ij} + \rho f_i
\end{equation*}
\begin{equation*}
    \rho \frac{D u_i}{Dt} = \frac{\partial}{\partial x_j} \sigma_{ij} + \rho f_i
\end{equation*}
\begin{equation*}
    \sigma_{ij} = -p \delta_{ij} + \tau_{ij}
\end{equation*}
\begin{equation*}
    \tau_{ij} = \mu\left(\frac{\partial u_i}{ \partial x_j} + \frac{\partial u_j}{\partial x_i}\right) + \left( \mu_v - \frac{2}{3} \mu\right) \delta_{ij} \frac{\partial u_k}{\partial x_k}
\end{equation*}
\begin{equation*}
    \rho \frac{D u_i}{D t} = -\frac{\partial p}{\partial x_i} + \frac{\partial}{\partial x_j}\left[\mu \left( \frac{\partial u_i}{x_j} + \frac{\partial u_j}{\partial x_i}\right) + \left( \mu_v - \frac{2}{3}\mu\right) \delta_{ij} \frac{\partial u_k}{\partial x_k}\right] + \rho f_i
\end{equation*}

\underline{\textbf{Energy Conservation}}
\begin{equation*}
    \rho \frac{D}{Dt}(e + \frac{1}{2} u_1^2 ) = \frac{\partial}{\partial x_j} (\sigma_{ij} u_i)) + \rho f_i u_i - \frac{\partial q_i}{\partial x_i} + \rho q_v 
\end{equation*}
\begin{equation*} \rho \frac{D}{Dt}(e + \frac{1}{2} u_1^2 ) = -\frac{\partial}{\partial x_i}(pu_i) + \frac{\partial }{\partial x_j}(\tau_{ij} u_i) + \rho f_i u_i - \frac{\partial q_i}{\partial x_i} + \rho q_v
\end{equation*}
    $\rho u_i \frac{D u_i}{Dt} = \rho \frac{D}{Dt}\left(\frac{u_i^2}{2}\right) = -u_i \frac{\partial p}{\partial x_i} + u_i \frac{\partial}{\partial x_j} \tau_{ij} + \rho f_i u_i$
    $\rho \frac{D e}{D t} = \rho \frac{D}{Dt}\left(e + \frac{1}{2} u_i^2 \right) - \rho \frac{D}{Dt}\left( \frac{u_i^2}{2}\right) = \\ = -p \frac{\partial u_i}{\partial x_i} + \tau_{ij} \frac{\partial u_i}{\partial x_j} + \rho q_v - \frac{\partial q_i}{\partial x_i}$ \\

\underline{\textbf{Dissipation Function $ \Phi $}} \\
Insert $h = e + \frac{p}{\rho}$ to obtain Enthalpy equation, introduce $h_t = h + \frac{u_i^2}{2}$ and add kinetic energy (p. 15). For perfect gasses, $h = c_p T$, $q_i = -k \frac{dT}{dx}$, derive the temperature equation. \\

\textbf{Entropy Equation}
\begin{equation*}
    \rho T \frac{Ds}{Dt} = \Phi + \rho q_v - \frac{\partial q_i}{\partial x_i}
\end{equation*}
\textbf{Vorticity Equation}
\begin{equation*}
    \rho \frac{D}{Dt} \left( \frac{\vec{\omega}}{\rho} \right) = \left( \vec{\omega} \cdot \nabla \right) \vec{u} + \frac{1}{\rho^2} \nabla \rho \times \nabla p + \nabla \times \left( \frac{1}{\rho} \nabla \cdot \vec{\tau}\right)
\end{equation*}
\textbf{Crocco Theorem (rewritten momentum equation using Enthalpy and Entropy)}
\begin{equation*}
    \frac{\partial u}{\partial t} + \nabla \left( \frac{1}{2} \vec{u}^2 + h + \psi \right) = \vec{u} \times \vec{\omega} + T \nabla s + \frac{1}{\rho} \nabla \cdot \vec{\tau}
\end{equation*}

\textbf{Compressible Bernoulli}
equation (integrate momentum equation law along particle path). Clasical not feasible
\begin{align*}
    \rho \left( \frac{D h_t}{D t} - f_i u_i\right) = 0 \\
    f_i = - \frac{\partial \psi}{\partial x_i} \\
    \psi \neq \psi(t) \\
    \frac{D}{Dt}\left( h_t + \psi \right) = 0
\end{align*}
Between 2 points along stream line
\begin{equation*}
    h_t + \psi = e + \frac{p}{\rho} + \frac{u_i^2}{2} + \psi = const.
\end{equation*}

\section{Simplification Strategies (p.20)}

\begin{itemize}
    \item Unsteady $\rightarrow$ steady (no wave propagation) (no time dependence)
    \item 3D $\rightarrow$ 2D $\rightarrow$ quasi 1-D
    \item Viscous, heat conduction $\rightarrow$ inviscid, adiabatic (isentropic, homentropic)
    \item Subsonic $\rightarrow$ transonic $\rightarrow$ supersonic $\rightarrow$ hypersonic (Elliptic $\rightarrow$ hyperbolic)
    \item Full nonlinear $\rightarrow$ linearised (solve for small pertubations around predefined flow state unique solvable problem, separation of influencing factors facilitated)
\end{itemize}

\section{Conservation Laws for Stream Tubes (p. 22)}

Quasi 1D, separate for environment. Outer surface formed by instantaneous streamlines, no flow across boundaries. Inlet + outlet. Shape (t). For small enough $A$, flow properties can be treated constant in any cross section. \\

\textbf{Mass Conservation}
\begin{equation*}
    \int_1^2 \frac{\partial}{\partial t}\left[ \rho(s,t) A(s,t)\right] ds + \rho_2 A_2 u_2 - \rho_1 A_1 u_1 = 0
\end{equation*}
\begin{equation*}
    \dot{m} = \rho A u = const.
\end{equation*}

\textbf{Momentum Conservation}
\begin{align*}
    \int_1^2 \frac{\partial}{\partial t} \left[ \rho(s,t) A(s,t) \right] ds + \rho_2 A_2 u_2 \vec{u}_2 - \rho_1 A_1 u_1 \vec{u}_1 = \\
    = -p_2 A_2 \vec{n}_2 + p_1 A_1 \vec{n}_1 + F_\tau |_1^2 + F_S
\end{align*}

Steady, frictionless
\begin{equation*}
    \rho_2 u_2^2 + p_2 = \rho_1 u_1^2 + p_1
\end{equation*}

\textbf{Energy Conservation (p.20)}

Steady, frictionless
\begin{equation*}
    e_2 + \frac{u_2^2}{2} + \frac{p_2}{\rho_2} = e_1 + \frac{u_1^2}{2} + \frac{p_1}{\rho_1}
\end{equation*}

Enthalpy subsitution $h = e + \frac{p}{\rho} \rightarrow h_{t1} = h_{t2} = const.$

\section{Steady one-dimensional Flow without Friction and Heat (p. 25)}

Assumptions: 

\begin{itemize}
    \item No friction (inviscid)
    \item No heat source or transport
    \item No flow through mantle
    \item Perfect gas
\end{itemize}
\begin{equation*}
    Ma = \frac{u}{a}
\end{equation*}
\begin{equation*}
    a^2 = \gamma R T 
\end{equation*}
Stagnation properties, when $u=0$ (Ruhegrösse), subscript 0: 
\begin{equation*}
    \frac{h_0}{h} = \frac{T_0}{T} = \left( \frac{a_0^2}{a^2}\right) = 1 + \frac{\gamma - 1}{2} Ma^2
\end{equation*}

Isentropic flow (p.26):
\begin{equation*}
    \frac{p_0}{p} = \left( \frac{T_0}{T}\right)^{\frac{\gamma}{\gamma -1}} = \left[ 1 + \frac{\gamma-1}{2} Ma^2 \right]^{\frac{\gamma}{\gamma-1}}
\end{equation*}
\begin{equation*}
    \frac{\rho_0}{\rho} = \left( \frac{T_0}{T} \right)^{\frac{1}{\gamma-1}} = \left[ 1 + \frac{\gamma-1}{2} Ma^2 \right]^{\frac{1}{\gamma-1}}
\end{equation*}
When $Ma < 0.3$, density changes $< 4.5 \%$: Assumption is: incompressible. The critical state is then ($Ma = 1$), $superscript~^*$
\begin{equation*}
    \frac{h^*}{h_0} = \frac{T^*}{T_0} = \left( \frac{a^{*2}}{a_0^2}\right) = \left[ 1 + \frac{\gamma -1 }{2}\right]^{-1} = \frac{2}{\gamma+1} = 0.8333 (\gamma = 1.4)
\end{equation*}
\begin{equation*}
    \frac{p^*}{p_0} = \left( \frac{2}{\gamma+1} \right)^{\frac{\gamma}{\gamma-1}} = 0.5283 (\gamma = 1.4)
\end{equation*}
\begin{equation*}
    \frac{\rho^*}{\rho_0} = \left( \frac{2}{\gamma+1} \right)^{\frac{1}{\gamma-1}} = 0.6339 (\gamma = 1.4)
\end{equation*}
Critical $Ma^*$ (isentropic flow stays limited when $Ma \rightarrow \infty$). The flow velocity stays finite even if $Ma$ goes to infinity:
\begin{align*}
    Ma^* = \frac{u}{a^*} = \frac{u}{a(Ma=1)} = \frac{u}{a}\frac{a}{a_0}\frac{a_0}{a^*} \\
    = Ma \sqrt{\frac{T}{T_0}}\sqrt{\frac{T_0}{T^*}} = \sqrt{\frac{\frac{\gamma+1}{2}Ma^2}{1+\frac{\gamma-1}{2}Ma^2}}
\end{align*}
\begin{equation*}
    Ma^* \rightarrow \sqrt{\frac{\gamma+1}{\gamma-1}}~(Ma \rightarrow \infty) = 2.4495~(\gamma = 1.4)
\end{equation*}

\underline{\textbf{Area velocity relation}}

A velocity increase $\rightarrow$ density decrease (always). If $Ma << 1$, then the density changes are small compared to the velocity changes. A small velocity increase at $Ma >> 1$ will lead to large density changes.
\begin{equation*}
    Ma^2 \frac{1}{u} \frac{d u}{d x} = -\frac{1}{\rho} \frac{d \rho}{d x}~~\textbf{(Mach-density relation)}
\end{equation*}
\begin{equation*}
    \left( Ma^2 -1 \right) \frac{1}{u} \frac{d u}{d x} = \frac{1}{A} \frac{d A}{d x}~~\textbf{(Mach-Area relation)}
\end{equation*}

If $Ma < 1$, then an area increase will result in a velocity reduction. If $Ma > 1$, then opposite applies. If $Ma = 1$, then a change in Area $A$ has no effect (chocked flow) \\

\underline{\textbf{Stationary normal shock}}
\begin{align*}
    \frac{u_2}{u_1} &= \frac{\rho_1}{\rho_2} = 1 - \frac{2}{\gamma+1} \left( 1- \frac{1}{Ma_1^2}\right) = \frac{1}{Ma^{*2}} \\
    \frac{p_2}{p_1} &= 1 + \frac{2 \gamma}{\gamma + 1} \left( Ma_1^2 - 1 \right) \\
    \frac{T_2}{T_1} &= \left[ 1 + \frac{2 \gamma}{\gamma + 1} \left( Ma_1^2 -1 \right) \right]\left[ 1-\frac{2}{\gamma+1} \left( 1-\frac{1}{Ma_1^2} \right) \right] \\
    &\frac{\Delta s}{R} = \frac{1}{\gamma-1} \left[ \ln \left( \frac{p_2}{p_1} \right) - \gamma \ln \left( \frac{\rho_2}{\rho_1}\right)\right] = \\
    & \frac{1}{\gamma-1}\left\{\left[ 1 + \frac{2\gamma}{\gamma+1}\left( Ma_1^2 -1 \right)\right] \left[ 1-\frac{2}{\gamma+1}\left(1-\frac{1}{Ma_1^2}\right)\right]\right\}
\end{align*}

$h_{01} = h_{02}$, $T_{01} = T_{02}$, and total enthalpy conserved (however stagnation pressure not constant, $p_{01}\neq p_{02}$):
\begin{align*}
    &\frac{p_{02}}{p_{01}} = \frac{p_{02}}{p_{2}} \frac{p_2}{p_1} \frac{p_1}{p_{01}} = \frac{p_2}{p_1} \left( \frac{T_{02}}{T_2} \right)^{\frac{\gamma}{\gamma-1}}\left(\frac{T_1}{T_{01}}\right)^{\frac{\gamma}{\gamma-1}} = \\
    & \left[ 1 + \frac{2 \gamma}{\gamma + 1} (Ma_1^2 -1)\right]^{\frac{\-1}{\gamma-1}} \left[1- \frac{2}{\gamma+1} \left( 1-\frac{1}{Ma_1^2}\right)\right]^{\frac{-\gamma}{\gamma-1}} 
\end{align*}
As $s$ increases, $u$ decreases. $Ma_2$ is always $<1$, when $Ma_1 \rightarrow \infty$:
\begin{equation*}
    Ma_2 \rightarrow \sqrt{\frac{\gamma-1}{2\gamma}} = 0.38 ~ (\gamma = 1.4)
\end{equation*}
\begin{equation*}
    Ma_2^2 = \left( \frac{u_2}{a_2}\right)^2 = \left( \frac{u_2}{u_1}\right)^2 \left( \frac{u_1}{a_1}\right)^2 \left( \frac{a_1}{a_2}\right)^2 = \left( \frac{u_2}{u_1}\right)^2 Ma_1^2 \left( \frac{T_1}{T_2}\right)
\end{equation*}
\begin{equation*}
    Ma_2 = \sqrt{\frac{1 + \frac{\gamma -1}{\gamma + 1} \left( Ma_1^2 -1\right)}{1 + \frac{2\gamma}{\gamma + 1}\left( Ma_1^2 - 1\right)}}
\end{equation*}

A weak shock occurs at $Ma_1$ close to one. See page 31 for equation \\

\underline{\textbf{Rankine Hugoniot (p.32) - Adiabatic Shock (no $Ma$ dependency)}}
\begin{equation*}
    \frac{p_2}{p_1} = 1 + \frac{2 \gamma \left( \frac{\rho_2}{\rho_1}-1\right)}{\gamma + 1 - (\gamma -1)\frac{\rho_2}{\rho_1}}
\end{equation*} 
\underline{\textbf{Moving Shock Wave (p.33)}}

Switch to reference frame (from frame fixed with moving shock front into a frame moving with shock)
\begin{equation*}
    u_1 \widehat{=} u_s~,~ p_1 \widehat{=} p_0~,~ \rho_1 \widehat{=} \rho_0
\end{equation*}
Flow behind
\begin{equation*}
    u_2 \widehat{=} u_s-u_d~,~ p_2 \widehat{=} p_d~,~ \rho_2 \widehat{=} \rho_d
\end{equation*}
Shock $u_d$
\begin{equation*}
    u_d = u_s - u_2 = u_1 - u_2 = u_1 \left( 1 - \frac{u_2}{u_1}\right) = u_1 \frac{2}{\gamma+1}\left(1- \frac{1}{Ma_1^2}\right)
\end{equation*}
\begin{equation*}
    Ma_d = \frac{u_d}{a_d} = \frac{u_1-u_2}{a_d} = \frac{u_1}{a_1}\frac{a_1}{a_d}\left( 1- \frac{u_2}{u_1}\right) = Ma_1 \sqrt{\frac{T_1}{T_2}}\left(1-\frac{u_2}{u_1}\right)
\end{equation*}
\begin{equation*}
    u_d = \frac{a_0}{\gamma} \frac{\frac{\Delta p}{p_0}}{\sqrt{1 + \frac{\gamma+1}{2\gamma} \frac{\Delta p}{p_0}}}~~(a_1 \hat{=} a_0), ~~ Ma_s = \frac{u_s}{a_0} = \sqrt{1 + \frac{\gamma+1}{2\gamma}\frac{\Delta p}{p_0}}
\end{equation*}

Pressure increase
\begin{equation*}
    \frac{\Delta p}{p_0} = \frac{p_d - p_0}{p_0} = \frac{2 \gamma}{\gamma + 1} \left( Ma_S^2 -1 \right),~~[Ma_1 = \frac{u_1}{a_1} = \frac{u_s}{a_s}=Ma_s]
\end{equation*}

The ratio (Pressure increase) has an asymptotic limit. For high $Ma_s$, the function becomes limited. $\frac{u_s}{u_d} \rightarrow \frac{\gamma+1}{2}$ (for high pressure differences) \newline

\underline{\textbf{Detonations ($Ma_2 >1$) and Deflagrations ($Ma_2 < 1$) (p.36, ZND)}}

\textbf{Assumption: Ignore adiabatic flow, include however heat release}

Rayleigh line: $\frac{p_1}{p_0} = 1 + \frac{\rho_0}{p_0} u_0^2 - \frac{\rho_0}{p_0}\frac{\rho_1}{\rho_0}u_1^2 = 1 + \gamma Ma_0^2 \left( 1 - \frac{\rho_0}{\rho_1}\right)$,
Rankine Hugeniot with heat: $\frac{p_2}{p_0} = \frac{(\gamma+1)-(\gamma-1)\frac{\rho_0}{\rho_2}+2\gamma \hat{q}}{(\gamma+1)\frac{\rho_0}{\rho_2}-(\gamma-1)}~,~ \hat{q} = \frac{q_{heat}}{c_p T_1}$, This gives us $p_1$ and $p_2$, the pressure of the shockwave before the combustion and downstream after the combustion layer \\

\underline{\textbf{Chapman-Jouget Point (p.37)}}

...is the intersection where $Ma=1$, so $Ma_2 = 1 = Ma_0 \sqrt{\frac{\rho_0}{\rho_2}} \sqrt{\frac{\rho_0}{\rho_2}}$
The limiting case for shock cycle (Rayleigh tangent to Hugoniot Line ): 
\begin{equation*}
    \frac{\rho_0}{\rho_2} |_c = \frac{u_2}{u_0} |_c = \frac{ \gamma Ma_0^2 + 1 }{Ma_0^2 (\gamma + 1)}
\end{equation*}

Behind the shock, the flow is subsonic $\leftrightarrow$ strong detonation. There is a weak deflagration if the density ratio $\frac{\rho_1}{\rho_2} >> 1$. The reaction front propagates at subsonic speed.  Weak detonation: flow remains supersonic (not explainable through ZND)\\

\underline{\textbf{Laval Nozzle (p. 39)}}

Varying cross-section:
\begin{align*}
    \frac{p(x)}{p_0} &= \left[ 1 + \frac{\gamma - 1}{2} Ma^2 (x) \right]^{\frac{-\gamma}{\gamma-1}} \\
    \frac{A^*}{A(x)} &= Ma(x) \left[ \frac{2}{\gamma+1} + \frac{\gamma-1}{\gamma+1}Ma^2(x)\right] \\
    u(x) &= Ma(x) a_0 \frac{a(x)}{a_0} = Ma(x) a_0 \sqrt{\frac{T(x)}{T_0}} = \frac{a_0 \cdot Ma(x)}{\sqrt{1 + \frac{\gamma-1}{2}Ma^2(x)}} \\
    u^* &= a^* ~,~ \text{if}~ Ma^* = 1
\end{align*}

In order to increase the $Ma_{exit}$, reduce the area ration (tune $A^*$). Different flow regimes are shown on p. 41. A variable exit area is in practice not possible

\section{Unsteady one-dimensional Flows}

\textbf{Wave equation for small perturbations}
Assuming small pertubations around equilibrium state with first order pertubations will result into following differential equation (enthalpy):

\begin{align*}
    \frac{\partial p'}{\partial t} - a_0^2 \frac{\partial \rho'}{\partial t} &= 0 \Longleftrightarrow p' = a_0^2 \rho' \\
    \frac{\partial \rho'}{\partial t} + \rho_0 \frac{\partial u'}{\partial x} &= 0 ~~ (\text{mass eq.}) \\
    \frac{\partial u'}{\partial t} + \frac{a_0^2}{\rho_0} \frac{\partial \rho'}{\partial x} &= 0 ~~ (\text{momentum eq.})
\end{align*}

Through cross-differentiation (elimination of terms), one arrives at the d'Alembert solution:
\begin{align*}
    u'(x,t) &= a_0 [F(x-a_0t) + G(x+a_0t)] \\
    \rho'(x,t) &= \rho_0 [F(x-a_0t) + G(x+a_0t)]
\end{align*}
Through characteristics one defines left and right propagatiting waves, $F(\eta)$ and $G(\xi)$. The characteristics are in this case straight lines. Initial conditions are at $t=0$, boundary conditions are at $x=b.c.$ \\

\textbf{Method of characteristics for nonlinear wave propagation}
Here, no small pertubations are assumed, while assuming homentropic flow ($s=const.$). The Riemann invariants (characteristics) are not straight anymore, and can be curved. Disturbances are no longer constant, but have a flow dependent value. Given $a$ and $u$ are given along a curve $C$, find where it intersects with two characteristics, which cross at point $Q$. (See p. 48) \\ 

\textbf{Piston Motion in tube (example for unsteady one-dimensional motion)}:

\begin{itemize}
    \item Boundary Condition: At $x = x_p(t)$, $u(x=x_p,t) = u_p(t)$
    \item How to solve: Left propagating wave from rest state, intersects $P$ at $u=u_p$. The characterisitic with $\eta = const$ which then can intersect the other characteristic with $\xi = const. $ yields point $Q$
    \item $x = \left[ a_0 + \frac{\gamma + 1}{2} u_p(\tau)\right](t-\tau) + x_p(\tau)$
\end{itemize}

\textbf{Simple expansion waves} \\
In the case for the piston moving to the left, the characteristics are limited by two factors:

\begin{itemize}
    \item $x=a_0 t$: Initially, at $t=0$, the characteristic is maximum and can only be as steep as $a_0$
    \item $u_p=-U$: The piston motion can only have a max. velocity at its endpoints ($x_p = -Ut$ and $Ut$)
    \item This gives an area of solutions, which is called a ``centered fan''
\end{itemize}
\begin{align*}
    Ma &= \frac{|U|}{a_0} \left[ 1- \frac{\gamma -1}{2} \frac{|U|}{a_0} \right]^{-1},
    \frac{\rho}{\rho_0} = \left[ 1- \frac{\gamma-1}{2} \frac{|U|}{a_0} \right]^{\frac{2}{\gamma-1}} \\
    \frac{p}{p_0} &= \left[ 1- \frac{\gamma-1}{2} \frac{|U|}{a_0} \right]^{\frac{2 \gamma}{\gamma-1}}
\end{align*}

\textbf{Simple Compression Waves},
see p. 54, explained for increasing velocity to the right \\ \\ 
\underline{\textbf{Reflections}} \\
\textbf{Reflection from solid wall}: $G=-F$ if boundary moves with velocity $0$ \\

\textbf{Reflection from free boundary (contact surface), p.56}: The ratio $\alpha$ is the impendance, and is the ratio of both $a$ of two regions \\

\textbf{Reflection from an open end with outflow, p.58}: At an orifice ($a$ = outer, $0$ = stagnation), the characteristics are:

\begin{equation*}
    G = F - \frac{4}{\gamma-1} a(p_a)
\end{equation*}

The speed of sound is computed via the isentropic relations:

\begin{equation*}
    \frac{a_a}{a_0} = \sqrt{\frac{T_a}{T_0}} = \left( \frac{p_a}{p_0} \right)^{\frac{\gamma-1}{2\gamma}}
\end{equation*}


\section{Two-dimensional steady supersonic Flow}
An oblique shock wave forms around a body with a sharp tip or a long a wall with a sudden profile change (at the turning point). Two parameters describe the problem. The inflow Mach number $Ma_1$ and turning angle $\theta$. The velocity can be described by a component normal to the shock front and tangential to the shock front $u_n$ and $u_t$. The downstream and upstream pressures $p_1$ and $p_2$ are assumed constant.
\begin{align*}
    u_{2t} &= u_{1t} = u_t \\
    u_{1n} &= u_1 \sin \beta~,~u_{2n} = u_2 \sin(\beta-\theta) \\
    \frac{u_{2n}}{u_{1n}} &= \frac{(\gamma-1) Ma_1^2 \sin^2 \beta + 2}{(\gamma+1) Ma_1^2 \sin^2 \beta}\\
    \frac{u_{2t}}{u_{1t}} &= \frac{\tan (\beta-\theta)}{\tan \beta} = \frac{1-\tan\theta \cot \beta}{1 + \tan \theta \tan \beta} = \frac{1-(\tan\theta/\tan \beta)}{1 + \tan \theta \tan \beta} \\
    \tan \theta &= \frac{1-\frac{u_{2n}}{u_{1n}}}{\cot \beta + (\frac{u_{2n}}{u_{1n}})\tan \beta} = \frac{\left( \sin^2\beta - \frac{1}{Ma_1^2}\right)\sqrt{ 1- \sin^2 \beta}}{\sin \beta \left( \frac{\gamma+1}{2}-\sin^2 \beta + \frac{1}{Ma_1^2}\right)}
\end{align*}
The equation for $\theta$ is an implicit function for $\beta$, whereas there are two solutions for it. Normally, $\theta$ (geometry) and incident $Ma_1$ are given. For $Ma_1 > 1$, $\beta$ is in the range of $[\arcsin (1/Ma_1), 90^{\circ}]$. For a given $Ma_1$, there is a maximum turning angle $\theta_{max}$ beyond which the shock is not anymore at the turning point. The shock is then ``detached'' (set $Ma \rightarrow \infty$ in equation for $\theta$).
\begin{itemize}
    \item Strong and weak shock: The maxima of all shock angle for oblique shocks can be connected. Higher Mach angles $\beta$ occur with strong shocks
    \item Separation between $Ma_2 < 1$ and $Ma_2 > 1$
    \item For the changes in pressure, density and other thermodynamic properties, the equations for the normal shock can be used with a new ''effective Mach number'' $Ma_{1,new} = Ma_1 \sin \beta$
\end{itemize}
\begin{align*}
    u_{2n} &= u_2 \sin (\beta - \theta) \\
    Ma_2^2 \sin^2 (\beta - \theta) &= \frac{1 + \frac{\gamma - 1}{\gamma + 1} (Ma_1^2 \sin^2 \beta - 1)}{1 + \frac{2 \gamma}{\gamma + 1} (Ma_1^2 \sin^2 \beta -1)}
\end{align*}
\begin{itemize}
    \item Small turning angles $\theta$: In the limit of $\theta \rightarrow 0$, $\beta$ becomes the Mach angle $\mu = \lim_{\theta \rightarrow 0} \beta = \arcsin \frac{1}{Ma_1}$
    \item Hypersonic flow $Ma_1 >> 1$: Shock angle and turning angle linearly dependent on gas property $\gamma$ alone, $\sin \beta \approx \beta = \frac{\gamma + 1}{2} \theta$
\end{itemize}

\underline{\textbf{Continuous turning of supersonic flows}} \\
When the surface is moving continuously, the model can be adapted.
\begin{align*}
    \frac{dp}{p} &= - \frac{\gamma Ma}{1 + \frac{\gamma-1}{2}Ma^2} dMa \\
    d\theta &= - \frac{\sqrt{Ma^2-1}}{1 + \frac{\gamma-1}{2}Ma^2}\frac{dMa}{Ma}
\end{align*}

These equations describe the \underline{``Prandtl-Meyer Compression''}. In case the flow direction changes positively ($d\theta > 0$), the Mach number decreases ($dMa < 0$), and the pressure as well ($dp < 0$). 
\begin{align*}
\nu(Ma) &= \sqrt{\frac{\gamma + 1}{\gamma -1}} \arctan \sqrt{\frac{\gamma-1}{\gamma+1}(Ma^2-1)}-\arctan\sqrt{\text{Ma}^2-1}   \\
\nu_{max} &= \nu(Ma \rightarrow \infty) = \frac{\pi}{2}\left( \sqrt{\frac{\gamma+1}{\gamma-1}-1} \right) \hat{=} 130.5^{\circ} ~ (\gamma = 1.4)
\end{align*}

The above \underline{Prandtl-Meyer function} outputs degrees, which couple supersonic turns and change in Mach numbers.
\begin{equation*}
    \Delta \theta = \theta_2 - \theta_1 = -\nu(Ma_2) + \nu(Ma_1)
\end{equation*}
An oblique schock occurs at a distanced point $P$ away from the wall. A \underline{``Prandtl-Meyer Expansion''} occurse if the turning angle is negative. The flow accelerates, and Mach lines diverge. If the flow turns around a point, all lines are going along a center. \\ \\
\underline{\textbf{Reflection and crossing of waves}} \\
Use same equations as before. Here, following things must be considered:
\begin{itemize}
    \item Wedge with bounding wall: Reflections back from wall create additional shocks until $\theta < \theta_{max}$ (function of $\beta$). The Mach numbers decrease
    \item Wedge without bounding wall: Point $P$ is where fluid expands (reflected by the free jet boundary)
    \item Walls which converge: For two different turning angles $\theta$, two oblique shocks meet at point $P$:
    \begin{itemize}
        \item $\theta_1-\theta_5 = \theta_4 - \theta_2$
        \item $p_4(Ma_2, \theta_4, p_2) = p_5(Ma_3, \theta_5, p_3)$
    \end{itemize}
    \item Prismatic wing: First a oblique shock, then fans, then shock due to turning back flow to inflow direction. Even without friction, a drag is introduced
\end{itemize}
\underline{\textbf{Detached Shocks}}
Normally, the oblique shock occurs for $\theta < \theta_{max}$. For larger angles, the shock detaches and shows a near hyperbolic shape. See p.78 for details when it comes to decreasing $Ma$. \\

\underline{\textbf{Supersonic nozzle exit flows}} \\
Supersonic nozzle flow into stagnant environment.
\begin{itemize}
    \item Under-expanded scenario ($p_{env}<p_{noz}$): First PM-expansion due to reduction of pressure. However later on again goes back (rocket plume). Interacting incoming and reflecting expansion fans create the going back.
    \item Over-expanded scenario ($p_{env}>p_{noz}$): The turning angle $\beta$ is much smaller, the plume is impinged
\end{itemize}

\section{Method Characteristics for planar homentropic supersonic Flows}

\section{Homentropic Flow around slender Wings}

\section{Homentropic Flow around axisymmetric slender Bodies}

\section{Similarity Relations}

\section{Steady flows with friction and heat transport}

\end{multicols*}
\end{document}

