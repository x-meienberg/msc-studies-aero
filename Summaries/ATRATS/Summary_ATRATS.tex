\documentclass[8pt, landscape, fleqn]{scrartcl}
\setlength{\parindent}{0pt}
\usepackage[ngerman]{babel}
%\usepackage[applemac]{inputenc}
\usepackage[utf8]{inputenc}
\usepackage[dvips]{geometry}
\usepackage{latexsym}
\usepackage{multicol}
\usepackage{amsmath}
\usepackage{graphicx}
\usepackage{array}
\usepackage{booktabs}
\usepackage{amsmath}
\usepackage{mathtools}
\usepackage{ulem}
\usepackage{amsfonts}
\usepackage{dsfont}
\usepackage{charter} %%% Schreibart
%\renewcommand{\familydefault}{\sfdefault}



%%%%%%%%%%Paket für Chemische Formeln
\usepackage{chemformula} 
\usepackage[version=3]{mhchem}
%%%%%%%%%%%%%%%%% Farbe
\usepackage{color}

\pagestyle{plain}
\typearea{20}
\columnsep 30pt
\columnseprule .4pt
\setlength{\extrarowheight}{0.9em}

\renewcommand{\arraystretch}{0.8}

\makeatletter
\renewcommand{\section}{\@startsection{section}{1}{0mm}%
{-2\baselineskip}{0.8\baselineskip}%
{\hrule depth 0.2pt width\columnwidth\hrule depth1.5pt
width0.25\columnwidth\vspace*{1.2em}\Large\bfseries\rmfamily}}
\makeatother


\makeatletter
\renewcommand{\subsection}{\@startsection{subsection}{1}{0mm}%
{-2\baselineskip}{0.8\baselineskip}%
{\hrule depth 0.2pt width\columnwidth\hrule depth0.75pt
width0.25\columnwidth\vspace*{1.2em}\large\bfseries\rmfamily}}
\makeatother

\makeatletter
\renewcommand{\subsubsection}{\@startsection{subsubsection}{1}{0mm}%
{-2\baselineskip}{0.8\baselineskip}%
{\hrule depth 0.2pt width\columnwidth\vspace*{1.2em}\normalsize\bfseries\rmfamily}}
\makeatother

\newcommand{\Mx}[1]{\begin{bmatrix}#1\end{bmatrix}}
\begin{document}
\part*{\LARGE\textrm{Advanced Techniques Risk Analysis of Technical Systems $\hfill$ Xeno Meienberg}}
\begin{multicols*}{3}

\section{Introduction, Definitions \& Overview}

Reliability 

\begin{itemize}
    \item ... is a characteristic of an item, expressed by the probability that the item performs its required function under given conditions \emph{during} a stated time interval, i.e. $(0,t]$
    \item Item = entity for investigation, i.e. component, assembly, equipment, subsystem, system 
    \item 
\end{itemize}

\section{Probability Theory and Reliability Analysis}

Definitions:

\begin{itemize}
    \item Experiment $\epsilon$
    \item Sample space $\Omega$
    \item Event $E$
\end{itemize}

An event $E$ is a subset of the sample space $\Omega$ and the experiment $\varepsilon$ yields a set of possible outcoms ($= E$) of the experiment  \newline \newline \emph{Certain Events} follow \emph{Boolean Logic}, an event $E$ can occur or not occur, meaning an \emph{Indicator Variable} $X_E$ is 0 when $E$ does not occur and 1 if $E$ occurs \newline \newline
\emph{Uncertain Events} follow can either be true or false, with each a probability associated to it. Event $E$ in sample space $\Omega$ is triggered with a probability that the outcome has happened or not 

\subsubsection*{Classical Probability}

\begin{itemize}
    \item The experiment $\epsilon$ has $N$ possible, elementary, mutually exclusive and equally probable outcomes $A_1, A_2,..., A_N \in \Omega$
    \item The event $E = A_1 \cup A_2 \cup ... \cup A_M$,  $M\leq N$
    \item The probability of event $E$ is defined as $p(E) = M / N$
\end{itemize}

\subsubsection*{Kolmogorov Axioms}

\begin{enumerate}
    \item $0 \leq P(E) \leq 1$
    \item $P(\Omega) = 1$, $P(\emptyset) = 0$
    \item Mutually exclusive events: $P(\cup_i E_i) = \sum p(E_i)$
    \item Non-mutually exlusive events: \newline $P(A\cup B) = P_A + P_B - P(A \cap B)$
    \item Conditional probability: $P(A|B) = P(A\cap B) / P(B)$
    \item Theorem of total probability: Given an event $A$ in $\Omega$ where the space is consisting of exclusive and exhaustive events $\cup_j E_j = \Omega$: $P(A) = \Sigma_i(P(A | E_i)P(E_i))$
\end{enumerate}

\subsection*{Random Variables}

\subsubsection*{CDF}

\subsubsection*{pdf}

\subsubsection*{pmf}




\end{multicols*}
\end{document}

