\documentclass[9pt, landscape, fleqn]{scrartcl}
\setlength{\parindent}{0pt}
\usepackage[ngerman]{babel}
%\usepackage[applemac]{inputenc}
\usepackage[utf8]{inputenc}
\usepackage[dvips]{geometry}
\usepackage{latexsym}
\usepackage{multicol}
\usepackage{amsmath}
\usepackage{graphicx}
\usepackage{array}
\usepackage{booktabs}
\usepackage{amsmath}
\usepackage{mathtools}
\usepackage{ulem}
\usepackage{amsfonts}
\usepackage{dsfont}
\usepackage{charter} %%% Schreibart
%\renewcommand{\familydefault}{\sfdefault}



%%%%%%%%%%Paket für Chemische Formeln
\usepackage{chemformula} 
\usepackage[version=3]{mhchem}
%%%%%%%%%%%%%%%%% Farbe
\usepackage{color}

\pagestyle{plain}
\typearea{45}
\columnsep 30pt
\columnseprule .4pt
\setlength{\extrarowheight}{0.9em}

\renewcommand{\arraystretch}{0.8}

\makeatletter
\renewcommand{\section}{\@startsection{section}{1}{0mm}%
{-2\baselineskip}{0.8\baselineskip}%
{\hrule depth 0.2pt width\columnwidth\hrule depth1.5pt
width0.25\columnwidth\vspace*{1.2em}\Large\bfseries\rmfamily}}
\makeatother


\makeatletter
\renewcommand{\subsection}{\@startsection{subsection}{1}{0mm}%
{-2\baselineskip}{0.8\baselineskip}%
{\hrule depth 0.2pt width\columnwidth\hrule depth0.75pt
width0.25\columnwidth\vspace*{1.2em}\large\bfseries\rmfamily}}
\makeatother

\makeatletter
\renewcommand{\subsubsection}{\@startsection{subsubsection}{1}{0mm}%
{-2\baselineskip}{0.8\baselineskip}%
{\hrule depth 0.2pt width\columnwidth\vspace*{1.2em}\normalsize\bfseries\rmfamily}}
\makeatother

\newcommand{\Mx}[1]{\begin{bmatrix}#1\end{bmatrix}}
\begin{document}
\part*{\LARGE\textrm{Aviation 1 - Summary $\hfill$ Xeno Meienberg}}
\begin{multicols*}{3}

\section{Air Transport as part of overall traffic}
\begin{itemize}
    \item Air transport: On one hand dismissed as commodity, on the other as as magnet for the population (high interest)
    \item Economist perspective: Air transport as part of econom. transport system
    \item Modern economy: Division of labour: Pre-requisite for this is a functioning air transport system
    \item Air transport is an \textbf{indicator of wealth and poverty}
    \item \textbf{Globalisation} builds strongly on air transport 
    \item Strong growth projected (based on CAGR of 3.7\% $\rightarrow$ 7.2 billion in 2035 / double of 2016 )
\end{itemize}

\underline{\textbf{Air transport vs. Aviation}}
\begin{itemize}
    \item \textbf{Air traffic/transport}: All operations used to change location of people, freight and post by air and incorporates all services directly associated with the change of location (flight, catering, airport)
    \item \textbf{Aviation}: Air transport + in-kind services to produce air transport services (manufacturing of airplanes and traffic control systems)
\end{itemize}

\underline{\textbf{Systemization of Air Transport}}:
\begin{itemize}
    \item Functional specification: Civil/Military
    \item Transport Object: Passenger/Freight/Post
    \item Commercial: Public/not public
    \item Non-commercial (not-public): Factory flights, company (internal), private, state
    \item Length of leg: Short (2000km), Mid (5000km), Long
    \item Legal: Inland(domestic), Cross-border (international)
    \item Aircraft Type (Engine): Turbo-prop, Jet, Piston engine
    \item Regularity: Regular (scheduled), On demand (chartered)
    \item Business model: Network / charter / low cost / business jet 
\end{itemize}

\underline{\textbf{Specialities reg. Supply and Demand}}
In contrast to other modes of transport, air transport has additional, special characteristics:
\begin{itemize}
    \item Governmental framework conditions (regulations, state carriers, cabotage ban (no provision of transport services within a country by a foreign transport company))
    \item Special Infrastructure/State Controlled (Airports, air traffic control, SLOT)
    \item Intermodal Transport (dependency, limited ability to network)
    \item High fixed costs / perishable inventory (up to 80\% fixed costs, production and use of services are combined, external production factor)
    \item Derived demand by GDP, and income as driver for demand
    \item Deregulation has increased supply (LCC)
\end{itemize}
\underline{\textbf{Performance Metrics for Air Transport}}
\begin{itemize}
    \item $PKM = PAX \cdot KM$ (passenger km = pax times km)
    \item $TKM = Tonne \cdot KM$ (transport/tonne km)
    \item Supply: $ASK = Available~Seats \cdot KM$ (available seat km)
    \item Demand: $RPK = Seats~sold~(passengers) \cdot KM$ (revenue passenger km)
    \item $PKM = 1.852\cdot PM$ (Miles - KM)
    \item $SLF = RPK / ASK$ (seat load factor in percentage), analogous for CLF (cargo load factor)
\end{itemize}
\underline{\textbf{Air transport data - Global (IATA) 2017 - see slide 15}} \\ \\
\underline{\textbf{Air transport and COVID}}
\begin{itemize}
    \item Demand shocks normally do not have long-lasting impacts (previously shocks of RPK minus 5-20\%, but recovery after 6-18 months)
    \item RPK is depending on regions (high RPK in Asia Pacific, Europe, lower in Africa)
    \item Different markets recover at different paces (depending on vaccine availability, large markets, GDP and leisure markets)
    \item High uncertainty for prediction: 2036-2037 - uncertainties include: COVID development, Business travels, Global economy, Global security, Climate attitude
\end{itemize}
\underline{\textbf{Air transport in Europe}}
\begin{itemize}
    \item Overall increase in SLF, PKM
    \item Seasonality: Summer season 50\% more flights
    \item EU traffic mainly stays in EU
    \item Largest traffic is LDN Heathrow (77m / year) in terms of PAX, Paris in terms of post/freight
    \item Most of non-EU traffic goes to non-EU Europe (36.4\%), North America (19.8\%) or Near East (13.3\%)
    \item Strongest Airport pairs: Paris-Tolouse, Madrid-Barcelona, largest growth: Palma de Mallorca
    \item Nearly all domestic flights, as carriers work in a hub model
    \item PKM modal split within Europe: Air has 9-10\% share, passenger cars 70\%, Buses and Coaches 8\%, Railways 7\%.
    \item Share of business model within Air transport: 50\% traditional scheduled, 32\% LCC, 7 \% business aviation, 3\% Charter (diminishing due to LCC) 
\end{itemize}
\underline{\textbf{Air transport in Switzerland}}
\begin{itemize}
    \item Traffic volume in comparison to GDP: Even though share of modes is low, air transport is 17x of Swiss GDP
    \item Lines and charters: stable at 450'000 movements (starts and landings) per year
    \item Increase of PAX by 5\% over time due to high SLF and larger aircrafts
    \item Freight and post stable
    \item 34\% of movements in CH are transfers (percentage of ingoing=outgoing transfer)
    \item Nr of Aircrafts were stable over the years (commercial), private however decreasing (80\% of aircrafts in CH for sports purposes)
    \item In CH: 5.6\% of GDP, 33.5 bn CHF value and 190'000 employees 
\end{itemize}
\underline{\textbf{Emissions}}
\begin{itemize}
    \item Number of flights probably grow by 42\% from 2017 until 2040
    \item Aviation around 3.6\% of EU28 greenhouse emmissions, 13.4\% of transport
    \item Environmental Efficiancy will increase, average fuel burn per passenger by then expected to be -12\% and noise reduction by -24\%
    \item CO2 Reduction by 21\% and NOX by 16\%
    \item How are these addressed: Technology and Design, Biofuels and Synthetic Fuels, Air Traffic Management, Market based measures
\end{itemize}
\underline{\textbf{Conversions}}
\begin{itemize}
    \item Nautical Mile: 1 NM = 1.852 KM 
    \item Statute Mile: 1 SM = 1.602 km
    \item Feet: M x 3.281 = FT 
    \item Knot: 1 KT = 1 NM/H = 1.852 KM/H
\end{itemize}
\section{Aircraft Operations}
\section{Aircraft Aerodynamics}
\begin{itemize}
    \item Lift: Aerodyn. Force perpendicular to the flight vector
    \item Drag: Aeordyn. Force in opposite direction to flight vector
\end{itemize}
\begin{itemize}
    \item Standard condition: $H = 2000 m$ and $\rho = 1~kg/m^3$
\end{itemize}
\underline{\textbf{Fuel consumption}}
\begin{equation*}
    Fuel~burn~per~distance \approx \frac{SFC}{M_\infty} \frac{Weight}{Lift/Drag}
\end{equation*}
$M_\infty$ affected by aerodynamics and Engine, Lift and Drag by Aerodynamics, SFC by engine \\ \\
Equilibrum of forces: Drag=Thrust (minimse thrust), Lift = Weight (mandatory). Design goal Lift $\gg$ Drag \\ \\ Flight Performance: How far: $c_L/c_D$, How long: $c_L^3 / c_D^2$ \\ \\
\underline{\textbf{Forces}}
\begin{align*}
    F_D = c_D \frac{1}{2}\rho_\infty V_\infty^2 A~~/~~F_L = c_L \frac{1}{2}\rho_\infty V_\infty^2 A = mg = L
\end{align*}
The fuel consumption depends on the drag ``Area'': $c_D A$ \\ \\
\underline{\textbf{Eq. of motion}}
\begin{align*}
    &T cos\alpha - D - mg sin\varphi = 0 \\
    &L + T sin\alpha - mg cos \varphi = 0 \\
    &\Theta = Attitude = \alpha + \varphi = AoA + flight/climb~angle  
\end{align*}
\underline{\textbf{Why does a wing generate lift}}: The wing diverts a mass of air downwards. For this the wing acts with a force to the fluid. In return the air generates a force of the same magnitude to the wing (actio = reactio). \\ \\
\underline{\textbf{Induced Drag (Drag due to Lift)}}
\begin{align*}
    D_i &= \frac{2}{\rho V^2 \pi}\left(\frac{L}{b}\right)^2~~,~~F^* = \pi/4 b^2 (Prandtl) \\
    c_{D_i} &= \frac{1}{\pi}c_L^2 \frac{F}{b^2} = \frac{c_L^2}{\pi \Lambda} \\
    b &= Wing~Span, F=Wing~Area, \Lambda = Aspect~Ratio
\end{align*}
\begin{itemize}
    \item Induced drag ($D_i$) depends on ratio of lift and wing span 
    \item The coefficent of induced drag ($c_{D_i}$) depends on the aspect ratio 
    \item A slim wing with high aspect ratio produces less induced drag than a compact wing with small aspect ratio
    \item Induced Drag stems from the principle of linear momentum. No friction is included. The induced drag is an additional contribution to total drag 
    \item In general however, for non elliptical lift distribution, an Oswald-Factor $e$ must be considered $c_{D_{i,true}} = c_{D_{i}} / e$ 
\end{itemize}
\underline{\textbf{Tip Vortices}}
\begin{itemize}
    \item Consider Oswald factor $e$
    \item Heavy Aircraft $\rightarrow$ Strong tip vortex $\rightarrow$ High Separation Distance / Time 
\end{itemize}

\underline{\textbf{Drag}}
Total Drag = Induced Drag + Parasite Drag 
\begin{itemize}
    \item Parasite Drag is depending on the Reynolds, Mach number and some velocity regimes
    \item Influence of Reynolds number: Laminar, turbulent flow and separation. Separation mainly due to pressure drag, turbulent and laminar flow due to friction drag (small contribution)
    \item Target of small drag: No separation, turbulent downstream and large range laminar flow
    \item Incluence of Mach Number: At airspeed M $\geq$ 0.7, the flow is no longer incompressible. Additional drag occurs
    \item Lift and drag depend on the angle of attack. The polar diagram describes the dependence of the lift and drag coefficients on the angle of attack $\alpha$
    \item Following components can lead to drag: skin friction drag, induced drag, profile drag, form drag, compressibility drag, interference drag, base drag, trim drag
\end{itemize}
\underline{\textbf{Flight Performance and characteristics}}
\begin{itemize}
    \item Atmosphere: Aerodynamic Forces depend on the air density. The engine power (thrust) depends on the ambient pressure and temperature
    \item Steady thrust: $tan \varphi = \frac{T cos \alpha - D}{T sin \alpha + L}$
    \item Without thrust: $tan \varphi = \frac{-D}{L} = \frac{-c_D}{c_L} = \frac{height}{distance} $ flight path angle points downwards
    \item $\varphi$ is a measure for the aerodynamic quality of an airplane (how far it can travel from a given altitude)
\end{itemize}
\underline{\textbf{Drag at steady horizontal flight}}
\begin{align*}
    &L = mg = c_L \frac{\rho}{2} V^2 F \\
    &V = \sqrt{\frac{2mg}{\rho F c_L}} \\
    &c_L = \frac{2mg}{\rho V^2 F} \\
    &D = D_{parasite} + D_{ind} = c_{D,para} \cdot \frac{\rho}{2}V^2 F + \frac{\rho}{2} V^2 F \frac{c_L^2}{\pi \Lambda e} \\
    &= \frac{\rho}{2} V^2 F (c_{D,para} + c_{D,ind}) = K_1 V^2 + K_2 \frac{1}{V^2}
\end{align*}
The function $D(V)$ intersects with the function of Thrust $T(V)$ twice due to its shape. There are two optimal points with velocities $V_1$ and $V_2$, where there is no excess thrust or drag ($T=D$). However the stall speed is determined as follows with the maximal Lift coefficient from the polar diagram:
\begin{equation*}
    V_{stall} = \sqrt{\frac{2mg}{\rho F c_{L,max}}}~~~(c_{L,max} = c_{a,max} ~from~diagram)
\end{equation*}
\begin{itemize}
    \item The maximum speed (ideally $V_2$ is determined from the maximum Thrust ($V_max, horiz$))
    \item The minimum speed is the stall speed $V_{stall} > V_1$
    \item The speed range is between minimum and maximum speed
\end{itemize}
\underline{\textbf{Stability control}}
\begin{itemize}
    \item Steady flight condition $\rightarrow$ Disturbance $\rightarrow$ Answer of airplane $\rightarrow$ Flight path
    \item Example: Horizontal flight, then Pilot command: elevator deflection or gust comes, then pitching moment, then determine if stable/unstable/indifferent
    \item $alpha$, $\Delta \alpha > 0$, $-\Delta c_m$, $\frac{dc_M}{d\alpha} > 0$ (positive criterion)
    \item Lilienthal: Positive long. stability, Wright: Negative long. stability
    \item Dassault Falcon: stable, Neuron (delta wing): instable, Raffale: indifferent/neutral
\end{itemize}
\underline{\textbf{Development trends}} \\ \\
Good airplane
\begin{itemize}
    \item High lift to drag ratio (low drag)
    \item High ratio of payload to weight (low empty weight)
    \item High engine efficiency (high ratio of engine diameter to shaft power for propeller) + high compressor, combustor and turbine efficiency
\end{itemize}
Highest efficiency drivers
\begin{itemize}
    \item Engine: -15\% (high combustion efficiency, geared fan)
    \item Energy: -5 \% (no bleed air, more electric aircraft)
    \item Aerodynamics: -10 \% (Wing tip, engine integration, empennage config, Detail improvement)
    \item Structure: -5\% (Detail improvment, composites, new alloys, new joining technologies)
    \item Air traffic management
    \item New configurations revolutions vs. tube-wing evolution (curent)
    \item Electric/hybrid aircraft - new design freedom
\end{itemize}
\section{Manufacturing and Maintenance}
\underline{\textbf{Abbreviations}}:
\begin{itemize}
    \item A/C: Aircraft
    \item AMP: Aircraft Maintenance Program
    \item CAMO: Continuing Airworthiness Management Organisation
    \item CRS: Certificate of Release to Service
    \item DOA: Design Organisation Approval
    \item EASA: European Aviation Safety Agency
    \item ETOPS: Extended Range Twin Operations
    \item FAA: Federal Aviation Agency
    \item FC: Flight Cycles
    \item FH: Flight Hours
    \item IFE: In-flight entertainment
    \item MEL: Minimum Equipment List 
    \item MOE: Maintenance Organisation Equipment
    \item MSN: Manufacturer Serial Number 
    \item PFC: Pre Flight Check
    \item POA: Production Organistion Approval 
    \item STC: Supplemental Type-Certificate 
    \item TC: Type Certificate 
    \item XWB: Extra Wide Body
\end{itemize}
\end{multicols*}
\end{document}

