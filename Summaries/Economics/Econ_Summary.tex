\documentclass[8pt, landscape, fleqn]{scrartcl}
\setlength{\parindent}{0pt}
\usepackage[ngerman]{babel}
%\usepackage[applemac]{inputenc}
\usepackage[utf8]{inputenc}
\usepackage[dvips]{geometry}
\usepackage{latexsym}
\usepackage{multicol}
\usepackage{amsmath}
\usepackage{graphicx}
\usepackage{array}
\usepackage{booktabs}
\usepackage{amsmath}
\usepackage{mathtools}
\usepackage{ulem}
\usepackage{amsfonts}
\usepackage{dsfont}
\usepackage{charter} %%% Schreibart
%\renewcommand{\familydefault}{\sfdefault}



%%%%%%%%%%Paket für Chemische Formeln
\usepackage{chemformula} 
\usepackage[version=3]{mhchem}
%%%%%%%%%%%%%%%%% Farbe
\usepackage{color}

\pagestyle{plain}
\typearea{20}
\columnsep 30pt
\columnseprule .4pt
\setlength{\extrarowheight}{0.9em}

\renewcommand{\arraystretch}{0.8}

\makeatletter
\renewcommand{\section}{\@startsection{section}{1}{0mm}%
{-2\baselineskip}{0.8\baselineskip}%
{\hrule depth 0.2pt width\columnwidth\hrule depth1.5pt
width0.25\columnwidth\vspace*{1.2em}\Large\bfseries\rmfamily}}
\makeatother


\makeatletter
\renewcommand{\subsection}{\@startsection{subsection}{1}{0mm}%
{-2\baselineskip}{0.8\baselineskip}%
{\hrule depth 0.2pt width\columnwidth\hrule depth0.75pt
width0.25\columnwidth\vspace*{1.2em}\large\bfseries\rmfamily}}
\makeatother

\makeatletter
\renewcommand{\subsubsection}{\@startsection{subsubsection}{1}{0mm}%
{-2\baselineskip}{0.8\baselineskip}%
{\hrule depth 0.2pt width\columnwidth\vspace*{1.2em}\normalsize\bfseries\rmfamily}}
\makeatother

\newcommand{\Mx}[1]{\begin{bmatrix}#1\end{bmatrix}}
\begin{document}
\part*{\LARGE\textrm{Ökonomie $\hfill$ Xeno Meienberg}}
\begin{multicols*}{3}

\section{Einführung}

\subsection{Gegenstand der Ökonomie}

\begin{itemize}
    \item Mikroökonomie (2,3,5,6,7)
    \item Makroökonomie (8-11)
\end{itemize}

Die Mikroökonomie befasst sich mit wirtschaftlichen Entscheidungen der einzelnen Haushalte und Unternehmen:

\begin{itemize}
    \item Nachfrage (nach Gütern, Arbeit)
    \item Angebot (an Gütern, Arbeit)
    \item Marktgeschehen (Markformen, Marktpreise, Gleichgewichte)
\end{itemize}

Die Makroökonomie befasst sich mit gesamtwirtschaftlichen Zusammenhängen

\begin{itemize}
    \item Aussenwirtschaftstheorie und -politik 
    \item Geldtheorie und -politik 
    \item Arbeitsmarkttheorie und -politik 
\end{itemize}

Die Rolle der Ökonomie: 

\begin{itemize}
    \item Ökonomie ist eine Denkmethode
    \item ...ist eine Sozialwissenschaft
    \item ...keine eindeutige Wissenschaft
    \item beantwortet die Fragen:
    \item Warum Menschen könomische Entscheidungen treffen
    \item ...wie man aus knappen Ressourcen das Optimum herausholen kann
    \item ...dass Ziele möglichst gut erreicht werden 
    \item ...Ökonomisches Denken bedeutet die Warhnehmung von Zielkonflikten und das Auswählen von Alternativen
    \item ...Differenz zwischen Ertrag und Kosten maximiert wird
\end{itemize}

\subsection{Methodisches Vorgehen der Ökonomie}

\begin{enumerate}
    \item Feststelung eines Problems
    \item Analyse, Theorie, Modelle (Annahmen, Abstraktion, Empirische Tests)
    \item Politik (Handlungsempfehlungen)
\end{enumerate}

\subsection{Gesellschaftliche Bedeutung ökonomischer Analysen}

Grundidee: Knappe Ressourcen optimal einsetzen für grössten Nutzen (Wohlfahrt) Ressourcen sind:

\begin{itemize}
    \item Natürliche Ressourcen
    \item Human Ressources 
    \item Sachliche Ressourcen / Sachkapital
    \item Soziale Ressourcen / Spielregeln 
\end{itemize}

Eine der Hauptfragen der Ökonomie: Gegeben Potential (Ressoucenportfolio), was ist das Maxmimum an Wohlfahrt dass man erreichen kann? Die Kernfragen zu beantworten sind: 

\begin{enumerate}
    \item Was soll produziert werden?
    \item Wie sollen Güter und Dienstleistungen produziert werden?
    \item Wie und an wen sollen die produzierten Güter und Dienstleistungen verteilt werden? Wer konsumiert?
\end{enumerate}

\textbf{Transformationskuve}: Menge zweier Güter $X_1$ und $X_2$ (Outputs), die in einer Gesellschaft maximal bei gegebenen Ressourcen produziert werden können. \newline \newline
\textbf{Produktions-Effizienz}: Ein Güterbündel ist produktionseffizient, wenn es zu den minimal möglichen Kosten hergestellt wird oder wenn es zu gegebenen Kosten kein anderes Güterbündel gibt, für welches eines der beiden Güter grösser ist als möglich. Ein produktionseffizientes Güterbündel liegt auf der Transformationskuve. \newline \newline 
\textbf{Opportunitätskosten}: Ressultiert aus der Tatsache, dass Ressourcen knapp sind. Die Mehrproduktion eines Guts führt zu einer kleineren Menge eines anderen Guts. Die Minderproduktion werden Opportunitätskosten genannt. \newline \newline
\textbf{Indifferenzkurven}: Die Kurve stellt alle Outputkombinationen dar, zwischen denen die Gesellschaft (oder Individuum) indifferent ist. Je weiter die Indifferenzkurven vom Ursprung entfernt sind, umso höher ist der Nutzen (das Niveau) der jeweiligen Kurve 
\section{Haushalte und Nachfrage}

\subsection{Grundlegende Annahmen für Nachfrage- und Angebotsverhalten}

\subsection{Marktnachfrage nach Gütern und Dienstleistungen}

\subsection{Ein Modell zu Konsumentscheidungen von Haushalten}

\subsubsection{Budgetrestriktion}

\subsubsection{Indifferenzkurven}

\subsubsection{Der optimale Konsumpunkt}

\subsection{Von der optimalen Entscheidung zur individuellen Nachfragefunktion}

\subsection{Preiselastizität der Nachfrage}

\section{Angebotsverhalten und Unternehmen}

\section{Kosten-Nutzen Analyse}

\section{Analyse von Märkten}

\section{Öffentliche Güter und externe Effekte}

\section{Verhaltensökonomie}

\section{Leistungskraft und Wohlfahrt von Ökonomien}

\section{Arbeitslosigkeit}

\section{Aussenwirtschaft}

\section{Geld}


\end{multicols*}
\end{document}